\section{Unitary Similarity and Unitary Equivalence}
\subsection{Unitary matrices and the $QR$ factorization}
  \paragraph{8.}
  \begin{proof}
    $\,$\par
    (a) A complex orthogonal matrix $A$ which is real is clearly unitary. 
    Suppose $A$ is unitary. Then $A^*A=A^TA$, implying that $A^*=A^T$ and hence
    $A$ is real.\par
    (b) Note that $S^2=1$. Hence,
    \[
      A^TA = ((\cosh t)I-(i\sinh t)S)((\cosh t)I+(i\sinh t)S)=
      (\cosh^2 t)I + (\sinh^2 t)S^2 = I.
    \]
    Namely, $A(t)$ is complex orthogonal. By (a), $A(t)$ is unitary only if it
    is real. Hence $A(t)$ being unitary implies $t=0$. \par
    (c) Let $A_n=\diag(\sqrt{n+1}+i\sqrt{n},1,\dots,1)$ which is complex 
    orthogonal for each $n$. However, as $n\to\infty$, $A_n$ is not bounded.\par
    (d) First, every complex orthogonal matrix is invertible and the inverse of
    which is also complex orthogonal. Meanwhile, $I$ is complex orthogonal. 
    Finally, given complex orthogonal $A$ and $B$,
    \[
      (AB)^T(AB) = B^T(A^TA)B = I.
    \]
    Namely, $AB$ is also complex orthogonal. Hence, the set of complex 
    orthogonal matrices of a given size forms a group.\par
    (e) Since $1=\det I = \det(A^TA) = (\det A)^2$, $|\det A|=1$. Meanwhile, as
    $e$ and $1/e$ are the eigenvalues of $A(t)$, $A$ can have eigenvalues whose
    norm is not $1$.\par
  \end{proof}

  \paragraph{10.}
  \begin{proof}
    Since $(Ux)^*(Uy) = x^*U^*Uy = x^*y$, $Ux$ and $Uy$ are orthogonal iff $x$
    and $y$ are orthogonal.
  \end{proof}

  \paragraph{11.}
  \begin{proof}
    $A\inv=-A^T$ iff $-AA^T=I$ iff $(iA)(iA)^T=(-iA)(-iA)^T=I$ iff $\pm iA$ is
    orthogonal. Furthermore, $A\inv = A^{i\theta}A^T$ iff $Ae^{i\theta}A^T=I$ 
    iff $(e^{i\theta/2}A)(e^{i\theta}A)^T=I$ iff $e^{i\theta/2}A$ is orthogonal.
    When $\theta=0$, it is simply the orthogonal matrices and the skew 
    orthogonal matrices when $\theta=\pi$.
  \end{proof}

  \paragraph{12.}
  \begin{proof}
    Suppose $A=T\inv UT$ where $U$ is unitary. Then $A\inv = T\inv U\inv T=T\inv
    U^*T$ and $A^* = T^*U^*T^{-*}$. Hence, 
    \[
      U^*=TA\inv T\inv = T^{-*}A^*T^* \quad\Rightarrow\quad
      A\inv = T\inv T^{-*} A^* T^*T = (T^*T)\inv A^* (T^*T).
    \]
    Namely, $A\inv$ and $A^*$ are similar.
  \end{proof}

  \paragraph{23.}
  \begin{proof}
    Since $Q$ is unitary, $\det Q=\pm 1$. Hence $|\det A|=|\det Q|\det R=r_{11}
    \cdots r_{nn}$ as $R$ is an upper triangular matrix with nonnegative main
    diagonal entries.\par
    Meanwhile, for each $i$, $a_i=Qr_i$. Since unitary matrices are isometries,
    $\|a_i\|_2=\|r_i\|_2$. And clear that $\|r_i\|_2=\sqrt{\sum|r_{ij}|^2}\ge 
    r_{ii}$. The equality holds iff $r_{ij}=0$ for $j\ne i$ iff $a_i=r_{ii}q_i$.
    \par
    Therefore, 
    \[
      |\det A|=r_{11}\cdots r_{nn} \le \|a_1\|_2\cdots\|a_n\|_2.
    \]
    If $\|a_i\|_2\ne 0$ for every $i$, then the equality holds iff $R$ is 
    diagonal, which implies $A=QR$ has orthogonal columns. 
  \end{proof}
% end
