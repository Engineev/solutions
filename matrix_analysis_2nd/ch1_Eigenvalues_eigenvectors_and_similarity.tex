\section{Eigenvalues, eigenvectors, and similarity}
\setcounter{subsection}{-1}

\subsection{Introduction}

  \paragraph{1.}
  \begin{proof}
    Let $S=\{x\in\mathbb{R}^n\,:\,x^Tx=1\}$, which is clearly a compact subset
    of $\mathbb{R}^n$. Consider the function $f:x\mapsto x^TAx$. Since,
    \[
      \|f(x+\delta)-f(x)\| = \|(x^TA)\delta + \delta^T(Ax) + \delta^TA\delta\|
      \le K\|\delta\|
    \]
    for every $x\in\mathbb{R}$ and some fixed $K$, $f$ is continuous. Hence,
    by Weierstrass's theorem, $f$ attains its maximum value at some point $x\in 
    S$. Namely, (1.0.3) has a solution $x$. Therefore, there exists some 
    $\lambda\in\mathbb{R}$ such that $2(Ax-\lambda x)=0$, implying that every
    real symmetric matrix has at least one real eigenvalue.
  \end{proof}

  \paragraph{2.}
  \begin{proof}
    Let $S=\{x\in\mathbb{R}^n\,:\,x^Tx=1\}$ and $m$ be the maximum value of $x
    \mapsto x^TAx$ in $S$. Suppose $\lambda$ is an eigenvalue of $A$ and $u\ne0$ 
    is its associated eigenvector, then
    \[
      Au=\lambda u \quad\Rightarrow\quad 
      u^TAu=\lambda\|u\|^2 \quad\Rightarrow\quad
      (u/\|u\|)^T A (u/\|u\|) = \lambda \quad\Rightarrow\quad
      m \ge \lambda.
    \]
    Meanwhile, by the previous discussion, $m$ itself is a eigenvalue of $A$.
    Hence, it is the largest real eigenvalue of $A$.
  \end{proof}

% end

\subsection{The eigenvalue-eigenvector equation}
  \paragraph{1.}
  \begin{proof}
    It follows from
    \[
      (A^{-1}-\lambda^{-1}I)x 
      = (A^{-1}-\lambda^{-1}A^{-1}A)x 
      = \lambda^{-1}A^{-1}(\lambda I-A)x = 0.
    \]
  \end{proof}

  \paragraph{3.}
  \begin{proof}
    Since $A\in M_n(\mathbb{R})$, $u,v\in\mathbb{R}^n$ and $\lambda\in
    \mathbb{R}$, 
    \[
      Ax=\lambda x\quad\Rightarrow\quad
      Au + iAv = \lambda u + i\lambda v
    \]
    implies $Au=\lambda u$ and $Av=\lambda v$. As $x\ne 0$, at least one of $u$
    and $v$ is nonzero and therefore $A$ has a real eigenvector associated with
    $\lambda$. It can happen that only one of $u$ and $v$ is an eigenvector of 
    $A$, because if $x\in\mathbb{R}^n$, which may happen as we discussed above,
    the imaginary part of $x$ is $0$. Finally, if $x$ is a real eigenvector of 
    $A$, then the eigenvalue $\lambda$ it associated with must be real. 
    Otherwise, at least one entry of $\lambda x$ is not real as $x\ne 0$, 
    contradicting with the fact that $Ax$ is real. 
  \end{proof}

  \paragraph{5.}
  \begin{proof}
    Let $p(t)=t^2-t$. Since $A$ is idempotent, $p(A)=A^2-A=0$. Hence, $0$ is the
    only eigenvalue of $p(A)$. By Theorem 1.1.6, the only values the eigenvalues
    of $A$ can be are the zeros of $p$, namely, $0$ and $1$. \par
    Suppose $A$ is nonsingular, then multiplying $A^{-1}$ on the both sides of
    $A^2=A$ yields $A=I$.
  \end{proof}

  \paragraph{7.}
  \begin{proof}
    Suppose $\lambda\in\sigma(A)$ and $x$ is its associated eigenvector, then
    \begin{align*}
       & 0 = (A-\lambda I)x = x^*(A^*-\bar{\lambda}I) = x^*(A-\bar{\lambda}I) \\
      \Rightarrow\quad&
      0 = x^*(A-\bar{\lambda}I)x = x^*Ax - \bar{\lambda}x^*x = 
      (\lambda-\bar{\lambda}) \|x\|^2.
    \end{align*}
    Hence, $\lambda=\bar{\lambda}$, implying all eigenvalues of $A$ are real.
  \end{proof}

  \paragraph{9.}
  \begin{solution}
    Solve the equation $\det(A-\lambda I) = 0$ and we get $\lambda=\pm i$.
  \end{solution}

  \paragraph{11.}
  \begin{proof}
    If $\rank(A-\lambda I)<n-1$, then $\adj(A-\lambda I)=0$ by (0.8.2) and 
    therefore we can always choose $y$ to be the $0$ and the other parts of the 
    proposition clearly hold. Hence, in the following discussion, we assume that
    $\rank(A-\lambda I)=n-1$. \par
    Apply the full-rank factorization and we get $\adj(A-\lambda I)=\alpha xy^*$
    for some nonzero $\alpha\in\mathbb{C}$ and $x,y\in\mathbb{C}^n$. Replacing 
    $x$ with $\alpha x$ and $\alpha$ with $1$ proves the first part.\par
    Suppose $\adj(A-\lambda I)=[\beta_1,\dots,\beta_n]$, then
    \[
      (A-\lambda I)\adj(A-\lambda I) \quad\Rightarrow\quad
      (A-\lambda I)\beta_k = 0 \quad(k=1,2,\dots,n),
    \]
    implying that $\beta_k$ is an eigenvector of $A$ associated with $\lambda$ 
    as long as it is nonzero.
  \end{proof}

  \paragraph{13.}
  \begin{proof}
    If $\rank A < n -1$, then $x$ is always an eigenvector of $\adj A$ 
    associated with $0$ as $\adj A = 0$. Hence, we may assume that $\rank = n - 
    1$. Then $\adj A = (\det A)A^{-1}$. By Exercise 1, $x$ is an eigenvector of 
    $A^{-1}$ and therefore an eigenvector of $\adj A$.
  \end{proof}
  
% end

\subsection{The characteristic polynomial and algebraic multiplicity}
  \paragraph{2.}
  \begin{proof}
    Suppose $A = [a_{ij}]_{m,n} = [\alpha_1,\dots,\alpha_n]^T$ and $B = 
    [b_{ij}]_{n,m} = [\beta_1,\dots,\beta_n]$, then
    \[
      \tr(AB)=\sum_{i=1}^n\alpha_i\beta_i
      =\sum_{i=1}^n\sum_{j=1}^ma_{ij}b_{ji}
      =\sum_{j=1}^m\sum_{i=1}^nb_{ji}a_{ij}
      =\tr(BA).
    \]
    Hence, for nonsingular $S\in M_n$, $\tr(S^{-1}AS)=\tr(S(S^{-1}A)) = \tr(A)$.
    \par For $A\in M_n$, $\det(S^{-1}AS) = \det(S)\det(S^{-1})\det(A)=\det(A)$,
    which means the determinant function on $M_n$ is similarity invariant.
  \end{proof}

  \paragraph{4.}
  \begin{proof}
    It follows immediately from the fact that $\sigma(A)\subset\{0, 1\}$ and 
    $S_k(A)$ is the sum of some $\prod\lambda_{i_j}$. 
  \end{proof}

  \paragraph{6.}
  \begin{proof}
    $\rank(A-\lambda I)=n-1$ implies the matrix $A-\lambda I$ is singular, and
    therefore $\lambda$ is an eigenvalue of $A$. However, it may not have 
    multiplicity $1$. For example\footnote{Thanks to Zhihan Jin, one of my 
    classmates.}, suppose $A=\begin{bsmallmatrix} 1 & -1 \\ 1 & -1 
    \end{bsmallmatrix}$. $\rank A = 1$ but $0$, the only eigenvalue of
    $A$ is of multiplicity $2$.
  \end{proof}

  \paragraph{8.}
  \begin{proof}
    $p_{A+\lambda I}(t)=\det(tI-(A+\lambda I))=\det((t-\lambda)I-A)=p_A(t-
    \lambda)$ and hence the eigenvalues of $A+\lambda I$, the zeros of $p_{A+
    \lambda}(t)$, are $\lambda_1+\lambda,\dots, \lambda_n+\lambda$.
  \end{proof}

  \paragraph{10.}
  \begin{proof}
    Since $p_A(t)$ has $n$ roots and non-real roots of a polynomial come in 
    paris, at least one of the roots is real. Hence, $A$ has at least one real
    eigenvalue.
  \end{proof}

  \paragraph{12.} TODO
    
  \paragraph{14.}
  \begin{proof}
    Suppose $C=\begin{bsmallmatrix} \mu & 0 \\ * & B \end{bsmallmatrix}$. By the
    exercise on p52,
    \[
      p_A(t) = (t-\lambda)p_C(t) = (t-\lambda)p_{C^T}(t) 
      = (t-\lambda)(t-\mu)p_B(t).
    \]
  \end{proof}

  \paragraph{16.}
  \begin{proof}
    $f(t)=\det(A+(tx)y^T = \det A + y^T(\adj A)tx = \det A + t\beta$ where 
    $\beta = y^T(\adj A)x$, a constant independent of $t$. Hence, for $t_1\ne
    t_2$
    \[
      \frac{t_2f(t_1)-t_1f(t_2)}{t_2-t_1} 
      = \frac{t_2(\det A + t_1\beta) - t_1(\det A + t_2\beta)}{t_2-t_1}
      = \det A.
    \]
    For the second part, we can get from calculation that
    \[
      f(-b)=\det(A-b[1,\dots,1]^T[1,\dots,1]) = (d_1-b)\cdots(d_n-b)=q(b)
    \]
    and $f(-c)=q(-c)$. Hence, if $b\ne c$,
    \[
      \det A = \frac{(-c)f(-b)- (-b)f(-c)}{(-c)-(-b)}=\frac{bq(c)-cq(b)}{b-c}.
    \]
    Now suppose $b=c$. Note that $f(t)$ is a linear function of $t$, which is 
    differentiable, implying that 
    \[
      \det A = \lim_{t_2\to t_1}\frac{t_2f(t_1)-t_1f(t_2)}{t_2-t_1} 
      = f\hp(t_1)t_1 - f(t_1).
    \]
    Meanwhile, since $q(t)$ is continuous, $q(t)\to f(-b)$ as $t\to b$. Thus,
    \[
      \det A = \lim_{c\to b}\frac{(-c)f(-b)- (-b)f(-c)}{(-c)-(-b)}
      = q(b) - bq\hp(b).
    \]
    Let 
    \[
      A_* = \lambda I - A = 
      \begin{bmatrix}
        \lambda & -b      & \cdots & -b \\
        -c      & \lambda & \ddots & \vdots \\
        \vdots  & \ddots  & \ddots & -b \\
        -c      & \cdots  & -c     & \lambda 
      \end{bmatrix}.
    \]
    and $q_*(t)=(\lambda - t)^n$, then by the previous result,
    \begin{align*}
      p_A(\lambda) &= \frac{-bq_*(-c) - (-c)q_*(-b)}{-c-(-b)} 
                    = \frac{b(\lambda+c)^n - c(\lambda+b)^n}{b-c}, 
                      & \text{if } b\ne c, \\ 
      p_A(\lambda) &= q_*(-b)-(-b)q_*\hp(-b) 
                    = (\lambda+b)^{n-1}(\lambda-(n-1)b),
                      & \text{if } b=c. 
    \end{align*}
  \end{proof}

  \paragraph{18.}
  \begin{proof}
    The identity can be derived immediately from Observation 1.2.4 and the 
    identity $a_1 = (-1)^{n-1}\tr\adj(A)$, the proof of which can be found on 
    p53.
  \end{proof}

  \paragraph{20.}
  \begin{proof}
    By (1.2.13), 
    \[
      \det(I+A) = (-1)^n p_A(-1) = 
      (-1)^n\left((-1)^n + \sum_{k=1}^n(-1)^{n-k}E_k(A)(-1)^k \right)
      = 1 + \sum_{k=1}^n E_k(A).
    \]
  \end{proof}

  \paragraph{22.} 
  \begin{proof}
    Suppose
    \[
      A = 
      \begin{bmatrix}
        t    & -1     &        & 0 \\
             & \ddots & \ddots &   \\
             &        & \ddots & -1 \\
             &        &        & t
      \end{bmatrix},
    \]
    then
    \begin{align*}
      p_{C_n(\vep)}(t)
      &= \det(A + [0,\dots,0,1]^T [-\vep,0,\dots,0]) \\
      &= \det A - \vep [1, 0,\dots,0](\adj A) [0,\dots,0,1]^T \\
      &= \det A - \vep ((\adj A)[1,n]) \\
      &= \det A - \vep \det A[\{n\}^c, \{1\}^c] \\
      &= t^n - \vep.
    \end{align*}
    And its spectrum, namely the set of roots of $p_{C_n(\vep)}$, is $\{
    \vep^{1/n} e^{2\pi ik/n}\,:\,k=0,1,\dots,n-1\}$. Hence,
    \[
      \rho(I+C_n(\vep)) = 1 + \rho(C_n(\vep)) = 1+\vep^{1/n}.
    \]
  \end{proof}  

% end

\subsection{Similarity}
  \paragraph{1.}
  \begin{proof}
    (a) Since $A$ and $B$ are diagonalizable and commute, by Theorem 1.3.21, 
    they are simultaneously diagonalizable. Hence, there exists some nonsingular
    $S\in M_n$ such that
    \begin{align*}
      A+B &=
      S\inv\diag(\lambda_1,\dots,\lambda_n)S 
      + S\inv\diag(\mu_{i_1},\dots,\mu_{i+n }) S \\
      &= S\inv\diag(\lambda_1+\mu_{i_1},\dots, \lambda_n+\mu_{i_n})S.
    \end{align*}
    Therefore, $\sigma(A+B)=\{\lambda_1+\mu_{i_1},\dots, \lambda_n+\mu_{i_n}\}$.
    \\ (b) By Exercise 1.1.6, $\sigma(B)=\{0\}$, completing the proof. \\
    (c) $\sigma(AB) = \{\lambda_1\mu_{i_1},\dots,\lambda_n\mu_{i_n}\}$, because
    \[
      S\inv(AB)S = (S\inv AS)(S\inv BS) = 
      \diag(\lambda_1\mu_{i_1},\dots,\lambda_n\mu_{i_n}).
    \]
  \end{proof}

  \paragraph{2.}
  \begin{proof}
    Suppose that $p(z)=\sum_{i=0}^n a_iz^i$ and $q(z)=\sum_{j=0}^m b_jz^j$, then
    \begin{align*}
      p(A)q(B) 
      = \left(\sum_{i=0}^n a_iA^i\right)\left(\sum_{j=0}^n b_jB^j\right)
      = \sum_{i,j}a_ib_jA^iB^j
      = \sum_{i,j}a_ib_jB^jA^j
      = q(B)p(A).
    \end{align*}
      
  \end{proof}

  \paragraph{3.}
  \begin{proof}
    \[
      \sum_{k=0}^n a_k A^k = \sum){k=0}^n S\inv (a_k\Lambda^k)S
      = S\inv p(\Lambda)S.
    \]
  \end{proof}

  \paragraph{4.}
  \begin{proof}
    First we assume that $A=\diag(\alpha_1,\dots,\alpha_n)$. Since $AB=BA$, by
    (0.7.7), $B$ is also diagonal matrix. Suppose that $B=\diag(\beta_1,\dots,
    \beta_n)$ where $b_i$ may coincide. Then $p(A)=B$ is equivalent to 
    $p(\alpha_i)=\beta_i$ for $i=1,\dots,n$. Since $\alpha_i$ are distinct, we 
    can construct such a polynomial by interpolation.\par
    For the general case, note that in the proof of Theorem 1.3.12, $n_i=1$ as
    long as $\alpha_i$ are distinct. Hence, $A$ and $B$ are simultaneously 
    diagonalizable. Suppose that $A=S\diag(\alpha_1,\dots,\alpha_n)S\inv$ and
    $B=S\diag(\beta_1,\dots,\beta_n)S\inv$. Let $p$ be the same polynomial as in
    the last paragraph. Then by P3,
    \[
      p(A)=Sp(\diag(\alpha_1,\dots,\alpha_n))S\inv = 
      S\diag(\beta_1,\dots,\beta_n)S\inv = B.
    \]
  \end{proof}

  \paragraph{5.}
  \begin{proof}
    Let $A=\begin{bsmallmatrix} 0 & 1 \\ 0 & 0\end{bsmallmatrix}$ and $B=I$.
    Clear that $A$ and $B$ commute but are not simultaneously diagonalizable 
    since $A$ can not be diagonalized. This does not violate 1.3.12 because
    in 1.3.12, $A$ and $B$ are required to be diagonalizable.
  \end{proof}

  \paragraph{6.}
  \begin{proof}
    (a) It follows immediately from the fact that the multiplication of block
    diagonal matrices are block-wise.\\
    (b) Suppose that $A=S\Lambda S\inv$, then
    \[
      p_A(t)=\det(tI-A)=\det(tSS\inv - S\Lambda S\inv) = \det(tI-\Lambda)=
      p_\Lambda(t).
    \]
    By P3, $p_\Lambda(A)=Sp_\Lambda(\Lambda)S\inv$. Therefore, $p_A(A)=p_\Lambda
    (A)=Sp_\Lambda(\Lambda)S\inv=0$.
  \end{proof}

  \paragraph{7.}
  \begin{proof}
    Suppose $B=\diag(b_1,\dots,b_n)$, then clear that $A=\diag(\sqrt{b_1},\dots,
    \sqrt{b_n})$ is a square root of $B$.\par
    Assume that such $A$ exists, then by Theorem 1.1.6, $\sigma(A)=\{0\}$, 
    implying that 
    \[
      A = \begin{bmatrix}
        0 & a \\ b & 0        
      \end{bmatrix}.
    \]
    Therefore,
    \[
      A^2 = \begin{bmatrix}
        ab & 0 \\ 0 & ab
      \end{bmatrix} \ne B.
    \]
    Contradiction.
  \end{proof}

  \paragraph{8.}
  \begin{proof}
    If $A$ and $B$ are simultaneously diagonalizable, then clear that $A$ and 
    $B$ commute. Now we suppose that $A$ and $B$ commute and $\lambda_1,\dots,
    \lambda_n$ are the distinct eigenvalues of $A$. Since for each $i$, the 
    eigenspace $E(A,\lambda_i)$ is an one-dimensional invariant subspace of $B$,
    then every vector in $E(A,\lambda_i)$ is also an eigenvector of $B$. Hence, 
    there exists a basis of $\mathbb{C}^n$ consisting of the common eigenvectors
    of $A$ and $B$, implying that they are simultaneously diagonalizable.
  \end{proof}

  \paragraph{9.}
  \begin{proof}
    By Theorem 1.3.22, $AB$ and $BA$ have the same eigenvalues. And as similar
    matrices are of the same rank, $AB$ and $BA$ are not similar.
  \end{proof}

  \paragraph{10.}
  \begin{proof}
    We argue by contradiction. Assume that there exists some vector in the list
    which belongs to the span of the previous vectors and suppose $x^{(p)}_q$ is
    the first such vector. Then it equals to some linear combination of the 
    previous vectors. Compute $Ax^{(p)}_q$ using the linearity first and then 
    using the fact that $x^{(p)}_q$ is an eigenvalues. Compare the two formula
    and we will obtain a contradiction.
  \end{proof}

  \paragraph{11.}
  \begin{proof}
    Suppose that $A,B\in M_n$ commute and $Ax=\lambda x$ where $x\ne 0$ and $k$
    the is the smallest integer such that $B^kx \in \text{span}\{x,Bx,\dots,
    B^{k-1}x\} = \mathcal{S}$. For every $u=\sum_{i=0}^{k-1} x_iB^ix \in
    \mathcal{S}$, $Bu$ is a linear combination of $\sum_{i=0}^{k-2} x_iB^{i+1}x$
    and $x_{k-1}B^kx \in\mathcal{S}$. Hence $Bu\in\mathcal{S}$ and therefore 
    $\mathcal{S}$ is $B$-invariant. By Observation 1.3.18, there exists some 
    $0\ne y\in\mathcal{S}$ which is an eigenvector of $B$. Meanwhile, since
    \[
      A\sum_{i=0}^{k-1} x_iB^i x = \sum_{i=0}^{k-1} x_i(AB^i)x =
      \sum_{i=0}^{k-1} x_i B^i\lambda x = \lambda\sum_{i=0}^{k-1} x_iB^i x,
    \]
    every nonzero vector in $\mathcal{S}$ is a eigenvector of $A$ and so does 
    $y$. Hence, $A$ and $B$ have a common eigenvector $y$.\par
    Now we argue by induction on $m$, the size of the finite commuting family 
    $\mathcal{F}=\{A_1,\dots,A_m\}$. Suppose that $y\ne 0$ is a common 
    eigenvector of $A_1,\dots,A_{m-1}$ and let $k$ be the smallest integer such
    that $A^k_my\in \text{span}\{y,A_my,\dots, A_m^{k-1}y\}=\mathcal{S}$. Then 
    by some argument similar to the previous one, $\mathcal{S}$ is 
    $A_m$-invariant and hence contains a eigenvector $z$ of $A_m$. Meanwhile, 
    since $A_i$ and $A_m$ commute, every nonzero vector in $\mathcal{S}$ is an
    eigenvector of $A_i$ for $i=1,\dots,m-1$ and so does $z$, concluding that 
    matrices in a finite commuting families share a common eigenvector.\par
    $M_n$ is linear space of dimension $n^2$ and $\mathcal{F}$ is a subspace of 
    $M_n$ since for any $A,B,C\in\mathcal{F}$ and $a,b\in\mathbb{C}$,
    \[
      (aA+bB)C = a(AC) + b(BC) = a(CA) + b(CB) = C(aA+bB).
    \]
    Let $\mathcal{B}=\{B_1,\dots,B_k\}$ be a basis of $\mathcal{F}$. Since 
    $\mathcal{B}$ is finite and commuting, (b) shows that the matrices in 
    $\mathcal{B}$ have a common eigenvector $x$. Hence, supposing $B_ix=
    \lambda_i x$,
    \[
      \left(\sum_{i=1}^k b_iB_i\right)x = \sum_{i=1}^k b_i(B_ix) = 
      \left(\sum_{i=1}^k b_i\lambda_i\right)x
    \]
    where $b_i$ are some scalars. Thus, $x$ is a eigenvector of every $A\in
    \mathcal{F}$.
  \end{proof}

  \paragraph{12.}
  \begin{proof}
    Suppose that $A$ is nonsingular. Then $BA=A\inv(AB)A$, i.e., $BA\sim AB$.
    Therefore $BA$ is diagonalizable as long as $AB$ is. We can produce the same
    result with a similar argument if $B$ is nonsingular.\par
    Suppose $A=\begin{bsmallmatrix}0 & 1 \\ 0 & 0\end{bsmallmatrix}$ and $B=
    \begin{bsmallmatrix} 1 & 1 \\ 0 & 0\end{bsmallmatrix}$. Then $AB=0$, which 
    is a diagonal matrix and $BA=\begin{bsmallmatrix} 0 & 1 \\ 0 & 0
    \end{bsmallmatrix}$, which is not diagonalizable since all eigenvectors of 
    $BA$ are of form $[k, 0]^T$.
  \end{proof}

  \paragraph{13.}
  \begin{proof}
    Since similar diagonalizable matrices have the same eigenvalues and 
    multiplicities, their characteristic polynomials are therefore the same and
    vice versa.\par
    However, this is not true for two matrices which are not both 
    diagonalizable. For example, $\begin{bsmallmatrix} 0 & 1 \\ 0 & 0 
    \end{bsmallmatrix}$ and $0_{2\times 2}$ have the same characteristic 
    polynomial but they are not similar.
  \end{proof}

  \paragraph{14.}
    This exercise provides several ways to prove a matrix to be not 
    diagonalizable.
  \begin{proof}
    (a) It follows immediately form the fact that the rank of a matrix is 
    similarity invariance. \\
    (b) It holds for diagonal matrices. And since rank is invariant under 
    similarity, it holds for all diagonalizable matrices.\\
    (c) A matrix is nilpotent iff it is similar to a matrix whose nonzero 
    entries are all above the main diagonal. Hence a diagonalizable matrix is
    nilpotent iff it is similar to $0$ iff it equals to $0$.\\
    (d) $\tr A=0$ implies the nonzero eigenvalues of $A$ comes in $\pm$ paris.
    Hence, $\rank A$ is even.\\
    (e) $\rank B=1$ but it has no nonzero eigenvalue. Hence, it is not 
    diagonalizable by (a). Since $\rank B^2=0\ne \rank B$, $B$ is not 
    diagonalizable by (b). $B$ is a nonzero nilpotent matrix and therefore is 
    not diagonalizable by (c). $\tr B=0$ but $\rank B=1$. Hence $B$ is not 
    diagonalizable by (d).
  \end{proof}

  \paragraph{15.}
  \begin{proof}
    Suppose that $A=S\Lambda S\inv$ where $\Lambda$ is a diagonal matrix, then
    \[
      p(A) = \sum_{k=0}^n a_k(S\Lambda S\inv)^k = 
      \sum_{k=0}^n a_kS\Lambda^kS\inv = Sp(\Lambda)S\inv.
    \]
    Clear that $p(\Lambda)$ is again a diagonal matrix. Hence, $p(A)$ is also
    diagonalizable. \par
    However the converse is not true. For example, $A = \begin{bsmallmatrix} 
    0 & 1 \\ 0 & 0 \end{bsmallmatrix}$ is not diagonalizable but $A^2 = 0$ 
    itself is a diagonal matrix.
  \end{proof}

  \paragraph{17.}
  \begin{proof}
    Suppose that $A=TBT\inv$ where $T\in M_n(\mathbb{R})$ is nonsingular, then
    $\bar{A}=\overline{TBT\inv} = \bar{T}\bar{B}\bar{T}\inv = T\bar{B}T\inv$ 
    since $T$ is real. And the converse is obviously true.
  \end{proof}

  \paragraph{19.}
  \begin{proof}
    Clear that $Q=Q^T$ and $Q^2 = I$, implying $Q=Q\inv$.\\
    (a) Suppose that $A=\begin{bsmallmatrix} A_{11} & A_{12} \\ A_{21} & A_{22}
    \end{bsmallmatrix}$. 
    \[
      0=K_{2n}A - AK_{2n} = 
      \begin{bmatrix}
        A_{21} - A_{12} & A_{22}-A_{11} \\ A_{11}-A_{22} & A_{12} - A_{21}
      \end{bmatrix}.
    \]
    Hence, $A$ is $2$-by-$2$ block centrosymmetric. And the proof of the 
    converse is trivial. If $A$ is nonsingular, then we have $A\inv K_{2n}=
    K_{2n}A\inv$, which implies $A\inv$ is $2$-by-$2$ block centrosymmetric. 
    Meanwhile, since $K_{2n}\inv = K_{2n}$, $K_{2n}AK_{2n} = A$. Suppose $B$ is 
    a $2$-by-$2$ block centrosymmetric matrix, then
    \[
      K_{2n}AB = K_{2n}A(K_{2n}BK_{2n}) = (K_{2n}AK_{2n})B_{2n}K_{2n} 
      = ABK_{2n}.
    \]
    Therefore, $AB$ is a $2$-by-$2$ block centrosymmetric matrix as well.\\
    (b)
    \begin{align*}
      Q\inv AQ &= \frac{1}{2}
      \begin{bmatrix} I_n & I_n \\ I_n & -I_n \end{bmatrix}
      \begin{bmatrix} B & C \\ C & B \end{bmatrix}
      \begin{bmatrix} I_n & I_n \\ I_n & -I_n \end{bmatrix} \\
      &= \begin{bmatrix} B+C & 0 \\ 0 & B+C \end{bmatrix} 
        = (B+C) \oplus (B-C).
    \end{align*} 
    (c) 
    \[
      \det A = \det(Q\inv AQ) = \det\begin{bmatrix}
        B+C & 0 \\ 0 & B+C
      \end{bmatrix} = \det(B^2+CB - BC-C^2)
    \]
    and $\rank A =\rank(B+C)+\rank(B-C)$ follows immediately from $Q\inv AQ=
    (B+C)\oplus(B-C)$.\\
    (d) $Q\inv \begin{bsmallmatrix} 0 & C \\ C & 0 \end{bsmallmatrix} = C\oplus
    (-C)$. Since $p_{C\oplus(-C)}(t)=p_C(t)p_{-C}(t)$, the eigenvalues occur in 
    $\pm$ pairs.
  \end{proof}

  \paragraph{20.}
  \begin{proof}
    (b) As $A$ is nonsingular, $A\inv A=I_n$ and therefore $R_1(A\inv)R_1(A)=
    R_1(A\inv A)=I_{2n}$ by (a). Hence, $R_1(A)$ is nonsingular and $R_1(A)\inv=
    R_1(A\inv)$, which also implies that $R_1(A)\inv$ has the same block 
    structure as $R_1(A)$. \\
    (g) By (f), $R_1(A)$ is similar to $A\oplus\bar{A}$. Therefore, $\sigma(R_1(
    A))=\{\lambda_1,\dots,\lambda_n,\bar{\lambda}_1,\dots,\bar{\lambda}_n\}$.\\
    (h) By (f), $\det R_1(A)=\det(A\oplus\bar{A})=|\det A|^2\ge 0$. Since $\rank
    (A\oplus\bar{A})=2\rank A$ and rank is invariant under similarity, $\rank 
    R_1(A)=2\rank A$.\\
    (i) It follows immediately from (h).
  \end{proof}

  \paragraph{23.}
  \begin{proof}
    Suppose that there exists some $X\in M_{n,m}$ such that $C=BX$ and let 
    $S=\begin{bsmallmatrix} I_n & X \\ 0 & I_m \end{bsmallmatrix}$. Clear that
    $S$ is nonsingular and $S\inv=\begin{bsmallmatrix} I_n & -X \\ 0 & I_m 
    \end{bsmallmatrix}$. Since
    \[
      S\begin{bmatrix}B & BX \\ 0_n & 0_m\end{bmatrix}S\inv = 
      \begin{bmatrix} B & 0 \\ 0 & 0_m\end{bmatrix},
    \]
    $A$ is similar to $B\oplus 0_m$.\par
    Now we suppose that $A$ is similar to $B\oplus 0_m$. Since similar matrices
    have the same rank, $\rank[B\,C] = \rank B$.    
  \end{proof}

  \paragraph{28.}
  \begin{proof}
    \begin{align*}
      \det(I_m+AB) 
      &= \det\begin{bmatrix}
        I_m+AB & A \\ 0 & I_n
      \end{bmatrix} \\
      &= \det\begin{bmatrix}
        I_m+AB & A \\ B+BAB & I_n + BA 
      \end{bmatrix} \\
      &= \det\begin{bmatrix}
        I_m & A \\ 0 & I_n+BA
      \end{bmatrix} \\
      &= \det (I_n+BA).
    \end{align*}
  \end{proof}

  \paragraph{29.}
  \begin{proof}
    Since $\det A=\sum_\sigma(\text{sgn}\sigma \prod_{i=1}^n a_{i\sigma(i)})$, 
    where $\sigma$ is any permutation of $\{1,2,\dots, n\}$, the determinant of 
    a matrix whose entries are integers is an integer.\par
    Suppose that the $a_{ij}$ is changed from $-1$ to $1$ and denote the new 
    matrix by $\tilde{A}$. Let $x,y\in\mathbb{C}^n$ be two vectors such that 
    $x_i=1$ and $y_j=2$, then $\tilde{A} = A + xy^T$. By Cauchy's identity,
    \[
      \det\tilde{A} = \det A + y^T(\adj A)x = \det A + 2\det A[\{i\}^c,\{j\}^c].
    \]
    Hence, the parity of $\det A$ is unchanged.\par
    \footnote{I don't know what $J_n$ actually is here and assume it to be the 
    matrix whose entries are all $1$.} Since changing a $-1$ entry to $1$ does 
    not change the parity of the determinant, we can change all the entries to 
    $1$. Hence, the parity of $\det A$ is the same as the parity of $\det(J_n-
    I)$. Induction on $n$ yields that it is opposite to the parity of $n$. Thus,
    if $n$ is even, then $\det A$ is odd and therefore nonzero, implying $A$ is 
    nonsingular.
  \end{proof}

  \paragraph{30.}
  \begin{proof}
    By Theorem 1.3.27, there exists some nonsingular $R=\diag(r_1,\dots,r_n)$
    such that $T=SR$, Hence,
    \[
      Tf(\Lambda)T\inv = SRf(\Lambda)R\inv S\inv = Sf(\Lambda)S\inv
    \]
    as $f(\Lambda)$, a diagonal matrix, commute with every matrix. Therefore,
    $\cos^2A+\sin^2A=I$ as $\cos^2x+\sin^2x=1$ for every $x\in\mathbb{R}$.
  \end{proof}

  \paragraph{31.}
  \begin{proof}
    The characteristic polynomial of the matrix is $p(t)=(t-a)^2+b^2=
    (t-a-ib)(t-a+ib)$. Hence its eigenvalues are $a\pm ib$.
  \end{proof}

  \paragraph{33.}
  \begin{proof}
    $\,$\\
    (a) Since $A$ is real, $A\bar{x} = \overline{\bar{A}x} = \overline{Ax} = 
    \overline{\lambda x} = \bar{\lambda}\bar{x}$.\\
    (b) Since $\lambda$ is not real, $x$ and $\bar{x}$ are associated with
    different eigenvalues. Hence, $x$ and $\bar{x}$ are linear independent.
    Suppose that $mu+nv=0$. Then
    \[
      0= m(x+\bar{x})-in(x-\bar{x}) = (m-in)x + (m+in)\bar{x}.
    \]
    Since $m-in=0$ and $m+in=0$, $m=n=0$. Thus, $u$ and $v$ are also linear 
    independent.\\
    (c) 
    \[
      Au = \frac{1}{2}A(x+\bar{x})=\frac{1}{2}(\lambda x+\bar{\lambda}\bar{x})
      = \frac{1}{2}[(a+ib)(u+iv)+(a-bi)(u-iv)] = au-bv.
    \]
    Similarly, $Av=bu+av$. Hence,
    \[
      A[u, v] = [Au, Av] = [au-bv, bu+av] = [u, v]B.
    \]\\
    (d) Since $S\begin{bsmallmatrix} I_2 \\ 0 \end{bsmallmatrix} = [u\,v]$, 
    $S\inv[u\,v]=\begin{bsmallmatrix} I_2 \\ 0 \end{bsmallmatrix}$ and the proof
    of the next result is trivial. \\
    (e) Since $p_A(t) = p_B(t)p_{A_1}(t)$, the result amounts to the fact that
    $\lambda$ and $\bar{\lambda}$ are two roots of $p_B(t)$.
  \end{proof}
  

% end

\subsection{Left and right eigenvectors and geometric multiplicity}
  \paragraph{2.}
  \begin{proof}
    Since
    \[
      p_A(t)=p_{A^T}(t)=p_{-A}(t)=\det(tI+A)=(-1)^n\det((-t)I-A)=(-1)^np_A(-t),
    \]
    $-\lambda$ is an eigenvalue of $A$ with multiplicity $k$ as long as 
    $\lambda$ is. Thus, if $n$ is odd, $0\in\sigma(A)$ as the nonzero eigenvalues 
    come in pairs. Hence, $A$ is singular. Since the principal submatrices of a 
    skew symmetric matrix are still skew symmetric, the ones with odd size are
    singular. Finally, since skew symmetric matrices are rank principal, having
    a nonsingular principal submatrix of size $r\times r$ if its rank is $r$, 
    $\rank A$ is even.
  \end{proof}

  \paragraph{4.}
  \begin{proof}
    Note that $S\inv = S$ and multiplying $S$ on the left and right are 
    respectively equivalent to changing the sign of the odd rows and columns.
    Hence, $S\inv AS = SAS = -A$. Since $p_{-A}(t)=(-1)^np_A(-t)$ and similar
    matrices have the same eigenvalues with the same multiplicities, $-\lambda$
    is an eigenvalue of $A$ with multiplicity $k$ as long as $\lambda$ is. Since
    the eigenvalues of $A$ come in $\pm$ paris, $0\in\sigma(A)$, hence $A$ is 
    singular, if $n$ is odd.
  \end{proof}

  \paragraph{6.}
  \begin{proof}
    (a) Since $x$ and $y$ are entrywise positive, $y^*x>0$. Hence, by Theorem 
    1.4.7, the subspace $\text{span}(x)$ and the orthogonal complement $W$ of 
    $y^*$ are both $A$-invariant. Let $u$ be an entrywise nonnegative right 
    eigenvector of $A$ associated with eigenvalue $\mu$. It belongs to some 
    $A$-invariant subspace. Since $y^*u>0$, it does not belong to $W$ and 
    therefore $u \in\text{span}(x)$. Thus, $\mu = \lambda$. If $u$ is a left 
    eigenvector, the argument is similar. \\
    (b) Since the algebraic multiplicity is no less than the geometric 
    multiplicity, $\lambda$ has geometric multiplicity $1$.
  \end{proof}

  \paragraph{10.}
  \begin{proof}
    Suppose $T=[\beta_1,\dots,\beta_n]$. Then
    \[
      T^*A = 
      \begin{bmatrix}
        \beta_1^* \\ \vdots \\ \beta_n^*
      \end{bmatrix}
      A =
      \begin{bmatrix}
        \beta_1^* A \\ \vdots \\ \beta_n^* A
      \end{bmatrix}
      =
      \begin{bmatrix}
        \lambda_1\beta_1^* \\ \vdots \\ \lambda_n\beta_n^*
      \end{bmatrix}
      = \Lambda T^*.
    \]
    where $\Lambda = \diag(\lambda_1,\dots,\lambda_n)$. Therefore, $AT^{-*}=
    T^{-*}\Lambda$, implying that $T^{-*}$ are the right eigenvectors of $A$.
  \end{proof}

  \paragraph{12.}
  \begin{proof}
    \noindent 
    (a) First suppose that every list of $n-1$ columns of $A-\lambda I=[\beta_1,
    \dots,\beta_n]$ is linearly independent and let $x\ne 0$ be an eigenvector 
    of $A$ associated with $\lambda$. We argue by contradiction, assuming that 
    $x_i$, the $i$-th entry of $x$, is zero. Then 
    \[
      0 = (A-\lambda)x = 
      x_1\beta_1 + \cdots + x_{i-1}\beta_{i-1} + x_{i+1}\beta_{i+1} + \cdots +
      x_n\beta_n.
    \]
    As $\beta_1,\dots,\beta_{i-1},\beta_{i+1},\dots,\beta_n$ are linearly 
    independent, this implies $x_i=0$ for each $i=1,\dots,n$, contradicting with
    the assumption $x\ne 0$. \\
    To prove the converse part, we continue to argue by contradiction and assume
    that $\beta_1,\dots,\beta_{i-1},\beta_{i+1},\dots,\beta_n$ are linearly 
    dependent. Consider the equation $[\beta_1,\dots,\beta_n]x = 0$. Even if we 
    restrict the $i$-th entry of $x$ to be $0$, the equation is still solvable,
    leading to a contradiction and completing the proof. \\
    (b) The previous result shows that every list of $n-1$ columns of $A-\lambda
    I$ is linearly independent. Since $A-\lambda I$ is singular, this implies
    $\rank(A-\lambda I)=n-1$. By the rank-nullity theorem, the geometric 
    multiplicity of $\lambda$ is $1$.
  \end{proof}

  \paragraph{14.}
  \begin{proof}
    \noindent
    (a) It follows immediately from
    \[
      (A-\lambda I)\adj(A-\lambda I)=p_A(\lambda)I=0 \quad\Rightarrow\quad
      A\adj(A-\lambda I) = \lambda\adj(A-\lambda I).
    \] \\
    (b) The proof is similar to the one of (a).\\
    (c) Given $\lambda\in\sigma(A)$, $\adj(A-\lambda)\ne 0$ iff $\rank(A-\lambda
    I)=n-1$ iff $\lambda$ has geometric multiplicity $1$.\\
    (d) It follows from 
    \[
      \adj(A-\lambda I) = 
      \begin{bmatrix} d-\lambda & -b \\ -c & a-\lambda \end{bmatrix}
    \]
    and the results of (a) and (b).
  \end{proof}
% ebd

