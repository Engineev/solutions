\section{Eigenvalues, eigenvectors, and similarity}

\subsection{Introduction}

  \paragraph{1.}
  \begin{proof}
    Let $S=\{x\in\mathbb{R}^n\,:\,x^Tx=1\}$, which is clearly a compact subset
    of $\mathbb{R}^n$. Consider the function $f:x\mapsto x^TAx$. Since,
    \[
      \|f(x+\delta)-f(x)\| = \|(x^TA)\delta + \delta^T(Ax) + \delta^TA\delta\|
      \le K\|\delta\|
    \]
    for every $x\in\mathbb{R}$ and some fixed $K$, $f$ is continuous. Hence,
    by Weierstrass's theorem, $f$ attains its maximum value at some point $x\in 
    S$. Namely, (1.0.3) has a solution $x$. Therefore, there exists some 
    $\lambda\in\mathbb{R}$ such that $2(Ax-\lambda x)=0$, implying that every
    real symmetric matrix has at least one real eigenvalue.
  \end{proof}

  \paragraph{2.}
  \begin{proof}
    Let $S=\{x\in\mathbb{R}^n\,:\,x^Tx=1\}$ and $m$ be the maximum value of $x
    \mapsto x^TAx$ in $S$. Suppose $\lambda$ is an eigenvalue of $A$ and $u\ne0$ 
    is its associated eigenvector, then
    \[
      Au=\lambda u \quad\Rightarrow\quad 
      u^TAu=\lambda\|u\|^2 \quad\Rightarrow\quad
      (u/\|u\|)^T A (u/\|u\|) = \lambda \quad\Rightarrow\quad
      m \ge \lambda.
    \]
    Meanwhile, by the previous discussion, $m$ itself is a eigenvalue of $A$.
    Hence, it is the largest real eigenvalue of $A$.
  \end{proof}

% end