\section{Topological Spaces and Continuous Functions}
\setcounter{subsection}{12}
\subsection{Basis for a Topology}
  \paragraph{1.}
  \begin{proof}
    Let $\mT$ be the topology of $X$. Since $\mT$ is a basis for itself and the 
    hypothesis implies that $A$ is a set in the topology generated by $\mT$, 
    $A\in\mT$, i.e., $A$ is open.
  \end{proof}
  
  \paragraph{4.}
  \begin{proof}
    $\,$\par
    (a) Put $\mT=\bigcap_\alpha \mT_\alpha$. Since $\varnothing$ and $X$ are 
    contained in all $\mT_\alpha$, they are also contained in $\mT$. Let $\{
    U_\beta\}_{\beta\in J}$ be an indexed family of elements of $\mT$ and put
    $U=\bigcup_{\beta\in J}U_\beta$. For every $\beta$, since $U_\beta$ is open
    with respect to each $\mT_\alpha$ , by definition, so is $\bigcup_{\beta\in
    J}$. Similarly, we can show that $\mT$ is closed under finite intersection.
    Thus, $\mT$ is a topology.\par
    The union $\bigcup\mT_\alpha$, however, may not be a topology. Take $X=\{a,
    b,c\}$ for example. $\mT_a=\{\varnothing, a, X\}$ and $\mT_b=\{\varnothing, 
    b, X\}$ are two topologies, but their union is not.\par
    (b) Let $\mT$ be the intersection of all topologies containing all $\mT
    _\alpha$. By (a), $\mT$ is a topology and clear that it is the unique
    smallest one. Now, let $\mT\hp=\bigcap T_\alpha$, which is again a 
    topology and is contained in all $T_\alpha$. It can be verified that $\mT
    \hp$ is the unique largest one.\par
    (c) $\{\varnothing, X, \{a\}, \{b\}, \{a,b\}, \{b,c\}\}$; $\{\varnothing,
    X, \{a\}\}$.
  \end{proof}
  
  \paragraph{5.}
  \begin{proof}
    Let $\mathcal{A}$ be a basis, $\mT$ the topology generated by 
    $\mathcal{A}$, $\{\mT_\alpha\}$ the collection of all topologies containing
    $\mathcal{A}$ and $\mT\hp=\bigcap\mT_\alpha$. For every union $U$ of
    elements of $\mathcal{A}$, since, for every $\alpha$, $\mathcal{A}\subset
    \mT_\alpha$ and $\mT_\alpha$ is closed under arbitrary union, $U\in
    \mT_\alpha$. Hence, $\mT\subset\mT\hp$. Consequently, $\mT\hp$ is also the
    intersection of all topologies containing $\mT$. Since $\mT$ contains 
    itself as a subset, $\mT\hp\subset\mT$. Thus, $\mT=\mT\hp$.\par
    Consider the collection of all finite intersections of $\mathcal{A}$, which
    is a basis, and apply the previous result to complete the proof.
  \end{proof}
  
  \paragraph{6.}
  \begin{proof}
    Let $\mT_l$ and $\mT_K$ be the topology of $\mathbb{R}_l$ and $\mathbb{R}
    _k$ respectively. $B=(-1,1)-K$ is a basis element of $\mT_k$ and $0\in B$.
    However, no half-open interval containing $0$ is in $B$. Hence, $\mT_l$ is
    no finer than $\mT_K$. Conversely, $C=[1,2)$ is a basis element of $\mT_l$ 
    and $1\in C$, but as $1\in K$, there is no basis element of $\mT_K$ 
    containing $1$. Hence, $\mT_K$ is no finer than $\mT_l$. Thus, they are not
    comparable.
  \end{proof}
  
  \paragraph{8.}
  \begin{proof}
    $\,$\par
    (a) First clear that $\mathcal{B}\subset\mT$. For every $U\in\mT$ and $x\in 
    U$, since $U$ is open, there exists some $\delta>0$ such that $(x-\delta,x+
    \delta)\subset U$. Hence, there exists some rational $a$ and $b$ such that
    $x-\delta<a<x<b<x+\delta$. Thus, by Lemma 13.2, $\mathcal{B}$ generates the 
    standard topology on $\mathbb{R}$.\par
    (b) Since $x\in[\lfloor x\rfloor,\lfloor x\rfloor+1)\in\mathcal{C}$ for 
    every $x\in\mathbb{R}$, the first condition for a basis is satisfied. 
    Meanwhile, for every $B_1=[a,b)$ and $B_2=[c,d)$ in $\mathcal{C}$, if they
    are not disjoint, $[c,b)=B_1\cap B_2$ is also in $\mathcal{C}$. Hence, the
    second condition is satisfied. Thus, $\mathcal{C}$ is a basis.\par
    Since $[\sqrt{2},2)$ can not be represented by union of elements in 
    $\mathcal{C}$, $\mathcal{C}$ does not generate the lower limit topology.
  \end{proof}
% end

\setcounter{subsection}{15}
\subsection{The Subspace Topology}
  \paragraph{1.}
  \begin{proof}
    Denote the topologies inherited from $X$ and $Y$ by $\mT$ and $\mT\hp$ 
    respectively. For every $E=H\in\mT$, supposing that $E=H\cap A$ where $H$
    is open in $X$, then, since $E \subset A\subset Y$, $E=(Y\cap H)\cap A$. 
    Namely, $E\in\mT\hp$. For the converse, suppose that $F=K\cap A$ where $K$
    is open in $Y$, then, for some $H$ open in $X$, $F=(H\cap Y)\cap A=H\cap 
    A$. Namely, $F\in\mT$. Thus, $\mT=\mT\hp$.
  \end{proof}
  
  \paragraph{2.}
  \begin{proof}
    Denote the corresponding subspace topologies by $\mathcal{S}$ and 
    $\mathcal{S}\hp$ respectively. Clear that $\mathcal{S}\hp$ is finer than 
    $\mathcal{S}$. The relation, however, may not be strict. As an example, put
    $Y=\{y\}$. Then both $\mathcal{S}$ and $\mathcal{S}\hp$ are $\{\varnothing,
    Y\}$.
  \end{proof}
  
  \paragraph{4.}
  \begin{proof}
    By Lemma 13.1, $(U, V)$ is open in $X\times Y$ iff $U=\bigcup U_\alpha$ and 
    $V=\bigcup V_\beta$ where all $U_\alpha$ and $V_\beta$ are open in $X$ and 
    $Y$ respectively. Hence, $\pi_1(U,V)=\bigcup U_\alpha$ and $\pi_2(U,V)=
    \bigcup V_\beta$ are also open. Thus, $\pi_1$ and $\pi_2$ are open maps.
  \end{proof}
  
  \paragraph{6.}
  \begin{proof}
    By Prob. 8(a), Sec. 13, $\{(a,b):\,a<b,\,a,b\in\mathbb{Q}\}$ is a basis for
    $\mathbb{R}$. The result then follows immediately from Theorem 15.1.
  \end{proof}
  
  \paragraph{7.}
  \begin{proof}
    No. Let $X=\mathbb{Q}$ with the usual order and $Y=\{x:\, 0\le x^2\le 2\}$.
    $Y$ is a proper subset of $X$ and is convex in $X$ but not an interval or a
    ray.
  \end{proof}
  
  \paragraph{9.}
  \begin{proof}
    $\mathcal{B}_d=\mathcal{P}(\mathbb{R})\times\{(b,d):\, b<d,\,b,d\in
    \mathbb{R}\}$ is a basis for $\mathbb{R}_d\times\mathbb{R}$ and by Example
    2, Sec. 14, $\mathcal{B}_o=\{\{a\}\times(b,d):\, a,b,d\in\mathbb{R},\,
    b<d\}$ is a basis for the dictionary order topology on $\mathbb{R}\times
    \mathbb{R}$. Clear that $\mathcal{B}_0\subset\mathcal{B}_d$. Meanwhile, for
    every $E\in\mathcal{P}(\mathbb{R})$, $E=\bigcup_{x\in E}\{x\}$. Hence, 
    $\mathcal{B}_d\subset\mathcal{B}_o$. Thus, these two topologies are the 
    same.\par
    The collection $\mathcal{B}$ of all products of open intervals is a basis
    for the standard topology on $\mathbb{R}^2$. Clear that $\mathcal{B}\subset
    \mathcal{B}_d$. Meanwhile, $\{0\}\times\mathbb{R}$ is open in $\mathbb{R}_d
    \times\mathbb{R}$ but not in the standard topological space. Thus, the 
    previous two topologies are strictly finer than the standard topology.
  \end{proof}
  
  \paragraph{10.}
  \begin{proof}
    Denote these topologies by $\mT_i$, $i=1,2,3$, respectively. $[0,1]\times
    (1/2,1]\in\mT_1\setminus\mT_2$. Hence, $\mT_2$ is no finer than $\mT_1$. 
    Meanwhile, since $\{1/2\}\times(1/2,1)\in\mT_2\setminus\mT_1$, $\mT_1$ is 
    no finer than $\mT_2$. Thus, $\mT_1$ and $\mT_2$ are not comparable.\par
    Now we show that $\mT_3$ is finer than both $\mT_1$ and $\mT_2$ and since
    $\mT_1$ and $\mT_2$ are not comparable, this relation is strict. Let 
    $\mathcal{B}_1$ be the collection of all products of open intervals in 
    $I$ and $\mathcal{B}_3$ the collection of all sets of form $\{a\}\times
    ((b,d)\cap[0,1])$ where $a\in[0,1]$. They are bases of $\mT_1$ and $\mT_3$.
    respectively. Since every element in $\mathcal{B}_1$ can be represented by
    an arbitrary union of elements in $\mathcal{B}_3$, $\mT_3$ is finer than
    $\mT_1$. Similarly, we assert that $\mT_3$ is also finer than $\mT_2$.
  \end{proof}
% end
\subsection{Closed Sets and Limit Points}
  \paragraph{2.}
  \begin{proof}
    Since $A$ is a subset of $Y\subset X$, $X\setminus A=(Y\setminus A)\cup
    (X\setminus Y)$. Since $A$ is closed in $Y$ and $Y$ in closed in $X$, this
    implies that $X\setminus A$ is open. Thus, $A$ is closed in $X$.
  \end{proof}
  
  \paragraph{4.}
  \begin{proof}
    Since $A$ is closed in $X$, $A^c$ is open in $X$. Hence, $U\setminus A=
    U\cap A^c$ is open. Similarly, $A\setminus U$ is closed.
  \end{proof}
  
  \paragraph{6.}
  \begin{proof}
    $\,$\par
    (a) For any $x\in X$, if the neighborhood $U$ of $x$ intersects $A$, then
    it intersects $B$ since $A\subset B$. Thus, $\bar{A}\subset\bar{B}$.\par
    (b) Since $\overline{A\cup B}$ is the smallest closed set containing $A
    \cup B$ and $\bar{A}\cup\bar{B}$ is closed set containing $A\cup B$, 
    $\overline{A\cup B}\subset\bar{A}\cup\bar{B}$. For the reverse inclusion,
    suppose that $x\in\bar{A}\cup\bar{B}$. If $x\in\bar{A}$, then all its 
    neighborhood intersects $A\cup B\supset A$. Hence, $x\in\overline{A\cup 
    B}$. Similarly for the case $x\in\bar{B}$. Therefore, $\bar{A}\cup\bar{B}
    \subset\overline{A\cup B}$. Thus, $\overline{A\cup B}=\bar{A}\cup\bar{B}$.
    \par
    (c) The previous argument, \textit{mutatis mutandis}, yields the inclusion.
    Let $X=\mathbb{R}$ and $A_n=[0,1/n]$. Then, $\overline{\bigcup A_n}=[0,1]$
    and $\bigcup\bar{A}_n=[0,1)$, which do not coincide.
  \end{proof}
  
  \paragraph{8.}
  \begin{proof}
    $\,$\par
    (a) We show that the equality holds. Since $\overline{A\cap B}$ is the
    smallest closed set containing $A\cap B$ and clear that $\bar{A}\cap
    \bar{B}$ is closed set containing $A\cap B$, $\overline{A\cap B}\subset
    \bar{A}\cap\bar{B}$. For the reverse inclusion, suppose that $x\in\bar{A}
    \cap\bar{B}$, then every neighborhood of $x$ intersects both $A$ and $B$.
    Hence, $x\in\overline{A\cap B}$. Thus, $\overline{A\cap B}=\bar{A}\cap
    \bar{B}$.\par
    (b) The previous argument, \textit{mutatis mutandis}, shows that 
    $\overline{\bigcap A_\alpha}\subset\bigcap\bar{A}_\alpha$. The reverse
    inclusion does not hold in general. For example, let $X=\mathbb{R}$ and
    $A_n=(0,1/n)$. Then $\overline{\bigcap A_n}=\varnothing$ but $\bigcap\bar{
    A_n}=\{0\}$.\par
    (c) We show that $\overline{A\setminus B}\supset\bar{A}\setminus\bar{B}$.
    \[
      \bar{A}\setminus\bar{B}\subset
      \bar{A}\setminus B=
      \bar{A}\cap B^c\subset
      \bar{A}\cap\overline{(B^c)}=
      \overline{A\cap B^c}=
      \overline{A\setminus B}
    \]
    where the second equality comes from part (a). The reverse inclusion does
    not hold in general. For example, let $X=\mathbb{R}$, $A=[0,1]$ and $B=(0,
    1)$. Then $\overline{A\setminus B}=\{0,1\}$ but $\bar{A}\setminus\bar{B}=
    \varnothing$.
  \end{proof}
  
  \paragraph{10.}
  \begin{proof}
    Let $X$ be a simply ordered set and $\mcal{T}$ the order topology for $X$.
    If $X$ contains only one single point, then, vacuously, $\mcal{T}$ is 
    Hausdorff. Otherwise, let $a<b$ be two distinct point in $X$. If $(a,b)\ne
    \varnothing$, that is, there exists some $c$ with $a<c<b$, then $[-\infty,
    c)$ and $(c,\infty]$ are two disjoint open sets containing $a$ and $b$
    respectively.  If $(a,b)=\varnothing$, then $[-\infty,b)\setminus(a,b)$ and 
    $(a,\infty]\setminus(a,b)$ are two such sets. Thus, we conclude that 
    $\mcal{T}$ is Hausdorff. (The case where the $\pm\infty$ can not be 
    attained is similar.)
  \end{proof}
  
  \paragraph{12.}
  \begin{proof}
    Let $Y$ be a subspace of the Hausdorff space $X$. Let $a,b$ be two distinct
    points of $Y$. Since $X$ is Hausdorff, there are two disjoint sets $U$ and 
    $V$ which contain $a$ and $b$ respectively and are open in $X$. Hence, 
    $U\cap Y$ and $V\cap Y$ are two disjoint open sets in $Y$ containing $a$
    and $b$ respectively. Thus, $Y$ is also Hausdorff.
  \end{proof}
  
  \paragraph{14.}
  \begin{proof}
    Let $\mcal{T}$ be the finite complement topology on $\mathbb{R}$. We show
    that for the sequence $(x_n)$ converges to every point of $\mathbb{R}$.
    Let $x$ be an arbitrary point of $\mathbb{R}$ and $U$ an neighborhood of 
    $x$. Since $U$ is open and nonempty, $U^c$ is finite. Hence, $U^c$ contains
    at most finitely many points in $(x_n)$. Thus, $x_n\in U$ for all 
    sufficiently large $n$. Namely, $x_n\to x$.
  \end{proof}
  
  \paragraph{21.b}
  \begin{solution}
    $A=(-\infty,1)\cup(1,2]\cup\{3\}\cup([4,5]\cap\mathbb{Q})\cup(6,7)\cup
    (7,8]$.
  \end{solution}
% end
\subsection{Continuous Functions}
  \paragraph{1.}
  \begin{proof}
    For every open set $V$ in $\mathbb{R}$, put $U=f\inv(V)$. Since the 
    collection of open intervals forms a basis for the topology on 
    $\mathbb{R}$, it suffices to show that for each $x\in U$, there is an open
    interval $I$ containing $x$ such that $I\subset U$ to conclude that $U$ is
    open. Put $y=f(x)$. Since $V$ is open, there is a $\vep>0$ such that $J=
    (y-\vep,y+\vep)\subset V$. By the $\vep-\delta$ condition, there is a 
    $\delta>0$ such that for all $x\hp\in(x-\delta,x+\delta)=I$, $f(x\hp)\in
    J\subset V$. Namely, $I\subset U$. Thus, $f$ is continuous in the sense of
    the open set definition.
  \end{proof}
  
  \paragraph{3.}
  \begin{proof}
    $\,$\par
    (a) $i$ is continuous iff every open set in $X$ is open in $X\hp$ iff 
    $\mcal{T}\subset\mcal{T}\hp$, i.e., $\mcal{T}\hp$ is finer than $\mcal{T}$.
    \par
    (b) By part (a), $\mcal{T}$ is finer than $\mcal{T}\hp$ and \textit{vice 
    versa}. Hence, $\mcal{T}=\mcal{T}\hp$.
  \end{proof}
  
  \paragraph{5.}
  \begin{proof}
    Clear that $f(x)=a+t(b-a)$ is a homeomorphism in both cases.
  \end{proof}
  
  \paragraph{7.a}
  \begin{proof}
    Let $x$ be a point in $\mathbb{R}_l$ and put $y=f(x)\in\mathbb{R}$. For
    every open interval $J$ containing $y$, since $f$ is continuous from the
    right, there exists a $\delta>0$ such that $f(I)\subset J$ where $I=[x, x+
    \delta)$. Since $I$ is open in $\mathbb{R}_l$, this implies that $f$ is
    continuous when considered as a function from $\mathbb{R}_l$ to 
    $\mathbb{R}$.
  \end{proof}
  
  \paragraph{9.}
  \begin{proof}
    $\,$\par
    (a) It follows from the pasting lemma and the fact that the finite union of 
    closed sets is still closed.\par
    (b) Put $A_0=\{0\}$ and $A_n=(-\infty,-1/n]\cup[1/n,\infty)$. Clear that 
    $\{A_n\}_{n=0}^\infty$ is a sequence of closed set whose union is 
    $\mathbb{R}$. Define $f:\mathbb{R}\to\mathbb{R}$ by $f(0)=0$ and $f(x)=1$
    for all $x\ne 0$. Clear that $f$ is not continuous but $f|_{A_n}$ are all
    continous.\par
    (c) By Theorem 18.1.4, it suffices to show that $f$ is continuous at every
    $x\in X$. For each $x$, let $U$ be the neighborhood of $x$ that only 
    intersects finitely many $A_\alpha$. Let $\{A_k\}_{k\in K}$ denote the 
    collection of such $A_\alpha$. Since $U$ is open, each $U\cap A_k$ is 
    closed in $U$. Hence, part (a), $f|_U$ is continuous and, therefore, $f$ is
    continuous at $x\in U$. Thus, $f$ is continuous on $X$.
  \end{proof}
  
  \paragraph{11.}
  \begin{proof}
    Let $y_0\in Y$ be fixed, we show that $h(x)=F(x\times y_0)$ is continuous.
    Let $x\in X$ be fixed and let $V$ be a neighborhood of $h(x)=F(x\times 
    y_0)$. Since $F$ is continuous, by Theorem 18.1.4, there is a basis 
    neighborhood $A\times B$ of $x\times y_0$ such that $F(A\times B)\subset 
    V$. Note that $h(A)=F(A\times\{y_0\})\subset F(A\times B)\subset V$. 
    Hence, $h$ is continuous. Since the roles of $X$ and $Y$ are 
    interchangeable, this implies that $F$ is continuous in each variable
    separately.
  \end{proof}
  
  \paragraph{13.}
  \begin{proof}
    Let $g_1$ and $g_2$ be continuous extensions of $f$. Clear that $g_1=g_2$ 
    on $A$. For every limit point $x$ of $A$, let $V_1$ and $V_2$ be 
    neighborhoods of $g_1(x)$ and $g_2(x)$ respectively. Since both $g_1$ and 
    $g_2$ are continuous, by Theorem 18.1.4, there exists neighborhoods $U_1$ 
    and $U_2$ of $x$ such that $g_1(U_1)\subset V_1$ and $g_2(U_2)\subset V_2$. 
    Since $x$ is a limit point of $A$. $A$ intersects both $U_1$ and $U_2$. 
    Hence, there exists some $x^*\in A$ such that $g_i(x^*)\in V_i$ for 
    $i=1,2$. Since $g_1(x^*)=g_2(x^*)$, this implies that $V_1$ and $V_2$ 
    intersect. As the choice of neighborhoods $V_1$ and $V_2$ are arbitrary, 
    this, by the definition of a Hausdorff space, implies that $g_1(x)=g_2(x)$.
    Namely, if the extension exists, it is unique.
  \end{proof}
% end






















