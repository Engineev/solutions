\section{Topological Spaces and Continuous Functions}
\setcounter{subsection}{12}
\subsection{Basis for a Topology}
  \paragraph{1.}
  \begin{proof}
    Let $\mT$ be the topology of $X$. Since $\mT$ is a basis for itself and the 
    hypothesis implies that $A$ is a set in the topology generated by $\mT$, 
    $A\in\mT$, i.e., $A$ is open.
  \end{proof}
  
  \paragraph{4.}
  \begin{proof}
    $\,$\par
    (a) Put $\mT=\bigcap_\alpha \mT_\alpha$. Since $\varnothing$ and $X$ are 
    contained in all $\mT_\alpha$, they are also contained in $\mT$. Let $\{
    U_\beta\}_{\beta\in J}$ be an indexed family of elements of $\mT$ and put
    $U=\bigcup_{\beta\in J}U_\beta$. For every $\beta$, since $U_\beta$ is open
    with respect to each $\mT_\alpha$ , by definition, so is $\bigcup_{\beta\in
    J}$. Similarly, we can show that $\mT$ is closed under finite intersection.
    Thus, $\mT$ is a topology.\par
    The union $\bigcup\mT_\alpha$, however, may not be a topology. Take $X=\{a,
    b,c\}$ for example. $\mT_a=\{\varnothing, a, X\}$ and $\mT_b=\{\varnothing, 
    b, X\}$ are two topologies, but their union is not.\par
    (b) Let $\mT$ be the intersection of all topologies containing all $\mT
    _\alpha$. By (a), $\mT$ is a topology and clear that it is the unique
    smallest one. Now, let $\mT\hp=\bigcap T_\alpha$, which is again a 
    topology and is contained in all $T_\alpha$. It can be verified that $\mT
    \hp$ is the unique largest one.\par
    (c) $\{\varnothing, X, \{a\}, \{b\}, \{a,b\}, \{b,c\}\}$; $\{\varnothing,
    X, \{a\}\}$.
  \end{proof}
  
  \paragraph{5.}
  \begin{proof}
    Let $\mathcal{A}$ be a basis, $\mT$ the topology generated by 
    $\mathcal{A}$, $\{\mT_\alpha\}$ the collection of all topologies containing
    $\mathcal{A}$ and $\mT\hp=\bigcap\mT_\alpha$. For every union $U$ of
    elements of $\mathcal{A}$, since, for every $\alpha$, $\mathcal{A}\subset
    \mT_\alpha$ and $\mT_\alpha$ is closed under arbitrary union, $U\in
    \mT_\alpha$. Hence, $\mT\subset\mT\hp$. Consequently, $\mT\hp$ is also the
    intersection of all topologies containing $\mT$. Since $\mT$ contains 
    itself as a subset, $\mT\hp\subset\mT$. Thus, $\mT=\mT\hp$.\par
    Consider the collection of all finite intersections of $\mathcal{A}$, which
    is a basis, and apply the previous result to complete the proof.
  \end{proof}
  
  \paragraph{6.}
  \begin{proof}
    Let $\mT_l$ and $\mT_K$ be the topology of $\mathbb{R}_l$ and $\mathbb{R}
    _k$ respectively. $B=(-1,1)-K$ is a basis element of $\mT_k$ and $0\in B$.
    However, no half-open interval containing $0$ is in $B$. Hence, $\mT_l$ is
    no finer than $\mT_K$. Conversely, $C=[1,2)$ is a basis element of $\mT_l$ 
    and $1\in C$, but as $1\in K$, there is no basis element of $\mT_K$ 
    containing $1$. Hence, $\mT_K$ is no finer than $\mT_l$. Thus, they are not
    comparable.
  \end{proof}
  
  \paragraph{8.}
  \begin{proof}
    $\,$\par
    (a) First clear that $\mathcal{B}\subset\mT$. For every $U\in\mT$ and $x\in 
    U$, since $U$ is open, there exists some $\delta>0$ such that $(x-\delta,x+
    \delta)\subset U$. Hence, there exists some rational $a$ and $b$ such that
    $x-\delta<a<x<b<x+\delta$. Thus, by Lemma 13.2, $\mathcal{B}$ generates the 
    standard topology on $\mathbb{R}$.\par
    (b) Since $x\in[\lfloor x\rfloor,\lfloor x\rfloor+1)\in\mathcal{C}$ for 
    every $x\in\mathbb{R}$, the first condition for a basis is satisfied. 
    Meanwhile, for every $B_1=[a,b)$ and $B_2=[c,d)$ in $\mathcal{C}$, if they
    are not disjoint, $[c,b)=B_1\cap B_2$ is also in $\mathcal{C}$. Hence, the
    second condition is satisfied. Thus, $\mathcal{C}$ is a basis.\par
    Since $[\sqrt{2},2)$ can not be represented by union of elements in 
    $\mathcal{C}$, $\mathcal{C}$ does not generate the lower limit topology.
  \end{proof}
% end
























