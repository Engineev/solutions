\section{Linear Map}
\subsection{The Vector Space of Linear Maps}
  \paragraph{1.}
  \begin{proof}
    If $T$ is linear, then $T(0,0,0)=0$ and therefore $b=0$. Meanwhile, 
    $T(2,2,2)=2T(1,1,1)$ implies $12+8c=12+2c$. Hence, $c=0$. The proof of the 
    converse part is trivial.
  \end{proof}

  \paragraph{3.}
  \begin{proof}
    Let $e_i$ be the $i$-th vector in the standard base of $\mathbb{F}^n$ and 
    suppose that $Te_i = \sum_{j=1}^n A_{1,j}e_j$. Then for $x=(x_1,\dots,x_n)^T 
    \in\mathbb{F}^n$,
    \[
      Tx = T\left(\sum_{i=1}^n x_ie_i\right) = \sum_{i=1}^n x_iTe_i =
      \sum_{i=1}^n x_i\sum_{j=1}^nA_{j,i}e_j = 
      \sum_{j=1}^n\left(\sum_{i=1}^nA_{j,i}x_i\right) e_j.
    \]
  \end{proof}

  \paragraph{5.}
  \begin{proof}
    Too lengthy to write it down...
  \end{proof}

  \paragraph{7.}
  \begin{proof}
    Let $\{x_0\}$ be a basis of $V$ and $\lambda$ be a scalar such that $Tx_0=
    \lambda x_0$. By the linearity of $T$, for every $x=kx_0$ in $V$, $Tx=kTx_0
    =k\lambda x_0=\lambda(kx_0)=\lambda x$.
  \end{proof}

  \paragraph{9.}
  \begin{solution}
    From the additivity condition we can derive that $\varphi(kz)=k\varphi(z)$
    for any $k\in\mathbb{Q}$. Hence we can try some functions where $\varphi(iz)
    =i\varphi(z)$ fails. It turns out that $\varphi(z)=\Im(z)$ is one of the 
    maps required.
  \end{solution}

  \paragraph{11.}
  \begin{proof}
    Let $\{\alpha_1,\dots,\alpha_p\}$ and $\{\alpha_1,\dots,\alpha_p,\beta_1,
    \dots,\beta_q\}$ be bases of $U$ and $V$ respectively. Then the linear map 
    which maps $\alpha_i$ to $T\alpha_i$ and maps $\beta$ to $0$. Clear that
    it is the desired linear map. 
  \end{proof}

  \paragraph{13.}
  \begin{proof}
    Suppose that $v_k$ is in the span of the other vectors and let $w_i=0$ for 
    each $i\ne k$ and $w_k\ne 0$. No $T\in\mathcal{L}(V, W)$ can maps $v_i$ to 
    $w_i$ since the linearity of $T$ would force $w_k$ to be $0$, leading to a 
    contradiction.
  \end{proof}

% end

\subsection{Null Spaces and Ranges}
  \paragraph{2.}
  \begin{proof}
    Since $S$ maps every vector of $V$ into the null space of $T$, the map $TS$
    is the zero map. Hence $(ST)^2 = S(TS)T = 0$.
  \end{proof}

  \paragraph{4.}
  \begin{proof}
    Suppose $S,T\in\mathcal{L}(\mathbb{R}^5,\mathbb{R}^4)$ maps and only maps 
    $e_1,e_2,e_3$ and $e_3,e_4,e_5$ to the zero vector respectively. Then $e_1,
    e_2,e_4,e_5\notin\nul(S+T)$, implying that $\dim\nul(S+T)<2$. Hence $\{T\in
    \mathcal{L}(\mathbb{R}^5,\mathbb{R}^4\,:\,\dim\nul T>2)\}$ is not a subspace
    of $\mathcal{L}(\mathbb{R}^5,\mathbb{R}^4)$.
  \end{proof}

  \paragraph{6.}
  \begin{proof}
    It follows immediately from the rank-nullity theorem and the fact that $\dim
    \nul T$ and $\dim\range T$ are integers.
  \end{proof}

  \paragraph{8.}
  \begin{proof}
    Let $\{w_1,\dots,w_m\}$ be a basis of $W$ and $S,T\in\mathcal{L}(V,W)$ be 
    two linear maps such that $\range S=\spn(w_1)$ and $\range T=\spn(w_2,\dots,
    w_n)$. Clear that $\range(S+T)=W$. Hence, the set described is not a 
    subspace of $\mathcal{L}(V,W)$.
  \end{proof}

  \paragraph{10.}
  \begin{proof}
    For every $y\in\range T$ there exists some $x=\sum x_iv_i\in V$ such that 
    \[
      y=Ty = T\left(\sum_{i=1}^n x_iv_i\right) = \sum_{i=1}^n x_iTv_i.
    \]
    Hence, $\range T=\spn(Tv_1,\dots,Tv_n)$.
  \end{proof}

  \paragraph{12.}
    For readers who familiar with the orbit-stabilizer theorem or just the 
    (group) homomorphism, the proof should be straightforward.
  \begin{proof}
    For every nonzero $y$ in $\range T$, there exists some $x\in V$ such that 
    $Tx=y$. For each $y\ne 0$, we choose one such $x$, put them all together and 
    put $0$ into them to get $U$. By the construction, clear that $T(U) = 
    \range T$ and , $U\cap\nul T=\{0\}$.
  \end{proof}

  \paragraph{14.}
  \begin{proof}
   By the rank-nullity theorem, 
   \[
     \dim\nul T + \dim\range T = 8 \quad\Rightarrow\quad
     \dim\range T = 5 = \dim\mathbb{R}^5.
   \]
   Hence, $\range T = \mathbb{R}^5$ and therefore $T$ is surjective.
  \end{proof}

  \paragraph{16.}
    Actually, the cosets of the kernel partition the whole space.
  \begin{proof}
    Let $\{v_1,\dots,v_n\}$ be a basis of $\range T$ and $Tu_i=v_i$ for $i=1,2,
    \dots,n$. Denote $\spn(u_1,\dots,u_n)$ by $U$. We now prove that $V=U+\nul
    T$. For every $x\in V$, suppose that $Tx=y=\sum y_iv_i$ and $\tilde{x}=\sum
    y_iu_i$. Note that $\tilde{x}\in U$ and $T(x-\tilde{x})=Tx-T\tilde{x}=0$, 
    i.e., $x-\tilde{x}\in\nul T$. Hence, $V=U+\nul T$. As both of $U$ and $\nul
    T$ are finite-dimensional, so is $V$.
  \end{proof}

  \paragraph{18.}
  \begin{proof}
    By the rank-nullity theorem, clear that $\dim V \ge \dim\range T =\dim W$ if
    there exists some surjective $T\in\mathcal{L}(V,W)$. \par
    Assume that $\dim V\ge\dim W$ and let $\{v_1,\dots,v_n\}$ and $\{w_1,\dots,
    w_m\}$ be bases of $V$ and $W$ respectively. Then the linear map which
    maps $v_i$ to $w_i$ for each $1\le i\le m$ is surjective.
  \end{proof}

  \paragraph{20.}
  \begin{proof}
    If $T$ is injective, then for every $y\in\range T$, there exists exactly 
    one $x\in V$ such that $y=Tx$. Let $S$ be the map which maps $y$ to such
    $x$. It is linear since for every $y_1,y_2\in\range T$ and scalar $a,b$,
    supposing $Sy_i = x_i$,
    \[
      T(ax_1+bx_2) = aTx_1 + bTx_2 = ay_1 + by_2.
    \]
    implying $S(ay_1+by_2) = ax_1+bx_2 = aSy_1 + bSy_2$. For every $x\in V$,
    $(ST)x = S(Tx) = x$. \par
    Suppose there exists some $S\in \mathcal{L}(W,V)$ such that $ST=I$. Then
    \[
      Tx_1 = Tx_2 \quad\Rightarrow\quad
      STx_1 = STx_2 \quad\Rightarrow\quad
      x_1 = x_2.
    \]
    Hence, $T$ is injective.
  \end{proof}

  \paragraph{22.}
  \begin{proof}
    Let $\tilde{T}$ be the restriction of $T$ to $\nul ST$. It is still a linear
    map since $\nul ST$ is a subspace of $U$. Note that $x\in\nul ST$ iff $(ST)x
    =0$ iff $Tx\in\nul S$. Hence, $\range\tilde{T}\subset\nul S$. Thus, by the 
    rank-nullity theorem,
    \[
      \dim\range\tilde{T} \le \dim\nul S \quad\Rightarrow\quad
      \dim\nul ST - \dim\nul\tilde{T} \le \dim\nul S.
    \]
    Since $\nul\tilde{T} \le \nul T$, this implies $\dim\nul ST \le \dim\nul S +
    \dim\nul T$.
  \end{proof}

  \paragraph{24.}
  \begin{proof}
    If there exists $S\in\mathcal{L}(W,W)$ such that $T_2=ST_1$, then $\nul T_2
    =\nul ST_1$. Hence for every $x\in\nul T_1$, as $S(T_1x)=S0 = 0$, $x\in\nul
    T_2$. Therefore, $\nul T_1\subset\nul T_2$.\par
    Now we suppose $\nul T_1\subset\nul T_2$ and construct $S$. Note that all we
    concerns is its behavior on some basis of $\range T_1$. Let $\{w_1, \dots, 
    w_n\}$ be a basis of $\range T_1$ and $T_1v_i = w_i$ for $i=1,\dots,n$. For
    each $x\in V$, let $U_x = \{x+y\,:\,y\in\nul T_2\}$ and $Sw_k=T_2x$ if $v_k
    \in U_x$. It can be verified that $S$ is well-defined and does satisfy the
    requirement as long as $\nul T_1\subset\nul T_2$.
  \end{proof}

  \paragraph{26.} TODO

  \paragraph{28.} TODO

  \paragraph{30.} TODO

% end