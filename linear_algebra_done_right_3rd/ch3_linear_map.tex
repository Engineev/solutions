\section{Linear Map}
\subsection{The Vector Space of Linear Maps}
  \paragraph{1.}
  \begin{proof}
    If $T$ is linear, then $T(0,0,0)=0$ and therefore $b=0$. Meanwhile, 
    $T(2,2,2)=2T(1,1,1)$ implies $12+8c=12+2c$. Hence, $c=0$. The proof of the 
    converse part is trivial.
  \end{proof}

  \paragraph{3.}
  \begin{proof}
    Let $e_i$ be the $i$-th vector in the standard base of $\mathbb{F}^n$ and 
    suppose that $Te_i = \sum_{j=1}^n A_{1,j}e_j$. Then for $x=(x_1,\dots,x_n)^T 
    \in\mathbb{F}^n$,
    \[
      Tx = T\left(\sum_{i=1}^n x_ie_i\right) = \sum_{i=1}^n x_iTe_i =
      \sum_{i=1}^n x_i\sum_{j=1}^nA_{j,i}e_j = 
      \sum_{j=1}^n\left(\sum_{i=1}^nA_{j,i}x_i\right) e_j.
    \]
  \end{proof}

  \paragraph{5.}
  \begin{proof}
    Too lengthy to write it down...
  \end{proof}

  \paragraph{7.}
  \begin{proof}
    Let $\{x_0\}$ be a basis of $V$ and $\lambda$ be a scalar such that $Tx_0=
    \lambda x_0$. By the linearity of $T$, for every $x=kx_0$ in $V$, $Tx=kTx_0
    =k\lambda x_0=\lambda(kx_0)=\lambda x$.
  \end{proof}

  \paragraph{9.}
  \begin{solution}
    From the additivity condition we can derive that $\varphi(kz)=k\varphi(z)$
    for any $k\in\mathbb{Q}$. Hence we can try some functions where $\varphi(iz)
    =i\varphi(z)$ fails. It turns out that $\varphi(z)=\Im(z)$ is one of the 
    maps required.
  \end{solution}

  \paragraph{11.}
  \begin{proof}
    Let $\{\alpha_1,\dots,\alpha_p\}$ and $\{\alpha_1,\dots,\alpha_p,\beta_1,
    \dots,\beta_q\}$ be bases of $U$ and $V$ respectively. Then the linear map 
    which maps $\alpha_i$ to $T\alpha_i$ and maps $\beta$ to $0$. Clear that
    it is the desired linear map. 
  \end{proof}

  \paragraph{13.}
  \begin{proof}
    Suppose that $v_k$ is in the span of the other vectors and let $w_i=0$ for 
    each $i\ne k$ and $w_k\ne 0$. No $T\in\mathcal{L}(V, W)$ can maps $v_i$ to 
    $w_i$ since the linearity of $T$ would force $w_k$ to be $0$, leading to a 
    contradiction.
  \end{proof}

% end