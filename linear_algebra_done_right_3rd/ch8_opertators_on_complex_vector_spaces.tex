\section{Operators on Complex Vector Spaces}
\subsection{Generalized Eigenvectors and Nilpotent Operators}
  \paragraph{3.}
  \begin{proof}
    Suppose $\dim V= N$. 
    \begin{align*}
      v\in G(T\inv, \lambda\inv) 
      &\Leftrightarrow (T\inv v-\lambda\inv I)^nv = 0 \\
      &\Leftrightarrow 
        \left(\sum_{k=0}^n\binom{n}{k}T^{-k}(-\lambda\inv)^{n-k}\right)v=0 \\
      &\Leftrightarrow
        T^n(-\lambda)^n
        \left(\sum_{k=0}^n\binom{n}{k}T^{-k}(-\lambda)^{k-n}\right)v=0 \\
      &\Leftrightarrow
        \left(\sum_{k=0}^n\binom{n}{k}T^{n-k}(-\lambda)^{k}\right)v=0 \\
      &\Leftrightarrow (T-\lambda I)^nv = 0 \\
      &\Leftrightarrow v\in G(T,\lambda).
    \end{align*}
  \end{proof}

  \paragraph{5.}
    Intuitively, $\nul T,\dots,\nul T^n$ is a sequence of subspaces where the 
    preceding ones are contained by succeeding ones. And $T^{k}v$ lies in the 
    additional part between two successive subspaces.
  \begin{proof}
    Note that $T^{m-1}v\ne 0$ but $T^mv=0$ implies for $k=1,\dots,m$
    \[
      T^{m-k}v\in\nul T^k,\quad T^{m-k}v\notin\nul T^{k-1}.
    \]
    Therefore, $T^{m-k}v\notin\spn(T^{m-1}v,\dots,T^{m-k+1}v)$; otherwise, 
    $T^{m-k}v=x_1T^{m-1}v+\cdots+x_{m-k+1}T^{m-k+1}v$, implying that $T^{m-k}v
    \in\nul T^{k-1}$. Hence, $v, \dots, T^{m-1}v$ are linearly independent.
  \end{proof}

  \paragraph{7.}
  \begin{proof}
    It follows immediately from 8.19.
  \end{proof}

  \paragraph{9.}
  \begin{proof}
    Suppose that $v$ is nonzero and $(TS-\lambda I)v=0$. Then 
    \[
      (STS-\lambda S)v=0 \quad\Rightarrow\quad
      (ST-\lambda I)Sv=0.
    \]
    If $Sv=0$, then $0=(TS-\lambda I)v=T(Sv)-\lambda v$ implies $\lambda=0$. If
    $Sv\ne 0$, then $\lambda$ is an eigenvalue of $ST$ and therefore equals $0$
    by Exercise 7. Hence, in every case, $\lambda=0$. Thus, $TS$ is also 
    nilpotent.
  \end{proof}

  \paragraph{13.}
  \begin{proof}
    We are going to show that $N^{k-1}=0$ if $N^k=0$ for $k>1$ to conclude that 
    $N=0$. Suppose that $N^k=0$ and let $M=N^{k-1}$. Then for every $v\in V$,
    \[
      \|M^*Mv\|^2 = \langle M^*Mv, M^*Mv\rangle
      = \langle M^*MM^*Mv,v\rangle = \langle M^*M^*MMv,v\rangle.
    \]
    Since $M^2=N^{2k-2}=0$, this implies $M^*M=0$. Hence, with the polar 
    decomposition, $M=0$.
  \end{proof}

  \paragraph{15.}
  \begin{proof}
    Suppose $\dim V=n$. Since $\nul N^{n-1}\ne\nul N^n$,
    \[
      0<\dim N < \cdots < \dim N^n.
    \]
    Hence, $\dim\nul N^j=j$ for $0\le j\le n$. Since $\dim\nul N^n=n$, $\nul N^n
    =V$ and therefore $N$ is nilpotent.
  \end{proof}
% end

\subsection{Decomposition of an Operator}
  \paragraph{10.}
  \begin{proof}
    By 8.29, there exists a basis of $V$ with respect to which $\mathcal{M}(T)$
    has the form described in 8.29. Suppose $\mathcal{M}(D)$ be the diagonal
    matrix whose diagonal entries are the ones of $\mathcal{M}(T)$ and 
    $\mathcal{M}(N)=\mathcal{M}(T)-\mathcal{M}(D)$. Clear that $N$ is nilpotent
    and 
    \[
      \mathcal{M}(N) = 
      \begin{bmatrix}
        A_1-\lambda_1 I_1 &        & 0 \\
                          & \ddots &   \\
        0                 &        & A_m-\lambda_m I_m
      \end{bmatrix}
    \]
    where each $I_k$ is the identity matrix with corresponding size. Note that
    $\lambda_k I_k$ and $A_k-\lambda_k I_k$ are commuting. Hence, by Problem 9,
    $\mathcal{M}(N)$ and $\mathcal{M}(D)$ are commuting and so do $D$ and $N$.
  \end{proof}
% end
