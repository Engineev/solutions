\section{Transformations and Expectations}

\paragraph{1.(b)}
\begin{solution}
  $f_X$ is continuous and $g^{-1}(y)=(y-3)/4$ is continuously differentiable,
  therefore
  \[
    f_Y(y)=f_X(g^{-1}(y))\left|\frac{\rd}{\rd y}g^{-1}(y)\right|
    = \frac{7}{4}e^{-7(y-3)/4}.
  \]
  $\mathcal{Y}=(3, \infty)$ and
  \[
    \int_\mathcal{Y} f_Y(y) 
    = -\int_3^\infty \exp\left(-\frac{7}{4}y+\frac{21}{4}\right)
      \rd\left(-\frac{7}{4}y+\frac{21}{4}\right)
    = 1.
  \]
\end{solution}

\paragraph{3.}
\begin{solution}
  \[
    f_Y(y)=P(Y=y)=P\left(\frac{X}{X+1}=y\right)
    = P\left(X=\frac{1}{1-y}-1\right) 
    = \frac{1}{3}\left(\frac{2}{3}\right)^{\frac{1}{1-y}-1},
  \]
  where $y= 0, 1/2, 2/3, \dots$
\end{solution}

\paragraph{5.}
\begin{solution}
  $g(x) = \sin^2x$ is monotone on $(0,\pi/2]$, $[\pi/2,\pi)$, $(\pi,3\pi/2]$ and
  $[3\pi/2, 2\pi)$ respectively and the ranges of $g$ on the intervals are the 
  same. Futhermore,
  \[\begin{matrix}
    &g_0^{-1}(y) = \arcsin\sqrt{y},    & g_1^{-1}(y)=-\arcsin(-\sqrt{y}) \\
    &g_2^{-1}(y) = \pi+\arcsin\sqrt{y}, & g_3^{-1}(y)=\pi-\arcsin(-\sqrt{y}),
  \end{matrix}\]
  all of which are continuously differentiable. Hence,
  \[
    f_Y(y) 
    = \sum_{i=0}^3 \frac{1}{2\pi}\left|\frac{\rd}{\rd y}g^{-1}_i(y)\right|
    = \frac{1}{\pi\sqrt{y(1-y)}}.
  \]
  and vanishes elsewhere.\par
  Next we show that the same answer can be obtained by differentiating (2.1.6).
  Note that $\rd(x+c) = \rd x$ and $x_2-x_1=2(\pi-x)1$, then we get
  \[
    f_Y(y)=\frac{\rd}{\rd y}P(Y\le y)
    = \frac{2}{\pi}\frac{\rd}{\rd y}x_1 
    = \frac{2}{\pi}\frac{\rd}{\rd y}\arcsin\sqrt{y}
    = \frac{1}{\pi\sqrt{y(1-y)}}.
  \]
\end{solution}

% \paragraph{7.(b)}
% \begin{proof}
%   Let $\tilde{f}_Y(y)$ be the function defined in Theorem 2.1.8 and now we prove
%   that $\tilde{f}_Y(y)$ is the pdf of $Y$. Clear that $\tilde{f}_Y(y)\ge 0$ for 
%   all $y\in\mathcal{Y}$. And
%   \[
%     \int_{-\infty}^\infty \tilde{f}_Y(y)\rd y
%   \]
% \end{proof}

\paragraph{9.}
\begin{solution}
  By Theorem 2.1.10, we only need to set $u=F_X$, that is,
  \[
    u(x) = \int_{-\infty}^x f(x)\rd x = 
    \begin{cases}
      0, & x\le 1, \\
      (x-1)^2/4, & 1<x<3, \\
      1, & x\ge 3,
    \end{cases}
  \]
  and the monotonicity is obvious.
\end{solution}

\paragraph{11.(a)}
\begin{solution}
  Direct calculating yields
  \[\begin{split}
    \E X^2 = \int_{-\infty}^\infty x^2 \frac{1}{\sqrt{2\pi}}e^{-x^2}\rd x
    = \frac{1}{\sqrt{2\pi}}\int_{-\infty}^\infty e^{-x^2/2}\rd x
    = 1.
  \end{split}\]
  And by Example 2.1.17,
  \[\begin{split}
     \E Y
    =& \int_0^\infty y\frac{1}{2\sqrt{y}}
      (f_X(\sqrt{y})+f_X(-\sqrt{y}))\rd y \\
    =& \frac{1}{\sqrt{2\pi}}\int_0^\infty\sqrt{y}e^{-y/2}\rd y
    = \sqrt{\frac{2}{\pi}}\int_0^\infty y^2e^{-y^2/2}\rd y = 1.
  \end{split}\]
\end{solution}

\paragraph{13.}
\begin{solution}
  $f_X(x) = P(X=x) = p^x(1-p) + (1-p)^xp$ ($n=1,2,3,\dots$) and
  \[\begin{split}
    \E X = \sum_{x=1}^\infty xf_X(x)
    = (1-p)\sum_{x=1}^\infty xp^x + p\sum_{x=1}^\infty x(1-p)^x 
    = \frac{p}{1-p}+\frac{1-p}{p}.
  \end{split}\]
\end{solution}

\paragraph{15.}
\begin{proof}
  Clear that $X+Y=(X\lor Y)+(X\land Y)$. Hence,
  \[
    \E(X\lor Y) = \E(X+Y-(X\land Y)) = \E X+\E Y-\E(X\land Y).
  \]
\end{proof}

\paragraph{18.}
\begin{proof}
  \[\begin{split}
    \E|x-a| &= \int_{-\infty}^\infty |x-a|f(x)\rd x  \\
    &= \int_{-\infty}^a(a-x)f(x)\rd x + \int_a^\infty(x-a)f(x)\rd x \\
    &= a\int_{-\infty}^a f(x)\rd x - \int_{-\infty}^axf(x)\rd x
       + \int_a^\infty xf(x)\rd x - a\int_a^\infty f(x)\rd x.
  \end{split}\]
  Differentiating the both sides yields
  \[\begin{split}
    \frac{\rd}{\rd a}\E|x-a| 
    &= \left(\int_{-\infty}^a f(x)\rd x + af(a)\right) 
       - af(a) - af(a) - 
       \left( \int_a^\infty f(x)\rd x - af(a)\right) \\
    &= \int_{-\infty}^af(x)\rd x - \int_a^\infty f(x)\rd x.
  \end{split}\]
  Meanwhile, the value at $m$ of the twice derivative is greater than $0$.
  Therefore, $x=m$ is the minimum point.
\end{proof}

\paragraph{24.(a)}
\begin{solution}
  \begin{align*}
    \E X &= \int_0^1 xax^{a-1}\rd x = \frac{a}{a+1}. \\
    \E(X^2) &= \int_0^1 x^2ax^{a-1}\rd x = \frac{a}{a+2}. \\
    \Var X &= (\E X)^2 - \E(X^2) = \frac{a}{(a+1)(a+2)}.
  \end{align*}
\end{solution}

\paragraph{32}
\begin{proof}
  \begin{align*}
    \frac{\rd}{\rd t}S(t)
    = \frac{\rd}{\rd t}\left(\log\int_{-\infty}^\infty e^{tx}f(x)\rd x\right)
    = \frac{1}{\int_{-\infty}^\infty e^{tx}f(x)\rd x}
       \int_{-\infty}^\infty xe^{tx}f(x)\rd x.
  \end{align*}
  Therefore,
  \[
    S^\prime(0) 
    = \frac{1}{\int_{-\infty}^\infty f(x)\rd x}
      \int_{-\infty}^\infty xf(x) \rd x = \E X.
  \]
\end{proof}



