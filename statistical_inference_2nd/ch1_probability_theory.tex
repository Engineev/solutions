\section{Probability Theory}

\paragraph{1.5(a)}
\begin{solution}
  A U.S. birth results in female identical twins.  
\end{solution}
\paragraph{1.5(b)}
\begin{solution}
  \[
    P(A\cap B\cap C) = 
    P(A|B\cap C)P(B|C)P(C) = 
    P(A|B)P(B|C)P(C) =
    \frac{1}{2}\frac{1}{3}\frac{1}{90}=\frac{1}{540}.
  \]
\end{solution}

\paragraph{1.24(b)}
\begin{solution}
  Suppose $E_i = \{\text{head first appears on $i$th toss}\}$, then
  \[
    P(\text{A wins}) = P(\bigcup_{i=k}^\infty E_{2k+1}) 
    = \sum_{k=0}^\infty P(E_{2k+1}) = \sum_{k=0}^\infty p(1-p)^{2k+1}
    = \frac{p}{1-(1-p)^2}.
  \]
\end{solution}

\paragraph{1.31(a)}
\begin{proof}
  To get the average $(x_1+\cdots+x_n)/n$, we need the unordered sample
  to be $\{x_1,x_2,\dots,x_n\}$. The number of ordered samples which results
  in it is $n!$ and there are $n^n$ ordered samples in total. Hence, the 
  probability is $n!/n^n$.\par
  For any other resulted average, there will exist some double counting when
  counting the ordered samples. Therefore, the outcome with average
  $(x_1+\cdots+x_n)/n$ is most likely.
\end{proof}

\paragraph{1.33}
\begin{solution}
  \[\begin{split}
    P(\text{male}|\text{color-blind}) &=
    P(\text{color-blind}|\text{male})\frac{P(\text{male})}{P(\text{color-blind})} \\
    &= 0.05\times \frac{0.5}{0.5\times 0.05 + 0.5\times 0.0025}  \\
    &= \frac{20}{21} = 0.9524.
  \end{split}\]
\end{solution}

\paragraph{1.36}
\begin{solution}
  The probabilities of all shots being missed and the target being hit exactly
  once are $(4/5)^5=0.32768$ and $5\times(1/5)(4/5)^4=0.4096$ respectively.
  Hence, 
  \[
    P(\text{being hit at least twice}) = 1 - 0.32768 - 0.4096 = 0.26272.
  \]
  And
  \[\begin{split}
    &P(\text{being hit at least twice}|\text{being hit at least once}) \\
    =& \frac{P(\text{being hit at least twice})}
       {P(\text{being hit at least once})}
    = \frac{0.26272}{0.4096} = 0.6414.
  \end{split}\]
\end{solution}

\paragraph{1.39(a)}
\begin{proof}
  $A$ and $B$ are mutually exclusive means that $A\cap B=\varnothing$. Hence,
  $P(A\cap B)=0$. However, $P(A),P(B)>0$. Therefore, $P(A\cap B)\ne P(A)P(B)$.
\end{proof}
\paragraph{1.39(b)}
\begin{proof}
  As $A$ and $B$ are independent, $P(A\cap B)=P(A)P(B)>0$, which implies that 
  $A\cap B\ne\varnothing$.
\end{proof}
\paragraph{Notes on 1.39}
  An intuitive proof: Since $A$ and $B$ are mutually exclusive, if we know that
  $A$ did not happen, then the possibility that $B$ happened will increase. 
  Hence, they are not independent.
% end

\paragraph{1.52}
\begin{proof}
  Clear that $g(x)\ge 0$ for all $x\in\mathbb{R}$ and
  \[\begin{split}
    \int_{-\infty}^\infty g(x)\rd x 
    &= \int_{x_0}^\infty \frac{f(x)}{1-F(x_0)} \rd x \\
    &= \frac{1}{1-F(x_0)}\left(
         \int_{-\infty}^\infty f(x)\rd x - \int_{-\infty}^{x_0}f(x)\rd x 
       \right) \\
    &= \frac{1}{1-F(x_0)}(1- F(x_0)) = 1.
  \end{split}\]
  Hence, by Theorem 1.6.5, $g(x)$ is a pdf.
\end{proof}