%---------%---------%---------%---------%---------%---------%---------%---------
\section{Inner Product Spaces. Hilbert Spaces}
\subsection{Inner Product Spaces. Hilbert Spaces}
  \paragraph{2.}
  \begin{proof}
    \[
      \|x+y\|^2 = \langle x+y, x+y\rangle = \|x\|^2+\|y\|^2+2\langle x, y\rangle
      =\|x\|^2+\|y\|^2,
    \]
    where the last equality comes from the hypothesis of orthogonality. Now we
    show that for mutually orthogonal $x_1,\dots,x_m$ 
    \[
      \left\|\sum_{i=1}^m x_i\right\|^2 = \sum_{i=1}^m\|x_i\|^2,
    \]
    by induction on $m$. The case where $m=2$ has already been showed and we
    assume that the equation holds for $m-1$. Since $x_m$ is orthogonal with
    each $i=1,\dots,m-1$, $x_m$ is orthogonal to $x_1+\cdots+x_{m-1}$. Hence,
    \[
      \left\|\sum_{i=1}^m x_i\right\|^2 = 
      \left\|\sum_{i=1}^{m-1}x_i\right\|^2 + \|x_m\|^2 = 
      \sum_{i=1}^m\|x_i\|^2,
    \]
    completing the proof.
  \end{proof}
  
  \paragraph{3.}
  \begin{proof}
    The equation implies $\langle x, y\rangle+\langle y, x\rangle=0$. The
    symmetric property of real inner products implies $\langle x, y\rangle=0$.
    Let $X=\mathbb{C}$ and $x=1, y=i$. It is easy to verify that $\|x+y\|^2=
    \|x\|^2+\|y\|^2=2$ but $x$ and $y$ are not orthogonal.
  \end{proof}
  
  \paragraph{7.}
  \begin{proof}
    It suffices to show that the zero vector is the only vector orthogonal to
    all vectors. Suppose that $\langle x_0, x\rangle=0$ for all $x\in X$, then
    $\|x_0\|^2=\langle x_0, x_0\rangle=0$. By the definiteness of the inner
    product, $x_0=0$.
  \end{proof}
  
  \paragraph{8.}
    We show that any norm satisfying the parallelogram equality can be derived
    form an inner product.
  \begin{proof}
    The proof of (IP3) is trivial and (IP4) follows immediately from the
    positive-definiteness of the norm. Hence we only show the linearity in the
    first factor here. For every $u, v, y\in X$, from the parallelogram equality
    we can derive, after some computation, that
    \begin{align*}
      4\langle u+v, y\rangle &= \|u+v+y\|^2-\|u+v-y\|^2  \\
      &= \|u+y\|^2-\|u-y\|^2+\|v+y\|^2-\|v-y\|^2 \\
      &= 4\langle u, y\rangle + 4\langle v, y\rangle.
    \end{align*}
    Namely, (IP1) holds. By induction we can show that $\langle nu, y\rangle=
    n\langle u, y\rangle$ for $n=1,2,\dots$. And since $\langle -u, y\rangle =
    \langle 0-u, y\rangle = \langle 0, y\rangle - \langle u, y\rangle = \langle
    u, y\rangle$, 
    \[
      \langle nu, y\rangle = n\langle u, y\rangle,
      \quad\text{for $n\in\mathbb{Z}$}.
    \]
    Furthermore, for any positive integer $m$,
    \[
      m\left\langle\frac{n}{m}u, y\right\rangle = 
      mn\left\langle\frac{1}{m}u, y\right\rangle =
      n\langle u, y\rangle.
    \]
    Dividing the both sides by $m$ yields
    \[
      \langle qu, y\rangle = q\langle u, y\rangle,
      \quad\text{for $q\in\mathbb{Q}$}.
    \]
    For every $\alpha\in\mathbb{R}$, let $(q_n)\subset\mathbb{Q}$ converges to
    $\alpha$. Now we show that $f(t)=\langle tu, y\rangle$ is continuous at $t=
    0$ and by the additivity we may conclude that $f$ is continuous on
    $\mathbb{R}$. Since
    \begin{align*}
      4|f(t)| 
      &= |\|tu+y\|^2 - \|tu-y\|^2| \\
      &= (\|tu+y\|+\|tu-y\|)|\|tu+y\|-\|tu-y\|| \\
      &\le 4t\|u\|(t\|u\|+\|y\|) \to 0
    \end{align*}
    as $t\to 0$, $f(t)$ is continuous. For every $\alpha\in\mathbb{R}$, let 
    $(q_n)\subset\mathbb{Q}$ be a convergent sequence with limit $\alpha$. Then
    \[
      \langle\alpha u, y\rangle=
      \lim\langle q_nu, y\rangle=
      \lim q_n\langle u, y\rangle=
      \alpha\langle u, y\rangle.
    \]
    Hence, $\langle\cdot,\cdot\rangle$ is linear in the first factor. Thus, it
    is an inner product. Meanwhile, it is easy to verify that the norm it 
    introduces is exactly the original norm. 
  \end{proof}
% end

\subsection{Further Properties of Inner Product Spaces}
  \paragraph{7.}
  \begin{proof}
    First we note that
    \[
      f(\alpha)=\|x+\alpha y\|^2-\|x-\alpha y\|^2 
      = 2\bar{\alpha}\langle x,y\rangle + 2\alpha\langle y,x\rangle.
    \]
    Clear that $x\perp y$ implies $f(\alpha)=0$ for all scalar $\alpha$. For the
    converse, we suppose $f(\alpha)=0$ and put $\alpha=\langle x, y\rangle$. 
    Then $0=f(\alpha)=2|\langle x, y\rangle|$. Thus, $x\perp y$.
  \end{proof}
  
  \paragraph{8.}
  \begin{proof}
    Clear that $x\perp y$ implies $\|x+\alpha y\|\ge \|x\|$. Therefore we only
    show the converse here. Without loss of generality, we assume $\|y\|=1$. 
    Then $\|x+\alpha y\|\ge \|x\|$ for all scalar $\alpha$ implies
    \[
      |\alpha|^2 + 
      \bar{\alpha}\langle x, y\rangle + \alpha\overline{\langle x, y\rangle}
      \ge 0.
    \]
    Put $\alpha=-\langle x, y\rangle$ and we get
    \[
      0 \le |\langle x, y\rangle|^2 - 2|\langle x, y\rangle|^2
      =-|\langle x, y\rangle|^2,
    \]
    which implies $\langle x, y\rangle=0$. Namely, $x\perp y$.
  \end{proof}
  
  \paragraph{9.}
  \begin{proof}
    For every $\vep>0$, put $\delta=\vep/\sqrt{b-a}$. Then for every $x_1,x_2\in
    V$ with $\|x_1-x_2\|_\infty<\delta$,
    \[
      \|x_1-x_2\|_2^2 = \int_a^b|x_1(t)-x_2(t)|^2\rd t
      \le (b-a)\delta^2 = \vep^2.
    \]
    Hence, $x\mapsto x$ is continuous.
  \end{proof}
  
  \paragraph{10.}
  \begin{proof}
    For every $u, w\in X$,
    \begin{align*}
      \langle Tu, w\rangle = 
      & \frac{1}{4}(\langle T(u+w), u+w \rangle - \langle T(u-w), u-w \rangle)\\
      &+ \frac{i}{4}
        (\langle T(u+iw), u+iw\rangle - \langle T(u-iw), u-iw\rangle).\\
    \end{align*}
    Note that each component of the right hand side is of form $\langle Tx, x
    \rangle$ and hence equals to $0$. Putting $w=Tu$ yields $\|Tu\|^2=0$ for all
    $u\in X$. Thus, $T=0$.
  \end{proof}
% end

\subsection{Orthogonal Complements and Direct Sums}
  \paragraph{7.}
  \begin{proof}
    $\,$\par
    (a) $x\in A^{\perp\perp}$ iff for all $y\in A^\perp$, $\langle x, y\rangle=
    0$. By the definition of $A^\perp$, the identity holds if $x\in A$. Hence,
    $A\subset A^{\perp\perp}$.\par
    (b) For all $x\in B^\perp$ and $y\in A\subset B$, $\langle x, y\rangle=0$ by
    definition. Hence, $x\in A^\perp$. Namely, $B^\perp\subset A^\perp$.\par
    (c) We show that $A^\perp$ is closed (no matter whether $A$ is or not) and
    invoke Lemma 3.3-6 to complete the proof. Suppose that $(x_n)\subset 
    A^\perp$ converges to $x$. For all $y\in A$, $\langle x_n, y\rangle=0$. By
    the continuity of the inner product, $\langle x, y\rangle=0$ and therefore
    $x\in A^\perp$. Hence, $A^\perp$ is closed. Thus, $A^\perp=A^{\perp\perp
    \perp}$.
  \end{proof}
  
  \paragraph{8.}
  \begin{proof}
    We have show this in Prob. 7.
  \end{proof}
  
  \paragraph{9.}
  \begin{proof}
    It has been shown in Lemma 3.3-6 that the closedness of $Y$ implies $Y=Y^{
    \perp\perp}$. Hence we only show the converse here. For every convergent 
    $(x_n)\subset Y$, $(x_n)\subset Y^{\perp\perp}$. Since $Y^{\perp\perp}$ is
    closed by Prob. 8, the limit $x$ of $(x_n)$ belongs to $Y^{\perp\perp}$ and
    hence belongs to $Y$. Thus, $Y$ is closed.
  \end{proof}
  
  \paragraph{10.} TODO
% end

\subsection{Orthonormal Sets and Sequences}
  \paragraph{3.}
  \begin{proof}
    The situation where $x$ and $y$ are linearly dependent is obvious and hence
    we assume they are linearly independent here. By the homogeneity of the
    Schwarz inequality, we may assume without loss of generality that $\|x\|=
    \|y\|=1$. Put $z=(y-x\langle y,x\rangle)/\|y-x\langle y,x\rangle\|$. Then
    $\{x,z\}$ is orthonormal and therefore by (12*)
    \[
      |\langle y,x\rangle|^2+|\langle y,z\rangle|^2 \le \|y\|^2 = 1.
    \]
    Since $|\langle y,z\rangle|^2$ is nonnegative, this implies $|\langle x,y
    \rangle|^2\le 1$, the Schwarz inequality.
  \end{proof}
  
  \paragraph{7.}
  \begin{proof}
    For each positive integer $n$, by the Schwarz inequality and (12*),
    \[
      \sum_{k=1}^n|\langle x,e_k\rangle\langle y,e_k\rangle|\le
      \sqrt{\sum_{i=1}^n|\langle x,e_k\rangle|^2}
      \sqrt{\sum_{i=1}^n|\langle y,e_k\rangle|^2} \le
      \|x\|\|y\|.
    \]
    Since all terms in the summation is nonnegative, this implies $\sum_{k=1}
    ^\infty|\langle x,e_k\rangle\langle y,e_k\rangle|\le\|x\|\|y\|$.
  \end{proof}
  
  \paragraph{8.}
  \begin{proof}
    It follows immediately from Bessel inequality.
  \end{proof}
% end

\subsection{Series Related to Orthonormal Sequences}
  \paragraph{1.}
  \begin{proof}
    By Theorem 3.5-2, $\alpha_k=\langle x,e_k\rangle$. Meanwhile by the 
    definition of the norm,
    \[
      \|x\|^2 = 
      \left\langle \sum_{k=1}^\infty\langle x,e_k\rangle e_k, x\right\rangle =
      \sum_{k=1}^\infty \langle x,e_k\rangle\langle e_k, x\rangle =
      \sum_{k=1}^\infty|\alpha_k|^2,
    \]
    where the second equality follows from the continuity of the inner product.
  \end{proof}
  
  \paragraph{3.}
  \begin{solution}
    Put $x\equiv 1$ on $[-\pi,\pi]$ and $e_k=\sin kt$. Since $x$ is even but 
    $e_k$ is odd for every $k$, the series does not converges to $x$.
  \end{solution}
  
  \paragraph{4.}
  \begin{proof}
    By the triangle inequality, $\|x_m+\cdots+x_n\| \le \|x_m\|+\cdots+\|x_n\|$
    for every $n\ge m>0$. Hence the convergence of $\sum \|x_k\|$ implies that
    $s_n$ is a Cauchy sequence.
  \end{proof}
  
  \paragraph{5.}
  \begin{proof}
    By Prob. 4, $\sum_{k=1}^n x_k$ is a Cauchy sequence. And since $H$ is 
    complete, $\sum_{k=1}^\infty x_k$ converges.
  \end{proof}
  
  \paragraph{7.}
  \begin{proof}
    The existence of $y$ follows from Theorem 3.5-2(c). And for each $k$,
    \[
      \langle x-y,e_k\rangle =
      \langle x,e_k\rangle - 
      \sum_{j=1}^\infty\langle x,e_j\rangle\langle e_k,e_j\rangle =
      \langle x,e_k\rangle - \langle x,e_k\rangle = 0,
    \]
    where the second equality comes from the fact that $(e_k)$ is orthonormal.
  \end{proof}
  
  \paragraph{8.}
    TODO: Show the validation of the change of the order of summation. Or maybe
    we can show the equality directly.
  \begin{proof}
    We suppose that $x\in\bar{M}$ here since the proof of the other direction is
    obvious. Then there exists $(p_n)\subset M$ such that $x=\sum_{n=1}^\infty
    p_n$. For each $n$, suppose $p_n=\sum_{k=1}^\infty \langle p_n,e_k\rangle 
    e_k$. This is valid because $p_n\in M$ and therefore is a finite linear 
    combination of $(e_k)$. In fact, there are only finitely many nonzero term
    in the summation. Then
    \[
      x=\sum_{n=1}^\infty p_n = 
      \sum_{n=1}^\infty\sum_{k=1}^\infty\langle p_n,e_k\rangle e_k=
      \sum_{k=1}^\infty\left(\sum_{n=1}^\infty\langle p_n,e_k\rangle\right)e_k.
    \]
  \end{proof}
  
  \paragraph{9.}
  \begin{proof}
    First we suppose $\bar{M}_1=\bar{M}_2$. Then by Prob. 8, each $e_n$ and
    $\tilde{e}_n$ can be represented by (a) and (b) respectively.\par
    For the converse, (a) implies, again by Prob. 8, $e_n\in\bar{M}_2$ and
    therefore $M_1\subset\bar{M}_2$. Since $\bar{M}_2$ is closed, $\bar{M}_1
    \subset\bar{M}_2$. \textit{Mutatis mutandis}, this also shows $\bar{M}_2
    \subset\bar{M}_1$. Thus, $\bar{M}_1=\bar{M}_2$.
  \end{proof}
  
  \paragraph{10.}
  \begin{proof}
    Note that for every $m>0$, there are only finite $e_\kappa$ such that $
    \langle x,e_\kappa\rangle\ge 1/m$. Otherwise we may choose a countable
    subset of them, which will violate the result in Prob. 8, Sec 3.4. Hence,
    the collection of all nonzero Fourier coefficient
    \[
      \bigcup_{m=1}^\infty\{ e_\kappa:\, \langle x,e_\kappa\rangle\ge 1/m \}
    \]
    is at most countable.
  \end{proof}
% end

\subsection{Total Orthonormal Sets and Sequences}
  \paragraph{4.}
  \begin{proof}
    Suppose that $x$ and $y$ satisfy (3). We only show the relation for real 
    cases here. The complex cases can be proved in a similar way. Using (9),
    Sec 3.1 and (3),
    \[
      \langle x,y\rangle = \frac{1}{4}(\|x+y\|^2-\|x-y\|^2) =
      \frac{1}{4}\sum_k
      \left(|\langle x+y,e_k\rangle|^2-|\langle x-y,e_k\rangle|^2 \right).
    \]
    Meanwhile,
    \[
      |\langle x\pm y,e_k\rangle|^2=
      \langle x\pm y,e_k\rangle\overline{\langle x\pm y,e_k\rangle}=
      |\langle x,e_k\rangle|^2+|\langle y,e_k\rangle|^2\pm
      2\langle x,e_k\rangle\overline{\langle y,e_k\rangle}.
    \]
    Hence, $\langle x,y\rangle=\sum_k\langle x,e_k\rangle\overline{\langle
    y,e_k\rangle}$.
  \end{proof}
  
  \paragraph{6.}
  \begin{proof}
    Suppose $M=(e_k)$. We collect the $e_k$ which does not belong to $\spn(e_1,
    \dots,e_{k=1})$ and denote the new sequence by $(\tilde{e}_k)$. Clear that
    $\spn(e_k)=\spn(\tilde{e}_k)$ and $(\tilde{e}_k)$ is linearly independent.
    Let $(f_k)$ be the sequence generated from $(\tilde{e}_k)$ by the
    Gram-Schmidt process. Then clear that $(f_k)$ is orthonormal. And since for
    every $n$, $\spn(\tilde{e}_1,\dots,\tilde{e}_n)=\spn(f_1,\dots,f_n)$,
    $M\subset\spn(\tilde{e}_k)=\spn(f_k)$. Finally, since $M$ is dense in $H$,
    $\overline{\spn(f_k)}=H$. Thus, $(f_k)$ is a total orthonormal sequence of
    $H$.
  \end{proof}
  
  \paragraph{7.}
  \begin{proof}
    It follows from the definition of the separable Hilbert space and Prob. 6
    immediately.
  \end{proof}
  
  \paragraph{9.}
  \begin{proof}
    $\langle v,x\rangle=\langle w,x\rangle$ implies $\langle v-w,x\rangle=0$ for
    all $x\in M$, that is, $x\perp M$. Since $M$ is total, by Theorem 3.6-2, $v
    -w=0$.
  \end{proof}
  
  \paragraph{10.}
  \begin{proof}
    It follows immediately from Theorem 3.6-2(b).
  \end{proof}
% end

















