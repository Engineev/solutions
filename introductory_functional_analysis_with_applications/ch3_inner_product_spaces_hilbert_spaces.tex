%---------%---------%---------%---------%---------%---------%---------%---------
\section{Inner Product Spaces. Hilbert Spaces}
\subsection{Inner Product Spaces. Hilbert Spaces}
  \paragraph{2.}
  \begin{proof}
    \[
      \|x+y\|^2 = \langle x+y, x+y\rangle = \|x\|^2+\|y\|^2+2\langle x, y\rangle
      =\|x\|^2+\|y\|^2,
    \]
    where the last equality comes from the hypothesis of orthogonality. Now we
    show that for mutually orthogonal $x_1,\dots,x_m$ 
    \[
      \left\|\sum_{i=1}^m x_i\right\|^2 = \sum_{i=1}^m\|x_i\|^2,
    \]
    by induction on $m$. The case where $m=2$ has already been showed and we
    assume that the equation holds for $m-1$. Since $x_m$ is orthogonal with
    each $i=1,\dots,m-1$, $x_m$ is orthogonal to $x_1+\cdots+x_{m-1}$. Hence,
    \[
      \left\|\sum_{i=1}^m x_i\right\|^2 = 
      \left\|\sum_{i=1}^{m-1}x_i\right\|^2 + \|x_m\|^2 = 
      \sum_{i=1}^m\|x_i\|^2,
    \]
    completing the proof.
  \end{proof}
  
  \paragraph{3.}
  \begin{proof}
    The equation implies $\langle x, y\rangle+\langle y, x\rangle=0$. The
    symmetric property of real inner products implies $\langle x, y\rangle=0$.
    Let $X=\mathbb{C}$ and $x=1, y=i$. It is easy to verify that $\|x+y\|^2=
    \|x\|^2+\|y\|^2=2$ but $x$ and $y$ are not orthogonal.
  \end{proof}
  
  \paragraph{7.}
  \begin{proof}
    It suffices to show that the zero vector is the only vector orthogonal to
    all vectors. Suppose that $\langle x_0, x\rangle=0$ for all $x\in X$, then
    $\|x_0\|^2=\langle x_0, x_0\rangle=0$. By the definiteness of the inner
    product, $x_0=0$.
  \end{proof}
  
  \paragraph{8.}
    We show that any norm satisfying the parallelogram equality can be derived
    form an inner product.
  \begin{proof}
    The proof of (IP3) is trivial and (IP4) follows immediately from the
    positive-definiteness of the norm. Hence we only show the linearity in the
    first factor here. For every $u, v, y\in X$, from the parallelogram equality
    we can derive, after some computation, that
    \begin{align*}
      4\langle u+v, y\rangle &= \|u+v+y\|^2-\|u+v-y\|^2  \\
      &= \|u+y\|^2-\|u-y\|^2+\|v+y\|^2-\|v-y\|^2 \\
      &= 4\langle u, y\rangle + 4\langle v, y\rangle.
    \end{align*}
    Namely, (IP1) holds. By induction we can show that $\langle nu, y\rangle=
    n\langle u, y\rangle$ for $n=1,2,\dots$. And since $\langle -u, y\rangle =
    \langle 0-u, y\rangle = \langle 0, y\rangle - \langle u, y\rangle = \langle
    u, y\rangle$, 
    \[
      \langle nu, y\rangle = n\langle u, y\rangle,
      \quad\text{for $n\in\mathbb{Z}$}.
    \]
    Furthermore, for any positive integer $m$,
    \[
      m\left\langle\frac{n}{m}u, y\right\rangle = 
      mn\left\langle\frac{1}{m}u, y\right\rangle =
      n\langle u, y\rangle.
    \]
    Dividing the both sides by $m$ yields
    \[
      \langle qu, y\rangle = q\langle u, y\rangle,
      \quad\text{for $q\in\mathbb{Q}$}.
    \]
    For every $\alpha\in\mathbb{R}$, let $(q_n)\subset\mathbb{Q}$ converges to
    $\alpha$. Now we show that $f(t)=\langle tu, y\rangle$ is continuous at $t=
    0$ and by the additivity we may conclude that $f$ is continuous on
    $\mathbb{R}$. Since
    \begin{align*}
      4|f(t)| 
      &= |\|tu+y\|^2 - \|tu-y\|^2| \\
      &= (\|tu+y\|+\|tu-y\|)|\|tu+y\|-\|tu-y\|| \\
      &\le 4t\|u\|(t\|u\|+\|y\|) \to 0
    \end{align*}
    as $t\to 0$, $f(t)$ is continuous. For every $\alpha\in\mathbb{R}$, let 
    $(q_n)\subset\mathbb{Q}$ be a convergent sequence with limit $\alpha$. Then
    \[
      \langle\alpha u, y\rangle=
      \lim\langle q_nu, y\rangle=
      \lim q_n\langle u, y\rangle=
      \alpha\langle u, y\rangle.
    \]
    Hence, $\langle\cdot,\cdot\rangle$ is linear in the first factor. Thus, it
    is an inner product. Meanwhile, it is easy to verify that the norm it 
    introduces is exactly the original norm. 
  \end{proof}

% end




















