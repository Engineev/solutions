%---------%---------%---------%---------%---------%---------%---------%---------
\section{Fundamental Theorems for Normed and Banach Spaces}
\setcounter{subsection}{1}
\subsection{Hahn-Banach Theorem}
  \paragraph{4.}
  \begin{proof}
    By the positive homogeneity, $p(2\times 0)=2p(0)$. Hence, $p(0)=0$. 
    Consequently, $0=p(x+(-x))\le p(x)+p(-x)$. Thus, $-p(x)\le p(-x)$.
  \end{proof}
  
  \paragraph{5.}
  \begin{proof}
    For every $x,y\in M$ and $\lambda\in[0,1]$,
    \[
      p(\lambda x+(1-\lambda)y)\le \lambda p(x)+(1-\lambda)p(y)
      \le \lambda\gamma+(1-\lambda)\gamma=\gamma.
    \]
    Hence, $\lambda x+(1-\lambda)y\in M$. Thus, $M$ is convex.
  \end{proof}
  
  \paragraph{6.}
  \begin{proof}
    For every $x,t\in X$,
    \[
      p(x-t)\le p(x)+p(-t) \quad\Rightarrow\quad
      p(x-t)-p(x)\le p(-t),
    \]
    and
    \[
      p(x)=p(x-t+t)\le p(x-t)+p(t) \quad\Rightarrow\quad
      -p(t)\le p(x-t)-p(x).
    \]
    Since $p(0)=0$ and $p$ is continuous at $0$, $p(t)\to 0$ and $p(-t)\to 0$ as
    $t\to 0$. Hence, $p(x-t)-p(x)\to 0$ as $t\to 0$, that is, $p$ is continuous
    on $X$.
  \end{proof}
  
  \paragraph{8.}
  \begin{proof}
    First, $p(0)\ge p(0+0)-p(0)=0$. For nonzero $x$, we argue by contradiction.
    Assume that there exists some $x$ with $0<\|x\|\le r$ such that $p(x)<0$.
    Then $np(x)<0$ for $n=1,2,\dots$. For $n$ sufficiently large, $n\|x\|>r$ and 
    therefore $p(nx)\ge 0$. However, by the subadditivity, $p(nx)\le np(x)<0$.
    Contradiction. Thus, $p(x)\ge 0$ on $X$.
  \end{proof}
  
  \paragraph{9.}
  \begin{proof}
    For all $x_1=\alpha_1x_0, x_2=\alpha_2x_0\in Z$ and scalars $a_1$ and $a_2$,
    \begin{align*}
      f(a_1x_1+a_2x_2)&=f((a_1\alpha_1+a_2\alpha_2)x_0)=
      (a_1\alpha_1+a_2\alpha_2)p(x_0) \\
      &=a_1\alpha_1p(x_0)+a_2\alpha_2p(x_0)=a_1f(x_1)+a_2f(x_2).
    \end{align*}
    Thus, $f$ is linear. Now we show that for $\alpha\in\mathbb{R}$, $\alpha p
    (x_0)\le p(\alpha x_0)$ to complete the proof. If $\alpha\ge 0$, then it
    follows from the positive homogeneity. For negative $\alpha$, $\alpha p
    (x_0)=-p(-\alpha x_0)$ and by Prob. 4, $-p(-\alpha x_0)\le p(\alpha x_0)$.
    Thus, $f(x)\le p(x)$ for all $x\in Z$.
  \end{proof}
  
  \paragraph{10.}
  \begin{proof}
    Let $Z$ and $f$ have the same meaning as in Prob. 9. By Hahn-Banach theorem,
    there exists a linear extension $\tilde{f}$ of $f$ to $X$ with $\tilde{f}(x)
    \le p(x)$ for all $x\in X$. Replacing $x$ with $-x$ gives $\tilde{f}(-x)\le
    p(x)$. Finally, the linearity of $\tilde{f}$ yields $-p(-x)\le\tilde{f}(x)$.
  \end{proof}
% end

\subsection{Hahn-Banach Theorem for Normed Spaces}
  \paragraph{1.}
  \begin{proof}
    By (2), $p(2\times 0)=2p(0)$. Hence, $p(0)=0$. And for every $x\in X$, by 
    (1),
    \[
      0=p(0)\le p(x)+p(-x)=2p(x),
    \]
    that is, $p(x)\ge 0$.
  \end{proof}
  
  \paragraph{2.}
  \begin{proof}
    By (1), $p(x)=p(x-y+y)\le p(x-y)+p(y)$. Therefore, $p(x)-p(y)\le p(x-y)$.
    Interchange the roles of $x$ and $y$ and we obtain $p(y)-p(x)\le p(y-x)=
    p(x-y)$, where the equality comes from (2). Thus, $|p(x)-p(y)|\le p(x-y)$.
  \end{proof}
  
  \paragraph{7.}
  \begin{proof}
    Define $\tilde{f}$ to be $x\mapsto\langle x, x_0/\|x_0\|\rangle$. Clear that
    it is a bounded linear functional on $X$ and $\tilde{f}(x_0)=\|x_0\|$. And
    by Riesz's Theorem, $\|\tilde{f}\|=\|x_0/\|x_0\|\|=1$.
  \end{proof}
  
  \paragraph{8.}
  \begin{proof}
    It follows immediately from Theorem 4.3-3.
  \end{proof}
  
  \paragraph{13.}
  \begin{proof}
    Just put $\hat{f}=\tilde{f}/\|x_0\|$.
  \end{proof}
  
  \paragraph{14.}
  \begin{proof}
    By Prob 13, there exists a $\hat{f}\in X\hp$ such that $\|\hat{f}\|=1/r$ and
    $\hat{f}(x_0)=1$. Let hyperplane $H_0=\{x\in X:\, \hat{f}(x)=1\}$ and half
    space $S_0=\{x\in X:\, \hat{f}(x)\le 1\}$. Then clear that $x_0\in H_0$ and
    for all $x\in S(0;r)$, $f(x) \le \|f\|\|x\| = r/r =1$. Hence, $x\in S_0$.
  \end{proof}
  
  \paragraph{15.}
  \begin{proof}
    If $\|x\|=c+2\vep>c$, then by Corollary 4.3-4, there exists some $0\ne f\in
    X\hp$ such that $|f(x)|/\|f\|\ge c+\vep$. Consequently, the functional $g=f/
    \|f\|$, which is of norm $1$, is such that $|g(x_0)|>c$. Contradiction.
  \end{proof}
% end
\setcounter{subsection}{4}
\subsection{Adjoint Operator}
  \paragraph{9.}
  \begin{proof}
    Note that every bounded linear functional is continuous by Theorem 2.7-9. 
    Hence, $M^a=(\mathcal{R}(T))^a$. Thus, $g\in M^a$ iff $g\in(\mathcal{R}
    (T))^a$ iff $g(Tx)=(T^\times g)(x)=0$ for all $x\in X$ iff $T^\times g=0$
    iff $g\in\mathcal{N}(T^\times)$. Namely, $M^a=\mathcal{N}(T^\times)$.
  \end{proof}
  
  \paragraph{10.}
  \begin{proof}
    For every $y=Tx\in\mathcal{R}(T)$, $g(Tx)=(T^\times g)(x)=0$ for all $g\in
    \mathcal{N}(T^\times)$. Hence, $y\in ^a\mathcal{N}(T^\times)$.
  \end{proof}
% end














