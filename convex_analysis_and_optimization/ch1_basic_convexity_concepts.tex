\section{Basic Convexity Concepts}
\setcounter{subsection}{1}
\subsection{Convex Sets and Functions}
  \paragraph{1.}
  \begin{proof}
    For every $y\in(\lambda_1+\lambda_2)C$, there is an $x\in C$ such that
    \[
      y=(\lambda_1+\lambda_2)x=\lambda_1x+\lambda_2x.
    \]
    Since $\lambda_ix\in\lambda_iC$, ($i=1,2$,) $y\in\lambda_1C+\lambda_2C$. 
    Thus, $(\lambda_1+\lambda_2)C\subset\lambda_1C+\lambda_2C$. For the 
    converse, suppose that $y_i=\lambda_ix_i\in\lambda_iC$. Then
    \[
      \lambda_1x_1+\lambda_2x_2=
      (\lambda_1+\lambda_2)
      \left(\frac{\lambda_1}{\lambda_1+\lambda_2}x_1+
      \frac{\lambda_2}{\lambda_1+\lambda_2}x_2\right)=
      (\lambda_1+\lambda_2)z.
    \] 
    By the convexity of $C$, $z\in C$. Hence, $\lambda_1x_1+\lambda_2x_2\in
    (\lambda_1+\lambda_2)C$. Namely, $(\lambda_1+\lambda_2)C\supset\lambda_1C+
    \lambda_2C$.\par
    If $C$ is not convex, the statement may be false. For example, put $n=1$, 
    $C=\{0,1\}$ and $\lambda_1=\lambda_2=1$. Then $(\lambda_1+\lambda_2)C=
    \{0,2\}$ but $\lambda_1C+\lambda_2C=\{0,1,2\}$.
  \end{proof}
  
  \paragraph{2.(d, e)}
  \begin{proof}
    $\,$\par
    (d) Let $C$ be a cone and $x\in\bar{C}$. Then there is a sequence $\{x_k\}
    \subset C$ with $x_k\to x$. For every positive $\lambda$, Clear that 
    $\lambda x=\lim_{k\to\infty}\lambda x_k$ and $\lambda x_k\in C$. Namely,
    $\{\lambda x_k\}\subset C$ converges to $\lambda x$. Hence, $\lambda x\in
    \bar{C}$. Thus, $\bar{C}$ is a cone.\par
    (e) Let $T$ a linear transformation on $\mathbb{R}^n$. Suppose $y=Tx$ for
    some $x\in C$. Then $\lambda y=\lambda Tx=T(\lambda x)\in T(C)$. Hence, 
    $T(C)$ is a cone. Suppose that there is an $v\in C$ such that $Tu=v$. Then
    $T(\lambda u)=\lambda v\in C$. Hence, the inverse image is also a cone.
  \end{proof}
  
  \paragraph{3. Lower Semicontinuity under Composition}
  \begin{proof}
    $\,$\par
    (a) For every $x\in\mathbb{R}^n$ and $\{x_k\}$ converging to $x$, put $y_k
    =f(x_k)$. Since $f$ is continuous, $y_k\to y=f(x)$. Hence,
    \[
      \liminf_{k\to\infty}h(x)=\liminf_{k\to\infty}g(y_k)\ge g(y)=h(x).
    \]
    Namely, $h$ is lower semicontinuous.\par
    (b) First we show that for every $\{y_k\}\subset\mathbb{R}$, 
    \begin{equation}
      \label{eq:1.3}
      \liminf_{k\to\infty}g(y_k) \ge g\left(\liminf_{k\to\infty}y_k\right).
    \end{equation}
    Put $y=\liminf y_k$. Since $y_k\ge y$ for every $k$ and $g$ is 
    nondecreasing, $g(y_k)\ge g(y)$. Hence, $\liminf g(y_k)\ge g(y)$.\par
    For every $x\in\mathbb{R}^n$ and $\{x_k\}$ converging to $x$, put $y_k=f
    (x_k)$. Since $f$ is lower semicontinuous, $\liminf y_k\ge f(x)$. Hence,
    \[
      \liminf_{k\to\infty}h(x)=
      \liminf_{k\to\infty}g(y_k)\ge
      g\left(\liminf_{k\to\infty}y_k\right)\ge
      g(f(x)),
    \]
    where the second and third inequalities come from \eqref{eq:1.3} and the
    monotonicity of $g$ respectively. Thus, $h$ is lower semicontinuous.\par
    To show that the monotonic nondecrease assumption is essential, put $n=1$
    and define both $f$ and $g$ by
    \[
      f(x)=g(x)=\begin{cases}
        1, & x<0 \\
        -1,& x\ge 0
      \end{cases}.
    \]
    Clear that both $f$ and $g$ are lower semicontinuous but $h=g\circ f$ takes
    value $-1$ for $x<0$ and $1$ for $x\ge 0$ and therefore is not lower 
    semicontinuous.
  \end{proof}
  
  \paragraph{4. Convexity under Composition}
  \begin{proof}
    $\,$\par
    (a) For every $\lambda\in[0,1]$ and $x,y\in C$,
    \begin{align*}
      h(\lambda x+(1-\lambda)y)
      &=g(f(\lambda x+(1-\lambda)y))\\
      &\le g(\lambda f(x)+(1-\lambda)f(y)) \\
      &\le \lambda h(x)+(1-\lambda)h(y),
    \end{align*}
    where the first inequality comes from the monotonicity of $g$ and convexity
    of $f$, and the second one comes from the convexity of $g$. Thus, $h$ is
    convex. If $g$ is increasing and $f$ is strictly convex, then the first
    inequality is strict, provided $\lambda\in(0,1)$ and $x\ne y$. Therefore,
    $h$ is strictly convex.\par
    (b) It follows from a similar argument.
  \end{proof}
  
  \paragraph{5. Examples of Convex Functions}
  \begin{proof}
    $\,$\par
    (a) For every $x\in\dom f$,
    \[
      \nabla^2f_1(x)=
      K\left\{\left[\frac{1}{x_ix_j}\right]_{ij}
      -n\diag\left\{\frac{1}{x_i^2}\right\}_i\right\},
    \]
    where $K=-(x_1\cdots x_n)^{1/n}/n^2<0$. For each $y\in\dom f_1$,
    \begin{align*}
      y^T\nabla^2f_1(x)y
      &=\frac{K}{n^2}\left\{\left(\frac{\sum_{1=1}^n y_i/x_i}{n}\right)^2
        -\frac{1}{n}\sum_{i=1}^n\frac{y_i^2}{x_i^2}\right\}
      \ge 0,
    \end{align*}
    where the inequality comes from the RMS-AM inequality. Hence, $f_1$ is 
    convex.\par
    (b) For every $x\in\mathbb{R}^n$,
    \[
      \nabla^2f_2= 
      K\left\{\left[e^{x_i}e^{x_j}\right]_{ij}
      -\left(\sum_{i=1}^ne^{x_i}\right)\diag\{e^{x_i}\}\right\},
    \]
    where $K=-1/(e^{x_1}+\cdots+e^{x_n})^2<0$. For each $y\in\mathbb{R}^n$,
    put $a=(e^{x_1/2},\dots,e^{x_n/2})$ and $b=(y_1e^{x_1/2},\dots,
    y_ne^{x_n/2})$. Then
    \begin{align*}
      y^T\nabla^2f_2(x)y
      &=K\left\{\left(\sum_{i=1}^ny_ie^{x_i}\right)^2
       -\left(\sum_{i=1}^ne^{x_i}\right)
        \left(\sum_{i=1}^ny_i^2e^{x_i}\right)\right\}\\
      &=K\{(a^Tb)^2-(a^Ta)(b^Tb)\}
      \ge 0,
    \end{align*}
    where the inequality comes from the Cauchy-Schwarz inequality. Thus, $f_2$
    is convex.\par
    (c) Since $\|\cdot\|:\mathbb{R}^n\to[0,\infty)$ is convex over 
    $\mathbb{R}^n$ and the function $x\mapsto x^p$ ($p\ge 1$) is convex and 
    nondecreasing on $[0,\infty)$, $f_3$ is convex by Prob. 1.4(a).\par
    (d) $-f$ is convex and negative, and the function $x\mapsto -1/x$ is convex
    and nondecreasing on $(-\infty,0)$, so, by Prob. 1.4(a), $f_4=-1/(-f)$
    is convex.\par
    (e) The function $g:x\mapsto \alpha x+\beta$ is convex and nondecreasing on
    $\mathbb{R}$. Hence, $f_5=g\circ f$ is convex by Prob. 1.4(a).\par
    (f) The function $g:x\mapsto e^{\beta x}$ is convex and nondecreasing on 
    $\mathbb{R}$ and the function $h:x\mapsto x^TAx$ is convex since $A$ is
    positive semidefinite. Hence, by Prob. 1.4(a), $f_6=g\circ h$ is convex.
    \par
    (g) For every $x,y\in\mathbb{R}^n$ and $\lambda\in[0,1]$,
    \begin{align*}
      f_7(\lambda x+(1-\lambda)y)
      &=f(A(\lambda x+(1-\lambda)y)+b)\\
      &=f(\lambda(Ax+b)+(1-\lambda)(Ay+b))\\
      &\le\lambda f_7(x)+(1-\lambda)f_7(y),
    \end{align*}
    where the inequality comes from the convexity of $f$. Hence, $f_7$ is 
    convex.
  \end{proof}
  
  \paragraph{6. Ascent/Descent Behavior of a Convex Function}
  \begin{proof}
    $\,$\par
    (a) Let $\lambda\in(0,1)$ be such that $x_2=\lambda_1x_1+(1-\lambda)x_3$.
    Then
    \[
      \frac{f(x_2)-f(x_1)}{x_2-x_1}
      \le\frac{\lambda f(x_1)+(1-\lambda)f(x_3)-f(x_1)}
         {\lambda x_1+(1-\lambda)x_3-x_1}
      =\frac{f(x_3)-f(x_1)}{x_3-x_1}.
    \]
    Similarly, we can show that
    \[
      \frac{f(x_3)-f(x_2)}{x_3-x_2}\ge\frac{f(x_3)-f(x_1)}{x_3-x_1}.
    \]
    Thus, 
    \[
      \frac{f(x_2)-f(x_1)}{x_2-x_1}\le\frac{f(x_3)-f(x_2)}{x_3-x_2}.
    \]
  \end{proof}
  
  \paragraph{7. Characterization of Differentiable Convex Functions}
  \begin{proof}
    If $f$ is convex over $C$, then by Proposition 1.2.5,
    \[
      f(y)-f(x)\ge\nabla f(x)^T(y-x),\quad
      f(x)-f(y)\ge\nabla f(y)^T(x-y)
    \]
    for every $x,y\in C$. Sum up these two inequalities and we get
    \begin{equation}
      \label{eq:1.7}
      (\nabla f(x)-\nabla f(y))^T(x-y)\ge 0.
    \end{equation}\par
    For the converse, we first prove a lemma: If $h:(a,b)\to\mathbb{R}$ is
    differentiable and its derivative is nondecreasing, then it is convex.
    By the mean value theorem, for every $x,y\in(a,b)$, $h(y)-h(x)=h\hp(\xi)
    (y-x)$ where $\xi$ is between $x$ and $y$. Since $h\hp$ is nondecreasing, 
    this implies that $h(y)-h(x)\ge h\hp(x)(y-x)$. Thus, $h$ is convex.\par
    Now we suppose \eqref{eq:1.7} holds for every $x,y\in C$. Define $h:[0,1]
    \to\mathbb{R}^n$ by $h(t)=x+t(y-x)$ and put $g=f\circ h$. Then
    \[
      Dg(t)=\nabla f(h(t))^T(y-x).
    \]
    Hence, for $1\ge t_2>t_1\ge 0$,
    \[
      Dg(t_2)-Dg(t_1)=
      (\nabla f(h(t_2))-\nabla f(h(t_1))^T
      \frac{h(t_2)-h(t_1)}{t_2-t_1}\ge 0.
    \]
    Namely, $Dg$ is nondecreasing. By our lemma, $g$ is convex. Since the 
    choice of $x,y\in C$ are arbitrary, we conclude that $f$ is convex over 
    $C$.
  \end{proof}
  
  \paragraph{8. Characterization of Twice Continuously Differentiable Convex 
  Functions}
  \begin{proof}
    We may assume without loss of generality that $0\in C$ and, in consequence, 
    $S=\aff(C)$. If $\dim S=0$, then there is nothing to prove. Suppose $m=
    \dim S>0$, let $Z\in\Hom(\mathbb{R}^m,S)$ be isometric\footnote{Consider 
    the linear transformation $X$ which maps an orthonormal basis of 
    $\mathbb{R}^m$ to an orthonormal basis of S. It can be verified that $X$ is 
    an isometry and is bijective.} and define $g:\mathbb{R}^m\to\mathbb{R}$ by 
    $u\mapsto f(Zu)$. Clear that $g$ is also twice continuously differentiable 
    and $\nabla^2g=Z^T\nabla^2f Z$.\par
    First we suppose that $y^T\nabla^2f(x)y\ge 0$ for all $x\in C$ and $y\in 
    S$. Since $Z$ is an isometry, this implies that $u^TZ^T\nabla^2f(x)Zu\ge 
    0$ for all $u\in\mathbb{R}^m$. Namely, $\nabla^2g(x)$ is positive 
    semidefinite on $\mathbb{R}^m$. Therefore, by Prop. 1.2.6, $g$ is convex.
    Thus, $f=g\circ Z\inv$ is also convex.\par
    Now we suppose that $f$ is convex over $C$ and assume, to obtain a 
    contradiction, that there is some $x\in C$ and $y\in S$ such that $y^T
    \nabla^2f(x)y<0$. Suppose $y=Zu$. Then this implies that $u^T\nabla^2g(x)u
    <0$. However, since $g$ is convex (as $f$ is) and $\mathbb{R}^m$ is open,
    by Prop. 1.2.6(c), $\nabla^2g(x)$ should be positive semidefinite on
    $\mathbb{R}^m$. Contradiction. Thus, $y^T\nabla^2f(x)y\ge 0$ for all $x\in 
    C$ and $y\in S$.
  \end{proof}
  
  \paragraph{9. Strong Convexity}
  \begin{proof}
    $\,$\par
    (a) Note that (1.16) implies that when restricted to the line segment
    connecting $x$ and $y$, the function $f$ has strictly increasing gradient.
    Hence, the argument in Prob. 1.7, \textit{mutatis mutandis}, gives a proof 
    of (a).\par
    (b) First we suppose that $\nabla^2f(x)-\alpha I$ is positive semidefinite.
    Then for every $y,x\in\mathbb{R}^n$, there exists some $\theta\in(0,1)$ and 
    $z=x+\theta(y-x)$ such that
    \begin{equation}
      \label{eq:1.9.1}
    \begin{split}
      f(y)
      &=f(x)+\nabla f(x)^T(y-x)+(y-x)^T\nabla^2f(z)(y-x)\\
      &=f(x)+\nabla f(x)^T(y-x)+(y-x)^T(\nabla^2f(z)-\alpha I)(y-x)
        +\alpha\|y-x\|^2\\
      &\ge f(x)+\nabla f(x)^T(y-x)+\alpha\|y-x\|^2.
    \end{split}
    \end{equation}
    Meanwhile, since $\nabla^2f(x)$ is positive semidefinite, $f$ is convex and
    therefore
    \begin{equation}
      \label{eq:1.9.2}
      f(y)-f(x)\le \nabla f(y)^T(y-x).
    \end{equation}
    The previous two inequalities imply (1.16), i.e., $f$ is strongly convex 
    with coefficient $\alpha$.\par
    Now suppose that (1.16) holds. For fixed $x$, let $u\in\mathbb{R}^n$ and 
    $t\in\mathbb{R}$. Then there exists some $\theta_1,\theta_2\in(0,1)$ such
    that
    \begin{align*}
      f(x+tu)&=f(x)+\nabla f(x)^Ttu+\frac{t^2}{2}u^T\nabla^2f(x+\theta_1tu)u,\\
      f(x)&=f(x+tu)-\nabla f(x+tu)^Ttu+\frac{t^2}{2}u^T\nabla^2
            f(x+\theta_2tu)u.
    \end{align*}
    Add these two equations and we get
    \[
      \frac{t^2}{2}u^T(\nabla^2f(x+\theta_1tu)+\nabla^2f(x+\theta_2tu))u
      =(\nabla f(x+tu)-\nabla f(x))^Ttu \ge \alpha\|tu\|^2.
    \]
    Namely,
    \[
      \frac{1}{2}u^T(\nabla^2f(x+\theta_1tu)+\nabla^2f(x+\theta_2tu))u\ge 
      \alpha\|u\|^2.
    \]
    Let $t\to 0$ and we obtain
    \[
      u^T\nabla^2f(x)u\ge \alpha\|u\|^2.
    \]
    Hence, all eigenvalues of $\nabla^2f(x)$ are no less than $\alpha$ and, in
    consequence, $\nabla^2f(x)-\alpha I$ is positive semidefinite.
  \end{proof}
  
  \paragraph{11. Arithmetic-Geometric Mean Inequality}
  \begin{proof}
    Since the function $x\mapsto -\log x$ is strictly convex on $(0,\infty)$.
    \begin{align*}
      -\log(\alpha_1x_1+\cdots+\alpha_nx_n)
      &\le-\alpha_1\log x_1-\cdots-\alpha_n\log x_n\\
      &=-\log(x_1^{\alpha_1}\cdots x_n^{\alpha_n}),
    \end{align*}
    where the equality is obtained when $x_1=\cdots=x_n$. Thus, 
    $x_1^{\alpha_1}\cdots x_n^{\alpha_n}\le\alpha_1x_1+\cdots+\alpha_nx_n$ with
    equality iff $x_1=\cdots=x_n$.
  \end{proof}
  
  \paragraph{12.}
  \begin{proof}
    If $x=0$ or $y=0$, then the inequality is trivial. If both $x$ and $y$ are
    nonzero, then, by Prob. 1.11, $x^{1/p}y^{1/q}\le x/p+y/q$. Replace $x$ and
    $y$ with $x^p$ and $y^q$ respectively and we get $xy\le x^p/p+y^q/q$.\par
    If all $y_i$ are zero or all $x_i$ are zero, then the inequality is 
    trivial. If there exists some nonzero $y_i$ and some nonzero $x_i$, then, 
    by the homogeneity, we may assume without loss of generality that
    \[
      \sum_{i=1}^n|x_i|^p=\sum_{i=1}^n|y_i|^q=1.
    \]
    Then, by Young's inequality,
    \[
      \sum_{i=1}^n|x_iy_i|\le
      \frac{1}{p}\sum_{i=1}^n|x_i|^p+\frac{1}{q}\sum_{i=1}^n|y_i|^q
      =\frac{1}{p}+\frac{1}{q}=1.
    \]
    Namely, Holder's inequality holds.
  \end{proof}
  
  \paragraph{13.}
  \begin{proof}
    For $x\notin\dom f$, $f(x)=\inf\varnothing=\infty$. For every $x_1,x_2\in
    \dom(f)$, since $C$ is convex, $x_\theta=(1-\theta)x_1+\theta x_2\in\dom
    (f)$. By definition, for every $\vep>0$, there exists some $(x_1,w_1),(x_2,
    w_2)\in C$ such that $w_i<f(x_i)+\vep$. Hence, 
    \[
      (1-\theta)w_1+\theta w_2<(1-\theta)f(x_1)+\theta f(x_2)+\vep.
    \]
    Since $C$ is convex, $(1-\theta)(x_1,w_1)+\theta(x_2,w_2)\in C$ and 
    therefore
    \[
      f(x_\theta)\le (1-\theta)w_1+\theta w_2.
    \]
    These two inequalities, together with the fact that the choice of $\vep$ is
    arbitrary, imply that $f(x_\theta)\le (1-\theta)f(x_1)+\theta f(x_2)$. 
    Thus, $f$ is convex.
  \end{proof}
% end


\subsection{Convex and Affine Hulls}
  \paragraph{14.}
  \begin{proof}
    Given $\varnothing\ne X\subset\mathbb{R}^n$, let $C$ be the collection of
    all convex combination of elements of $X$. Clear that $X\subset C$. 
    Meanwhile, for every $x,y\in C$, they are the convex combination of points 
    in $X$ and therefore so is $(1-\theta)x+\theta y$ for every $\theta\in(0,
    1)$. Hence, $C$ is a convex set containing $X$. Thus, $\conv(X)\subset C$. 
    For every $x\in C$, $x$ is a convex combination of points in $X$ and 
    therefore is contained in any convex set containing $X$; See Fig. 1.3.1. 
    Hence, $x\in\conv(C)$. Thus, $C=\conv(C)$.
  \end{proof}
  
  \paragraph{15.}
  \begin{proof}
    Let $D=\bigcup_{x\in C}\{\gamma x:\,\gamma\ge 0\}$. It follows immediately
    from the definition that $D\subset\cone(C)$. For every $x\in\cone(C)$. If
    $x=0$, then clear that $x\in D$. If $x\ne 0$, then it can be written as
    $x=\alpha_1x_1+\cdots+\alpha_mx_m$ where $m>0$, $\alpha_i>0$ and $x_i\in 
    C$. Hence
    \[
      x=\frac{1}{\alpha}\sum\frac{\alpha_i}{\alpha}x_i
      \quad\text{where }\alpha=\sum\alpha_i.
    \]
    Since $C$ is convex, $\sum\alpha_ix_i/\alpha\in C$ and therefore $x\in D$.
    Thus, $D=\cone(C)$.
  \end{proof}
  
  \paragraph{16.}
  \begin{proof}
    $\,$\par
    (a) First we show that $C$ is closed. Suppose that $\{x_k\}\subset C$ 
    converges to some $x\in\mathbb{R}^n$. Then for every $i\in I$ and $k=1,2,
    \dots$, $a_i^Tx_k\le 0$. Let $k\to\infty$, by the continuity of the inner
    product, $a_i^Tx\le 0$. Hence, $C$ is closed.\par
    For the convexity, let $x,y\in C$ and $\theta\in(0,1)$. Then for every 
    $i\in I$,
    \[
      a_i^T((1-\theta)x+\theta y)=(1-\theta)a_i^Tx+\theta a_i^Ty\le 0.
    \]
    Namely, $(1-\theta)x+\theta y\in C$. Thus, $C$ is convex.\par
    Finally, since for all $\lambda>0$, $a_i^T(\lambda x)\le 0$ as long as
    $a_i^Tx\le 0$. Hence, $C$ is cone. Thus, we conclude that $C$ is a closed
    convex cone.\par
    (b) Let $C$ be a cone. Suppose that $C$ is convex, then for every $x,y\in 
    C$, $(x+y)/2\in C$. Hence, $x+y=2((x+y)/2)\in C$ as $C$ is a cone. Namely,
    $C+C\subset C$. For the converse, suppose that $C+C\subset C$. For every
    $x,y\in C$ and $\theta\in(0,1)$, since $C$ is a cone, $(1-\theta)x,\theta 
    y\in C$ and therefore $(1-\theta)x+\theta y\in C+C\subset C$. Hence, $C$
    is convex.\par
    (c) For every $x\in C_1$ and $y\in C_2$,
    \[
      x+y=\frac{1}{2}(2x)+\frac{1}{2}(2y)=\conv\{2x,2y\}\subset
      \conv(C_1\cup C_2).
    \]
    Hence, $C_1+C_2\subset\conv(C_1\cup C_2)$. For the converse, we show that
    $C_1+C_2$ is a convex set containing $C_1\cup C_2$. Since $0\in C_1$, $C_2
    \subset 0+C_2\subset C_1+C_2$. Similarly, $C_1\subset C_1+C_2$. Meanwhile,
    by Prop. 1.2.1(b), $C_1+C_2$ is convex. Hence, $\conv(C_1\cup C_2)\subset
    C_1+C_2$. Thus, $\conv(C_1\cup C_2)=C_1+C_2$.\par
    Since $C_1$ and $C_2$ are cones, for $\alpha\in(0,1)$, $C_1=\alpha C_1$ and 
    $C_2=(1-\alpha)C_2$ and therefore $C_1\cap C_2=\alpha C_1\cap(1-\alpha)
    C_2$. For $\alpha\in\{0,1\}$, $\alpha C_1\cap(1-\alpha)C_2=\{0\}\in C_1\cap 
    C_2$. Thus, $C_1\cap C_2=\bigcup_{\alpha\in[0,1]}(\alpha C_1\cap(1-\alpha)
    C_2)$.
  \end{proof}
  
  \paragraph{18. Convex Hulls, Affine Hulls, and Generated Cones}
  \begin{proof}
    $\,$\par
    (a) We may assume without loss of generality that $0\in X$, so that the
    affine hulls are subspaces of $\mathbb{R}^n$. Since $X$ is contained by 
    $\conv(X)$ and $\cl(X)$, $\aff(X)$ is  contained by $\aff(\conv(X))$ and 
    $\aff(\cl(X))$. For the converse, note that a convex combination of points
    in $X$ is also a linear combination, hence $\conv(X)\subset\aff(X)$ and 
    therefore $\aff(\conv(X))\subset\aff(X)$. Meanwhile, since finite 
    dimensional vector spaces are all closed, $\cl(X)\subset\aff(X)$ and 
    therefore $\aff(\cl(X))\subset\aff(X)$. Thus, $\aff(X)=\aff(\conv(X))=\aff
    (\cl(X))$.\par
    (b) Clear that $\cone(X)\subset\cone(\conv(X))$. For the converse, suppose
    $x\in\cone(\conv(X))$. If $x=0$, then $x\in\cone(X)$ in a trivial way. If 
    $x\ne 0$, then $x=\alpha_1x_1+\cdots+\alpha_px_p$ where $x_i\in\conv(X)$, 
    $p>0$ and $\alpha_i>0$. Meanwhile, for each $i$, suppose that $x_i=
    \beta_{i,1}x_{i,1}+\cdots+\beta_{i,q}x_{i,q}$ where $q>0$, $\beta_{i,j}>0$ 
    and $\sum_j\beta_{i,j}=1$. Hence,
    \[
      x=\sum_i\alpha_i\sum_j\beta_{i,j}x_{i,j}=
      \sum_{i,j}\alpha_i\beta_{i,j}x_{i,j}.
    \]
    Namely, $x$ is a positive combination of points in $X$ and therefore $x\in
    \cone(X)$. Hence, $\cone(\conv(X))\subset\cone(X)$. Thus, $\cone(\conv(X))
    =\cone(X)$.\par
    (c) Since $\conv(X)\subset\cone(X)$, $\aff(\conv(X))\subset\aff(\cone(X))$.
    Let $X=[-1,1]\times\{1\}\subset\mathbb{R}^2$. Then clear that $\aff(
    \conv(X))$ is the line crossing $(0,1)$ and parallel to the $x$-axis while
    $\aff(\cone(X))=\mathbb{R}^2$.\par
    (d) Since $0\in\conv(X)\subset\cone(X)$, both $\aff(\conv(X))$ and $\aff
    (\cone(X))$ are subspaces of $\mathbb{R}^n$. By part (c), we already have
    $\aff(\conv(X))\subset\aff(\cone(X))$. Hence, we only need to show that 
    $\dim\aff(\conv(X))\ge\dim\aff(\cone(X))$ to complete the proof. Suppose 
    that $\dim\aff(\cone(X))=m$. By Prop. 1.3.1, there exists $b_1,\dots,b_m
    \in X$ such that linearly independent and span $\aff(\cone(X))$. Note that
    $\{b_1,\dots,b_m\}$ is also a set of linearly independent set in $\aff(
    \conv(X))$. Hence, $\dim\aff(\conv(X))\ge m$. Thus, $\aff(\conv(X))=\aff
    (\cone(X))$.
  \end{proof}
  
  \paragraph{19.}
  \begin{proof}
    We denote these two representation by $f$ and $g$ respectively. For every
    $(x,w)\in\conv(\bigcup_{i\in I}\epi(f_i))$, there exists some positive 
    $\alpha_1,\dots,\alpha_m$ with $\sum\alpha_j=1$ and $(x_1,w_1),\allowbreak
    \dots,\allowbreak(x_m,w_m)\in\bigcup\epi(f_i)$ such that $(x,w)=\sum_j
    \alpha_j(x_j,w_j)$. Namely, for fix $x$,
    \begin{align*}
      f(x)=\inf\left\{\sum_j\alpha_jw_j:\,x=\sum_j\alpha_jx_j,\,
      (x_j,w_j)\in\bigcup_{i}\epi(f_i),\,
      \alpha_j\ge 0,\sum_j\alpha_j=1,m>0\right\}.
    \end{align*}
    By the definition of $\epi$, $(x_j,w_j)\in\bigcup_i\epi(f_i)$ implies 
    $f_{i_j}(x_j)\le w_j$ for some $i_j$. Hence, $f(x)\ge g(x)$. Meanwhile,
    since the union of graphs of $f_i$ is contained in $\bigcup\epi(f_i)$, 
    $f(x)\le g(x)$. Thus, $f(x)=g(x)$.
  \end{proof}
  
  \paragraph{20. Convexification of Nonconvex Functions}
  \begin{proof}
    $\,$\par
    (a) The convexity follows from Prob. 13 immediately. For each $x$, let 
    $f_x$ takes value $f(x)$ and $\infty$ for other points. Then $\{f_x\}$ is 
    a collection of convex functions. Then, by Prob. 19, $F$ has the 
    representation given.\par
    (b) Put $M=\inf_{x\in\conv(X)}F(x)$. By definition, for all $y\in X\subset
    \conv(X)$, $M\le F(y)$ and $F(y)\le f(y)$. Hence, $M\le\inf_{y\in X}f(y)$. 
    For the converse, again by definition, for every $\vep>0$, there exists
    some $x\in\conv(X)$ such that $M+\vep\ge F(x)$. By part (a), this implies
    there exists nonnegative $\alpha_1,\dots,\alpha_m$ with $\sum\alpha_i=1$
    and $x_1,\dots,x_m\in X$ such that $\sum\alpha_ix_i=x$ and $M+\vep\ge 
    \sum\alpha_if(x_i)$. Since $\sum\alpha_if(x_i)$ is a weighted average of 
    values of $f$, it is no less than $\inf_{x\in X}f(x)$. Since the choice of
    $\vep>0$ is arbitrary, we conclude that $M\ge\inf_{x\in X}f(x)$. Thus,
    $\inf_{x\in\conv(X)}F(x)=\inf_{x\in X}f(x)$.\par
    (c) It follows immediately from part (b).
  \end{proof}
  
  \paragraph{21. Minimization of Linear Functions}
  \begin{proof}
    Note that the convexification of $f:X\to\mathbb{R}$ is just $c^Tx$ with
    domain $\conv(X)$. Hence, the equation follows from Prob. 20. Suppose that
    the infimum of the left-hand side is attained, that is, there is some
    $x^*\in\conv(X)$ such that $c^Tx^*=\inf_{x\in\conv(X)}c^Tx$. Then by the
    definition of the convex hull, $x^*$ is the convex combination of some 
    points $x_1,\dots,x_m$ of $X$ and, as $c^Tx$ is linear, $c^Tx^*$ is the
    weighted average of $c^Tx_1,\dots,c^Tx_m$. As a consequence, $c^Tx^*\ge
    \min\{c^Tx_1,\dots,c^Tx_m\}$. Thus, the infimum in the right-hand side can 
    also be attained. For the converse, it is obvious.
  \end{proof}
  
  \paragraph{22. Extension of Caratheodory's Theorem}
  \begin{proof}
    TODO % TODO: 1.22
  \end{proof}
  
  \paragraph{23.}
  \begin{proof}
    Since $X$ is bounded, $\cl(X)$ is also bounded and therefore compact. 
    Hence, by Prop. 1.3.2, $\conv(\cl(X))$ is compact. In consequence, $\cl(
    \conv(\cl(X)))=\conv(\cl(X))$. Thus, $\cl(\conv(X))\subset\cl(\conv(\cl(X))
    )=\conv(\cl(X))$. For the converse, it follows from the fact that $\conv
    (\cl(\conv(X)))=\cl(\conv(X))$ and $\conv(\cl(X))\subset\conv(\cl(
    \conv(X)))$. Thus, $\cl(\conv(X))=\conv(\cl(X))$. \par
    If $X$ is compact, then it is bounded and closed. Hence, $\conv(X)=\conv
    (\cl(X))=\cl(\conv(X))$. Namely, $\conv(X)$ is also closed. Meanwhile, 
    $\conv(X)$ is bounded as $X$ is. Thus, $\conv(X)$ is compact.
  \end{proof}
  
  \paragraph{24. Radon's Theorem}
  \begin{proof}
    TODO % TODO: 1.24
  \end{proof}
  
  \paragraph{25. Helly's Theorem [Hel21]}
  \begin{proof}
    We use induction on the size of the collection. If the size is no more
    than $n+1$, then the statement clearly holds. Assume that, for all 
    collection of no more than $M$ sets, the statement holds. We show that the 
    statement holds for every collection of $M+1$ sets.\par
    Let $C_1,\dots,C_{m+1}$ be a collection of $M+1$ convex sets. For each $j=
    1,\dots,M+1$, put $B_j=\bigcap_{i\ne j}C_i$. By the induction hypothesis,
    all $B_j$ are nonempty. Choose $x_j\in B_j$ ($j=1,\dots,M+1$). Note that 
    $M+1\ge n+2$. Hence, by Radon's Theorem, we can partition $\{1,\dots,M+1\}$
    into to two sets $P$ and $Q$ such that
    \[
      D=\conv(\{x_p:\, p\in P\})\cap\conv(\{x_q:\, q\in Q\})\ne\varnothing.
    \]
    Let $x\in D$ and we show that $x\in\bigcap C_j$ to complete the proof. By
    the construction of $B_j$, we know that for each $p\in P$, $x_p\in C_q$ for
    every $q\in Q$. Since all $C_q$ are convex, $x$, a convex combination of 
    $x_p$, belongs to all $C_q$. Similarly, we can show that $x$ belongs to all 
    $C_p$. Thus, $x\in\bigcap C_j$. Namely, the intersection of $C_1,\dots,
    C_{M+1}$ is nonempty.
  \end{proof}
  
  \paragraph{26.}
  \begin{proof}
    First, clear that for any $I$, $\inf_x\max_i f_i(x)\le f^*$. For the 
    converse, we assume, to obtain a contradiction, that for all index set $I$
    with no more than $n+1$ indices, $\inf_x\max_i f_i(x)<f^*$. Then, putting
    $X_i=\{x:\, f_i(x)<f^*\}$, $i=1,\dots,M$, this implies that every 
    subcollection of $X_1,\dots,X_M$, provided it contains no more than $n+1$
    sets, has nonempty intersection. Meanwhile, $X_i$ are convex sets as $f_i$
    are convex functions. Hence, by Helly's theorem, $\bigcap_{i=1}^MX_i$ is
    nonempty, which contradicts the infimum assumption of $f^*$. Thus, there 
    exists some $I$ such that $\inf_x\max_i f_i(x)\ge f^*$ and therefore the
    two values coincide.
  \end{proof}
% end
\subsection{Relative Interior, Closure, and Continuity}
  \paragraph{27.}
  \begin{proof}
    Fix $x\in\ri(C)$ and, for every $\bar{x}\in\cl(C)$, put $x_\theta=(1-
    \theta)x+\theta\bar{x}$. By the line segment principle, for every $\theta
    \in[0,1)$, $x_\theta\in\ri(C)$. If $f(\bar{x})=\infty$, then $f(\bar{x})
    \ge\gamma$ vacuously. If $f(\bar{x})<\infty$, then 
    \[
      (1-\theta)f(x)+\theta f(\bar{x})\ge f(x_\theta)\ge \gamma.
    \]
    Let $\theta\to 1$ and we get $f(\bar{x})\ge\gamma$.
  \end{proof}
  
  \paragraph{28.}
  \begin{proof}
    By Prop. 1.4.5(b), we may assume without loss of generality that $0\in C$
    and therefore $\aff(C)=S$. First, suppose that $x\in\ri(C)\subset C$. Then
    there exists some $\delta>0$ such that $B\cap S\subset C$ where $B=\{y:\,
    \|x-y\|<\delta\}$. For all $y\in B$, suppose $y=y_1+y_2$ where $y_1\in S$
    and $y_2\in S^\perp$. Then $y-x=(y-y_1)+(y_1-x)$. Since $y_1-x\in S$, $y
    -y_1\in S$. Therefore, 
    \[
      \|y-x\|^2=\|y-y_1\|^2+\|y_1-x\|
      \quad\Rightarrow\quad
      \|y_1-x\|\le \|y-x\|\le \delta.
    \]
    Hence, $y_1\in B\cap S\subset C$. As a consequence, $y=y_1+y_2\in C+
    S^\perp$. Thus, $B\subset C+S^\perp$, implying that $\ri(C)\subset\inte(C+
    S^\perp)\cap C$.\par
    For the reverse inclusion, suppose $x\in\inte(C+S^\perp)\cap C$. Then, 
    there exists some $\delta>0$ such that $B\subset C+S^\perp$. Hence,
    \[
      B\cap S\subset(C+S^\perp)\cap S
      =\bigcup_{u\in S^\perp}\left\{(u+C)\cap S\right\}
      =C.
    \]
    Thus, $x\in\ri(C)$ and therefore $\inte(C+S^\perp)\cap C\subset\ri(C)$.
  \end{proof}
  
  \paragraph{29.}
  \begin{proof}
    $\,$\par
    (a) Let $C\subset\mathbb{R}^n$ be a convex set of dimension $m$ and $S
    \subset C$ the simplex whose dimension attains the maximum $m\hp$. Let 
    $x_0,\dots,x_{m\hp}$ be the vertices of $S$.\par
    First we show that $m\ge m\hp$. By definition, $x_1-x_0,\dots,x_{m\hp}-x_0$
    are linearly independent. Hence, $\dim\aff(S)\ge m\hp$. Meanwhile, as $C
    \supset S$, $\aff(C)\supset\aff(C)$. Thus, $m=\dim\aff(C)\ge\dim\aff(S)\ge 
    m\hp$.\par
    For the converse, we first show that $\aff(S)\supset\aff(C)$. For every
    $x\in C$, $x$ is an affine combination of $x_0,\dots,x_{m\hp}$, otherwise,
    $x_0,\dots,x_{m\hp},x$ are the vertices of a $(m\hp+1)$-dimensional 
    simplex contained by $C$, which would contradict the maximum property of 
    $S$. Thus, $\aff(S)\supset\aff(C)$, implying that $\dim\aff(S)\ge m$. By 
    Prob. 18(a), $\aff(S)=\aff\{x_0,\dots,x_{m\hp}\}$. Hence, $m\hp=\dim\aff
    (S)$. Thus, $m\hp\ge m$.\par
    (b) Let $C$ be a nonempty convex set. If $\dim\aff(C)=0$, then the result
    hold vacuously. Suppose $\dim\aff(C)=m>0$. Then, by part (a), there exists
    a $m$-dimensional simplex $S\subset C$. By the previous discussion, we know
    that $\aff(C)=\aff(S)$. Hence, it suffices to show that $S$ has a nonempty
    interior. Let $x_0,\dots,x_m$ be the vertices of $S$. Put
    \[
      f(x)=x_0+\begin{bmatrix}
        x_1-x_0 & \cdots x_m-x_0
      \end{bmatrix}u,
      \quad u\in\mathbb{R}^m.
    \]
    $f$ is an affine function from $\mathbb{R}^m$ to $\aff(S)$ and $S=f
    (\tilde{S})$ where $\tilde{S}=\{u\in\mathbb{R}^m:\, 0\le u_i\le 1,i=1,
    \dots,m\}$. Clear that $\tilde{S}$ has nonempty interior. Hence, $S$ also
    nonempty interior relative to $\aff(C)$ as affine functions preserve the
    norm.
  \end{proof}
  
  \paragraph{31.}
  \begin{proof}
    $\,$\par
    (a) Suppose that $x\in\ri(C)$ and assume, to obtain a contradiction, that 
    there exists some $\bar{x}\in\aff(C)$ such that for all $\gamma>1$, 
    \[
      x_\gamma=x+(\gamma-1)(x-\bar{x})\notin C.
    \]
    Since $x_\gamma-x=(\gamma-1)(x-\bar{x})$, $x_\gamma\in\aff(C)$. Hence, for
    all $\delta>0$, let $B_\delta=\{y:\, \|x-y\|<\delta\}$. Then, $x_{\delta/
    2}\in B\cap\aff(C)$ but $x\notin C$, contradicting the assumption that $x$
    is a relative interior point. Hence, for all $\bar{x}\in\aff(C)$, there is
    some $\gamma>1$ such that $x+(\gamma-1)(x-\bar{x})\in C$. The converse
    follows immediately from Prop. 1.4.1(c).\par
    (b) Clear that $\cone(C)\subset\aff(C)$. For the reverse inclusion, suppose
    $\bar{x}\in\aff(C)$. Since $0\in C$, $-\bar{x}\in\aff(C)$. Then, by part 
    (a), there exists some $\gamma>1$ such that $(\gamma-1)\bar{x}\in C$. Thus,
    \[
      x=\frac{1}{\gamma-1}((\gamma-1)\bar{x})\in\cone(C).
    \]
    Hence, $\cone(C)\supset\aff(C)$. \par
    (c) By Prob. 1.18, $\cone(X)=\cone(\conv(X))$ and $\aff(X)=\aff(\conv(X))$.
    Since $0\in\ri(\conv(X))$, $\cone(\conv(X))=\aff(\conv(X))$, by part (b).
    Thus, $\cone(X)=\aff(X)$.
  \end{proof}
  
  \paragraph{32.}
  \begin{proof}
    $\,$\par
    (a) If $0\in\ri(C)$, then, by Prob. 1.31, $\cone(X)=\aff(X)$. Thus, $\cone
    (X)$ is closed. Now, suppose $0\notin\cl(C)$. Let $x\in\cl(\cone(X))$. Note
    that $x\ne 0$. Let $(x^{(k)})\subset\cone(X)$ converge to $x$. By Prob. 15,
    we may represent each $x^{(k)}$ by $x^{(k)}=\gamma_kx_k$ where $\gamma_k
    >0$ and $x_k\in C$. We show that the sequence $(x_k,\gamma_k)$ is bounded.
    The boundedness of $(x_k)\subset C$ comes from the compactness of $C$. 
    Assume, to obtain a contradiction, that $\gamma_k$ is unbounded. Then, 
    there is a subsequence $(\gamma_{k_j})$ which converges to $\infty$. Since
    $0\notin\cl(C)$, there exists a positive $\delta$ such that $\|x_k\|>
    \delta$ for each $k$. Hence,
    \[
      \lim_{j\to\infty}\|x^{(k_j)}\|=
      \lim_{j\to\infty}|\gamma_{k_j}|\|x_{k_j}\|=\infty,
    \]
    which contradict the hypothesis $x^{(k)}\to x$. Hence, $(\gamma_k)$ is 
    bounded and, therefore, so is $(x_k,\gamma_k)$. Then, it has a a 
    subsequence converging to some point, say, $(\tilde{x},\gamma)$ and 
    therefore $x=\gamma\tilde{x}$. Since $\gamma_k\ge 0$, $\gamma\ge 0$. And
    by the closedness of $C$, $\tilde{x}\in C$. Thus, $x\in\cone(C)$. Namely,
    $\cone(C)$ is closed.\par
    (b) Let $C_1=\{(x,y)\in\mathbb{R}^2:\, x>0,y\ge 1/x\}$, which is a closed
    convex set in $\mathbb{R}^2$. However, $\cone(X)=\{(x,y):\, x>0,y>0\}$ is
    not closed.\par
    Let $C_2=\{(x,y)\in\mathbb{R}^2:\, (x-1)^2+y^2\le 1\}$, which is convex and
    compact but contains the origin. $\cone(C_2)$, the positive half plane, is
    not closed.\par
    (c) By Prop. 1.3.2, $\conv(C)$ is compact as $C$ is. Since the origin is 
    not in the relative boundary of $\conv(C)$, by part (a), $\cone(\conv(C))$
    is closed. Meanwhile, Prob. 1.18 implies $\cone(C)=\cone(\conv(C))$. Thus,
    $\cone(C)$ is closed.
  \end{proof}
  
  \paragraph{34.}
  \begin{proof}
    We shall construct special linear transformation and apply Prop. 1.4.4 to
    show the result. Consider the space $V=\mathbb{R}^{n+m}$. Then $A$ is
    characterized by the set $G=\{(x,Ax):\,x\in\mathbb{R}^n\}\subset V$, which,
    by the linearity of $A$, is a subspace of $V$. Put $D=\mathbb{R}^n\times C$ 
    and let $P:V\to\mathbb{R}^n$ be the projection mapping. Then, $A\inv\cdot 
    C=T\cdot(G\cap D)$. Thus, by Prop. 1.4.4(a),
    \begin{equation}
      \label{eq:1.34-1}
      \ri(A\inv\cdot C)=\ri(T\cdot(G\cap D))=T\cdot\ri(G\cap D).
    \end{equation}
    Since $G$ is a subspace, $\ri(G)=G$. Meanwhile, $\ri(G)=\mathbb{R}^n\times
    \ri(C)$. Since $A\inv\cdot\ri(C)$ is nonempty, $\ri(G)$ and $\ri(D)$ has
    nonempty intersection. Therefore, by Prop. 1.4.5,
    \[
      \ri(G\cap D)=\ri(G)\cap\ri(D)=G\cap(\mathbb{R}^n\times\ri(C))=
      (A\inv\cdot\ri(C))\times\ri(C).
    \]
    Hence,
    \[
      T\cdot\ri(G\cap D)=T\cdot((A\inv\cdot\ri(C))\times\ri(C))=
      A\inv\cdot\ri(C).
    \]
    This, together with \eqref{eq:1.34-1}, imply that $\ri(A\inv\cdot C)=A\inv
    \cdot\ri(C)$.\par
    From a similar argument we can obtain that 
    \begin{align*}
      \cl(A\inv\cdot C)\supset T\cdot\cl(G\cap D)=A\inv\cdot\cl(C).
    \end{align*}
    For the reverse direction, suppose $x\in\cl(A\inv\cdot C)$ and let $(x_k)
    \subset A\inv\cdot C$ be a sequence converging to $x$. Suppose $y_k=Ax_k$,
    which is contained in $C$. By the continuity of $A$, $y_k\to Ax=y$ as $x_k
    \to x$. Note that $y\in\cl(C)$. Hence, $x\in A\inv\cdot\cl(C)$. Thus, $\cl
    (A\inv\cdot C)\subset A\inv\cdot\cl(C)$, completing the proof.
  \end{proof}
  
  \paragraph{35. Closure of a Convex Function}
  \begin{proof}
    $\,$\par
    (a) By Prop. 1.2.2, $\cl f$ is lower semicontinuous. Let $g$ be a lower
    semicontinuous function majorized by $f$. For $x\notin\dom f$, $(\cl f)(x)
    =\infty\ge g(x)$. Suppose that $x\in\dom f$. Note that $(\cl f)(x)=\inf\{
    w:\,(x,w)\in\cl(\epi f)\}$. Hence, for every $\vep>0$, there exists a
    $(x,w)\in\cl(\epi f)$ such that $(\cl f)(x)+\vep>w$. Suppose that $(x_k,
    w_k)\subset\epi f$ converges to $(x,w)$. Then
    \[
      g(x)\le\liminf_{k\to\infty}f(x_k)\le\lim_{k\to\infty}w_k
      =w < (\cl f)(x)+\vep.
    \]
    Since the choice of $\vep>0$ is arbitrary, we conclude that $g(x)\le(\cl f)
    (x)$.\par
    (b) By definition, $\cl f$ is closed and, since the closure of the convex
    set $\epi f$ is convex, $\cl f$ is convex. Since $f$ is proper, there 
    exists some point at which $f$ is not $\infty$. In consequence, $\epi f$ is
    nonempty and, therefore, so is $\cl(\epi f)$. Hence, there exists some 
    point at which $\cl f$ is not $\infty$. To show that $\cl f>-\infty$, we 
    argue by contradiction. Assume that $\cl(\epi f)$ contains a vertical line,
    say, $L=\{(\bar{x},w):\, w\in\mathbb{R}\}$. Let $((\bar{x},w_k))$ be such
    that $w_k\to-\infty$. Since $f$, a convex function, is continuous over
    $\ri(\dom f)$, $\bar{x}$ can not be in $\ri(\dom f)$, otherwise $f$ will 
    take value $-\infty$ at $\bar{x}$. Let $\bar{x}\ne x\in\dom f$ and fix $(x,
    w)\in\ri(\epi f)$. Then, for every $\theta\in(0,1)$, by the line segment
    principle, $(x_{\theta},w_{k,\theta})=(1-\theta)(x,w)+\theta(\bar{x},w_k)
    \in\ri(\epi f)$. For each $\theta$, $\theta w_k\to-\infty$ as $k\to\infty$,
    which implies that $\cl(\epi f)$ contains the vertical line $\{(x_\theta,w)
    :\,w\in\mathbb{R}\}$. This is impossible by our preceding discussion as
    $x_\theta\in\ri(\dom f)$. Thus, $\cl f>-\infty$. We conclude that $\cl f$ 
    is a closed proper convex function. And it follows from the continuity of
    $f$ over $\ri(\dom f)$ that $\cl f$ and $f$ coincide over $\ri(\dom f)$.
    \par
    (c) By part (a), $\cl f$ is lower semicontinuous and majorized by $f$. 
    Hence,
    \[
      (\cl f)(y)\le\lim(\cl f)(y+\alpha(x-y))\le\lim f(y+\alpha(x-y)).
    \]
    For the converse, let $(x,w)\in\ri(\epi f)$. Then for each $\alpha\in
    (0,1)$, by the line segment principle,
    \[
      (y, (\cl f)(y))+\alpha(x-y, w-(\cl f)(y))=
      (y+\alpha(x-y), (\cl f)(y)+\alpha(w-(\cl f)(y)))\in
      \ri(\epi f).
    \]
    Namely,
    \[
      f(y+\alpha(x-y))\le
      (\cl f)(y)+\alpha(w-(\cl f)(y)).
    \]
    Let $\alpha\downarrow 0$ and we get $\lim f(y+\alpha(x-y))\le(\cl f)(y)$.
    Thus, $\lim f(y+\alpha(x-y))=(\cl f)(y)$.
  \end{proof}
% end
\subsection{Recession Cones}
  \paragraph{36.}
    "$C\cap M$ is bounded" should be "$\cl(C)\cap M$ is bounded".
  \begin{proof}
    Since $\cl(C)$ is convex as $C$ is and $\cl(C)\cap\bar{M}$ is bounded iff
    $C\cap\bar{M}$ is, we may assume without loss of generality that $C$ is
    closed. By the recession cone theorem, the boundedness of the nonempty 
    closed convex set $C\cap M$ implies that $R_{C\cap M}=\{0\}$. Meanwhile, 
    since both $C$ and $M$ are closed convex sets and their intersection is
    nonempty, again by the recession cone theorem, $R_{C\cap M}=R_C\cap R_M$.
    Since, $\bar{M}$ is an affine set parallel to $M$, $R_{\bar{M}}=R_M$. 
    Hence, $R_C\cap R_{\bar{M}}=R_{C\cap M}=\{0\}$. If $C\cap\bar{M}$ is 
    empty, then it is bounded vacuously. If it is nonempty, then $R_{C\cap
    \bar{M}}=R_C\cap R_{\bar{M}}=\{0\}$, implying that $C\cap\bar{M}$ is
    bounded.
  \end{proof}
  
  \paragraph{39. Recession Cones of Relative Interiors}
  \begin{proof}
    $\,$\par
    (a) Since $\ri(C)=\ri(\cl(C))$ and $\cl(C)$ is closed, it follows from 
    Prop. 1.5.1(d) that $R_{\ri(C)}=R_{\cl(C)}$.\par
    (b) By part (a), $y\in R_{\ri(C)}$ iff $y\in R_{\cl(C)}$ iff, by Prop. 
    1.5.1(b), there is an $x\in\ri(C)$ such that $x+\alpha y\in\cl(C)$ for all
    $\alpha\ge 0$. By the line segment principle, for all $\alpha\ge 0$, $x+
    \alpha y\in\ri(C)$ as long as $x+(\alpha+1)y\in\cl(C)$. This, together 
    with $\ri(C)\subset\cl(C)$, imply that $y\in R_{\ri(C)}$ iff there is an
    $x\in\ri(C)$ such that $x+\alpha y\in\ri(C)$ for all $\alpha\ge 0$.\par
    (c) By Prop. 1.5.1, $R_{\cl(C)}\subset R_{\cl(\bar{C})}$ as $\cl(C)
    \subset \cl(\bar{C})$ and both of them are convex and closed. By Prob. 38,
    $R_C\subset R_{\cl(C)}$. By part (a), $R_{\cl(\bar{C})}=R_{\ri(\bar{C})}
    =R_{\bar{C}}$. Thus, $R_C\subset R_{\bar{C}}$.\par
    For an example showing the necessity of $\bar{C}=\ri(\bar{C})$, let $C=
    \{(x_1,x_2):\, x_1\ge 0, 0\le x_2\le 1\}$ and $\bar{C}=\{(x_1,x_2):\,
    x_1\ge 0, 0\le x_2<2\}\cup\{(0,2)\}$. Clear that $C\subset\bar{C}$ but
    $(1,0)$, a direction in $R_C$, does not belongs to $R_{\bar{C}}$.
  \end{proof}
% end
















