\section{Lebesgue Measure}
\subsection{Introduction}
  \paragraph{1.}
  \begin{proof}
    Since $\mathfrak{M}$ is an $\sigma$-algebra, $B\backslash A \in\mathfrak{M}$
    as long as $A,B\in\mathfrak{M}$. Since $B\backslash A$ and $A$ are disjoint,
    $mB=mA+m(B\backslash A)\ge mA$ since $m$ is nonnegative.
  \end{proof}

  \paragraph{2.}
  \begin{proof}
    Let $A_0 = E_0$ and $E_k=A_k\backslash A_{k-1}$ for $k\ge 1$. Clear that 
    $E_i$ and $E_j$ are disjoint for distinct $i$ and $j$, $\bigcup A_n=\bigcup
    E_n$ and $A_i\subset E_i$ for every $i$. Hence,
    \[
      m\left(\bigcup E_n\right) = m\left(\bigcup A_n\right)
      = \sum mA_n \le \sum mE_n,
    \]
    where the last inequality comes from Exercise 1.
  \end{proof}

  \paragraph{3.}
  \begin{proof}
    Suppose that $mA<\infty$. Then $mA=m(A\cup\varnothing)=mA+m\varnothing$, 
    implying that $m\varnothing=0$.
  \end{proof}
% end

\subsection{Outer Measure}
  \paragraph{5.}
  \begin{proof}
    We show that $\{I_n\}$ must cover the entire $[0,1]$ by contradiction. 
    Assume that $x\notin I_k$ for $k=1,2,\dots,n$. Then, as $I_k$ are open and 
    $n$ is finite, there exists some $\vep>0$ such that $(x-\vep,x+\vep)$ and 
    $I_k$ are disjoint for every $k$. Since $\mathbb{Q}$ is dense in 
    $\mathbb{R}$, there exists some rational number in $(x-\vep, x+\vep)$, 
    contradicting with the hypothesis that $\{I_k\}$ covers all rational numbers
    between $0$ and $1$.
  \end{proof}

  \paragraph{6.}
  \begin{proof}
    By the definition of the outer measure, for every $\vep > 0$, there exists 
    some collection $\{I_n\}$ of open intervals that covers $A$ and $\sum l(I_n)
    \le m^*A+\vep$. Let $O=\bigcup I_n$. $O$ is a countable union of open sets 
    and therefore is also open. And by Proposition 2, $m^*O\le \sum l(I_n)$. 
    Thus, $m^*O\le m^*A+\vep$. \par
    Let $\vep_n = 1/n$ and for each $n$, by the previous discussion, we can 
    always get an open set $O_k$ such that $A\subset O_k$ and $m^*O\le m^*A+
    \vep_m$. Let $G$ be the countable intersection of these open sets. Clear 
    that $G$ is a $G_\delta$ set covering $A$ and $m^*A=m^*G$.
  \end{proof}

  \paragraph{7.}
  \begin{proof}
    If $m^*E=\infty$, it is trivial. Suppose that $m^*E\le\infty$. For any $x\in
    \mathbb{R}$, collection $\{I_n\}$ of open intervals covers $E+x$ iff $\{I_n
    -x\}$ covers $E$. Since the length of intervals is translation invariant, 
    this implies $m^*(E+x)=m^*E$.
  \end{proof}

  \paragraph{8.}
  \begin{proof}
    Clear that $m^*A\le m^*(A\cup B)$. Meanwhile, $m^*(A\cup B) = m^*A + m^*B =
    m^*B$. Hence, $m^*(A\cup B)=m^*B$.
  \end{proof}
% end
