\section{The Lebesgue Integral}
\setcounter{subsection}{1}

\subsection{The Lebesgue Integral of a Bounded Function}
  \paragraph{2.}
  \begin{proof}
    $\,$\par
    (a) By Problem 2.51, $h$ is upper semicontinuous as $f$ is bounded and by 
    Problem 2.50, ${x:\, h(x)<\lambda}$ is open and hence measurable for every
    $\lambda\in\mathbb{R}$. Thus, $h$ is measurable.\par
    Let $\varphi(x)\ge f(x)$ be a step function and $x_0$ any point other 
    than the endpoints of the intervals occurring in $\varphi$. Then there 
    exists some $\delta>0$ such that for all $x\in(x_0-\delta,x_0+\delta)$, 
    $\varphi(x_0) = \varphi(x) \ge f(x)$. Hence,
    \[
      h(x_0) = \inf_{\delta<0}\sup_{|x-x_0|<\delta}f(x) \le \varphi(x_0).
    \]
    Namely, $\varphi\ge h$ except at a finite number of points. Hence, $\int_a^b
    \varphi \ge \int_a^b h$ and therefore
    \[
      R\upint_a^b f = \inf_{\varphi\ge f}\int_a^b\varphi(x)\rd x \ge \int_a^b h.
    \]\par
    We can also derive from the previous discussion that there is a sequence of 
    $<\varphi_n>$ of step functions satisfying $\varphi \downarrow h$. By 
    Proposition 6,
    \[
      \int_a^b h = \lim\int_a^b\varphi_n \ge R\upint_a^b f.
    \]
    Hence, $R\upint_a^b f = \int_a^b h$.\par
    (b) First suppose that $f$ is Riemann integrable and let $h$ and $g$ be the 
    upper and lower envelope of $f$ respectively. By part (a), $f$ is Riemann 
    integrable implies $\int_a^b(h-g) = 0$. Together with the fact that $h\ge 
    g$, we conclude that $h=g$ a.e.. Therefore, by Problem 2.50, $f$ is 
    continuous except on a set of measure zero.\par
    Note that the argument remains true if we reverse the order, verifying the 
    converse part. Hence, the proposition holds.
  \end{proof}
% end
