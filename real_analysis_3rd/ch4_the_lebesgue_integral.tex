\section{The Lebesgue Integral}
\setcounter{subsection}{1}

\subsection{The Lebesgue Integral of a Bounded Function}
  \paragraph{2.}
  \begin{proof}
    $\,$\par
    (a) By Problem 2.51, $h$ is upper semicontinuous as $f$ is bounded and by 
    Problem 2.50, ${x:\, h(x)<\lambda}$ is open and hence measurable for every
    $\lambda\in\mathbb{R}$. Thus, $h$ is measurable.\par
    Let $\varphi(x)\ge f(x)$ be a step function and $x_0$ any point other 
    than the endpoints of the intervals occurring in $\varphi$. Then there 
    exists some $\delta>0$ such that for all $x\in(x_0-\delta,x_0+\delta)$, 
    $\varphi(x_0) = \varphi(x) \ge f(x)$. Hence,
    \[
      h(x_0) = \inf_{\delta<0}\sup_{|x-x_0|<\delta}f(x) \le \varphi(x_0).
    \]
    Namely, $\varphi\ge h$ except at a finite number of points. Hence, $\int_a^b
    \varphi \ge \int_a^b h$ and therefore
    \[
      R\upint_a^b f = \inf_{\varphi\ge f}\int_a^b\varphi(x)\rd x \ge \int_a^b h.
    \]\par
    We can also derive from the previous discussion that there is a sequence of 
    $\langle\varphi_n\rangle$ of step functions satisfying $\varphi \downarrow h$. By 
    Proposition 6,
    \[
      \int_a^b h = \lim\int_a^b\varphi_n \ge R\upint_a^b f.
    \]
    Hence, $R\upint_a^b f = \int_a^b h$.\par
    (b) First suppose that $f$ is Riemann integrable and let $h$ and $g$ be the 
    upper and lower envelope of $f$ respectively. By part (a), $f$ is Riemann 
    integrable implies $\int_a^b(h-g) = 0$. Together with the fact that $h\ge 
    g$, we conclude that $h=g$ a.e.. Therefore, by Problem 2.50, $f$ is 
    continuous except on a set of measure zero.\par
    Note that the argument remains true if we reverse the order, verifying the 
    converse part. Hence, the proposition holds.
  \end{proof}
% end

\subsection{The Integral of a Nonnegative Function}
  \paragraph{3.}
  \begin{proof}
    Suppose that $E_n=\{x:\, f(x)>1/n\}$. Then, 
    \[
      0=\int f \ge \int_{E_n} f \ge \frac{mE_n}{n}
    \]
    implies $mE_n=0$. Hence, $m\{x:\, f(x)>0\}=m(\bigcup E_n)\le\sum mE_n=0$. 
    Namely, $f=0$ a.e.
  \end{proof}

  \paragraph{5.}
  \begin{proof}
    For any fixed $x_0\in\mathbb{R}$, let $f_n(x) = f\cdot\chi_{(-\infty,x_0-
    1/n]}$, which is a increasing sequence of nonnegative measurable function
    whose limit is $f\cdot\chi_{-\infty,x_0}$. Then by Theorem 10, 
    \[
      F(x_0)=\int_{-\infty}^{x_0} f = \int f\cdot\chi_{-\infty,x_0}
      = \lim\int f\cdot\chi_{(-\infty,x_0-1/n]} = \lim F(x_0-1/n).
    \]
    Meanwhile, since 
    \[
      |F(x_0)-F(x_0+1/n)| = \left|\int_{x_0}^{x_0+1/n}f(x)\rd x\right|=
      \left|\int_{-1/n}^0 g(x)\rd x \right|,
    \]
    where $g(x)=f(x_0-x)$, arguing on $g$ in a similar manner yields $F(x_0)=
    \lim F(x_0+1/n)$. Thus, $F$ is continuous.
  \end{proof}

  \paragraph{6.}
  \begin{proof}
    By Theorem 9, $\int f \le \lowlim\int f_n$. Meanwhile, $f_n\le f$ implies
    $\int f_n\le \int f$ and therefore $\uplim \int f_n \le \int f$. Hence, 
    $\int f =\lim\int f_n$.
  \end{proof}

  \paragraph{7.}
  \begin{solution}
    $\,$\par
    (a) Let $f_n(x)=n\cdot\chi_{[0,1/n]}$. $f_n$ converges to $f=0$ except on 
    $x=0$. For each $n$, $\int f_n = 1$ but $\int f=0$. Hence, the inequality
    could be strict.\par
    (b) Let $f_n(x)=\chi_{[n,\infty)}$. Then $\langle f_n\rangle$ is a 
    decreasing sequence which converges to $f=0$, the integral of which is $0$.
    However, for every $n$, $\int f_n = \infty$.
  \end{solution}

  \paragraph{8.}
  \begin{proof}
    Let $g_n = \inf\{f_n,f_{n+1},\dots\}$. Clear that 
    \begin{equation}
      \label{eq:4.3.8}
      \int g_n \le \int f_n.
    \end{equation}
    Meanwhile $\langle g_n\rangle$ is a increasing sequence converging to 
    $\lowlim f_n$. Hence, by the Monotone Convergence Theorem and 
    \eqref{eq:4.3.8}
    \[
      \int \lowlim f_n = \int \lim_{n\to\infty}g_n = 
      \lim_{n\to\infty}\int g_n \le \lowlim\int f_n.
    \]
  \end{proof}

  \paragraph{9.}
  \begin{proof}
    By Fatou's Lemma,
    \begin{equation}
      \label{eq:4.3.9}
      \int_E f \le \lowlim\int_E f_n.
    \end{equation}
    Similarly, $\int_{\bar{E}}f \le \lowlim\int_{\bar{E}} f_n$ and therefore
    \[
      \int_E f_n = \int f_n-\int_{\bar{E}} f_n\quad\Rightarrow\quad
      \uplim\int_E f_n \le \int f- \int_{\bar{E}} f = \int_{\bar{E}} f.
    \]
    \eqref{eq:4.3.9} and the inequality above together implies $\int_E f_n\to
    \int f$.
  \end{proof}
% end

\subsection{The General Lebesgue Integral}
  \paragraph{12.}
  \begin{proof}
    Note that $\langle g+f_n\rangle$ is a sequence of nonnegative measurable
    functions. Hence by Problem 8,
    \[
      \int_E\lowlim(g+f_n) \le \lowlim\int_E(g+f_n) \quad\Rightarrow\quad
      \int_E\lowlim f_n \le \lowlim\int_E f_n.
    \]
    The second inequality follows immediately from the definition of lower and
    upper limit. Replacing $g+f_n$ with $g-f_n$ and arguing in a similar manner
    gives the last inequality.
  \end{proof}

  \paragraph{13.}
  \begin{proof}
    $f_n\ge -h$ implies $f_n+h\ge 0$. Hence, $\int(f_n+h)$ always has a meaning.
    And since $g$ is integrable, $\int f_n = \int(f_n+h)-\int h$ also has a
    meaning. Similarly, $\int f$ has a meaning. Meanwhile,
    \[
      \int f = \int(f+h) - \int h \le \lowlim\int(f_n+h) - \int h 
      = \lowlim\int f_n.
    \]
  \end{proof}


  \paragraph{15.}
  \begin{proof}
    $\,$\par
    (a) By Problem 4, for every $\vep>0$, there exists some simple functions 
    $\varphi_1\le f^+$ and $\varphi_2\le f^-$ such that 
    \[
      \int_E f^+ - \int_E\varphi_1 < \vep \quad\text{and}\quad
      \int_E f^- - \int_E\varphi_2 < \vep.
    \]
    Let $\varphi=\varphi_1-\varphi_2$, which is also a simple function. 
    Meanwhile,
    \[
      \int_E|f-\varphi| \le \int_E(f^+-\varphi_1) + \int_E(f^--\varphi_2) < 2\vep.
    \]
  \end{proof}
  
  \paragraph{16.}
  \begin{proof}
    For every integrable $f$, by Problem 15, there exists some step function $\psi
    =\sum_{k=1}^N c_k\chi_{E_k}$ such that $\int|f-\psi|<\vep$. Note that
    \begin{equation}
      \label{eq:4.4.16}
      \lim_{n\to\infty}\int_{-\infty}^\infty \psi(x)\cos nx\rd x =
      \lim_{n\to\infty}\sum_{k=1}^N c_k\int_{E_k}\cos nx\rd x = 0.      
    \end{equation}
    Hence,
    \begin{align*}
      \left|\int_{-\infty}^\infty f(x)\cos nx\rd x \right|
      &= \left|\int_{-\infty}^\infty (f(x)-\psi(x)+\psi(x))\cos nx\rd x\right|\\
      &\le \int_{-\infty}^\infty |f(x)-\psi(x)||\cos nx|\rd x + 
           \left|\int_{-\infty}^\infty \psi(x)\cos nx\rd x\right| \\
      &\le \vep + \left|\int_{-\infty}^\infty \psi(x)\cos nx\rd x\right| \\
      &\to 0 \text{ as }n\to\infty.
    \end{align*}
  \end{proof}

  \paragraph{18.}
  \begin{proof}
    Let $\langle t_n\rangle$ be any sequence with $t_n\ne 0$ and tending to $0$.
    Then $\langle f(x,t_n)\rangle$ is sequence of functions satisfying the 
    hypotheses of Lebesgue Convergence Theorem. Meanwhile, $f(x,t_n)\to f$ as 
    $n\to\infty$. Hence,
    \[
      \lim_{n\to\infty}\int f(x,t_n)\rd x = \int f(x)\rd x.
    \]
    Since the choice of $\langle t_n\rangle$ is arbitrary, by Problem 2.49f, 
    \[
      \lim_{t\to 0}\int f(x,t)\rd x = \int f(x)\rd x.
    \]
    If $f$ is continuous in $t$ for each $x$, then $\lim_{\Delta t\to 0}f(x,t+
    \Delta t)=f(x,t)$ holds for every $t$. Therefore, replacing $t$ with 
    $\Delta t$ in the previous result yields
    \[
      \lim_{\Delta t\to 0}\int f(x,t+\Delta t)\rd x = \int f(x,t)\rd x.
    \]
    Namely, $h(t)$ is continuous.
  \end{proof}

% end
