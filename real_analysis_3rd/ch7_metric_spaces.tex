\section{Metric Spaces}
\setcounter{subsection}{6}
\subsection{Compact Metric Space}
\paragraph{27.}
\begin{proof}
  If $\rho(F, K) > 0$, then clear that $F\cap K = \varnothing$. For the reverse
  direction, consider the function $h(x) = \rho(x, F)= \inf_{y\in F}\rho(x,y)$.
  Clear that for $x\in K$, $h(x) \le \rho(K, F)$.

  First, we show that $h$ is continuous. Let $x$ be fixed. For every
  $x\hp \in X$
  \[
    h(x) \le \rho(x, y) \le \rho(x, x\hp) + \rho(x\hp, y),
    \quad\forall y\in F.
  \]
  Take infimum on the right hand side and we get
  \[
    h(x) \le \rho(x, x\hp) + h(x\hp)
    \quad\Rightarrow\quad
    h(x) - h(x\hp) \le \rho(x, x\hp).
  \]
  Similarly, we have $h(x\hp) - h(x) \le \rho(x, x\hp)$. Thus, $h$ is
  continuous.
  
  Since $K$ is compact and $h$ is continuous, $h$ attains its infimum $c$
  at some point $x_0 \in K$. Assume, to obtain a contradiction, that $c = 0$.
  Then, for every $\vep > 0$, there is a $y \in F$ s.t. $\rho(x_0, y) < \vep$.
  Namely, $x_0$ is a cluster point of $F$. Since $F$ is closed, $x_0 \in F$, 
  which contradicts with $F\cap K=\varnothing$. Thus, $c > 0$ and therefore
  $\rho(F, K) > 0$.
\end{proof}

\paragraph{28.}
\begin{proof}
  $\,$\par
  (a) Let $\vep > 0$ be fixed. Since $f$ is uniformly continuous, there exists
  some $\delta > 0$ s.t. for every $x_1,x_2\in X$ with $\rho(x_1, x_2) < 
  \delta$, $\rho(f(x_1), f(x_2)) < \vep$. Let $B\subset X$ be a ball with
  radius $\delta/2$. Then $f(B) \subset Y$ is contained by some ball of 
  radius $\vep$. Hence, we can cover $Y$ with finitely many balls of radius
  $\vep$ as long as we can cover $X$ with finitely many balls of radius 
  $\delta / 2$ and this can be done since $X$ is totally bounded.
  
  (b) Clear that $X = (0, 1)$ is totally bounded while $Y = (1, \infty)$ is
  not. The function $1/x$ a continuous function that maps $X$ onto $Y$. Hence,
  the result does not hold. 
\end{proof}


\paragraph{29.}
\begin{proof}
  $\,$\par
  (a) First, by the definition of open cover and open sets, $\varphi(x) > 0$.
  Meanwhile, since $X$ is compact, it is bounded and, therefore, 
  $\varphi(x) < \infty$.
  
  (b) Let $x$ be fixed and $r$ be such that there exists some $O\in \mcal{U}$
  with $B_{x, r}\subset O$. Let $s = r - \rho(x, y)$. If $s \le 0$, then
  clear that $s \le \varphi(y)$. If $s > 0$, then we have $B_{y, s}
  \subset B_{x, r} \subset O$. Hence, $s \le \varphi(y)$. Thus, 
  $r - \rho(x, y) \le \varphi(y)$. Take supremum on the left hand side and we
  get $\varphi(x) - \rho(x, y) \le \varphi(y)$.
  
  (c) It follows immediately from (b).
  
  (d) Let $(x_n)$ be a sequence s.t. $\varphi(x_n) \to \vep$. Since $X$ is
  sequentially compact, $(x_n)$ has a convergent subsequence. For the sake of 
  convenience, we assume the subsequence is $(x_n)$ itself. Suppose $x_n
  \to x$. Since $\varphi$ is continuous, $\varphi(x) = \lim\varphi(x_n) = 
  \vep$. Thus, by (a), $\vep > 0$.
  
  (e) Let $\delta$ be any positive number that is less than $\vep$. For every
  $x$, since $\delta < \vep = \inf\varphi$, $\delta < \varphi(x)$ and,
  therefore, $B_{x, \delta} \subset O$ for some $O \in \mcal{U}$.
\end{proof}

\subsection{Baire Category}
\paragraph{31.(a)}
\begin{proof}
  Suppose that $F$ is nowhere dense. Since $F$ is closed, $F^c$ is dense.
  Therefore, for every point $x\in F$, every neighborhood of $x$ contains a
  point of $F^c$. Thus, $F$ contains to open set. For the reverse direction, 
  since $F$ contains no open set, every neighborhood of every $x\in X$ contains
  a point in $F^c$, which implies that $F^c$ is dense.
\end{proof}

\paragraph{34.}
\begin{proof}
  For the first part, it suffices to show that $\partial E$ is nowhere dense
  for any set $E \subset X$. Since $\partial E$ is closed and contains no open
  set, by 31.(a), it is nowhere dense.
  
  For the second part, assume, to obtain a contradiction, that $F$ is not
  nowhere dense. Then by 31.(a), it contains an nonempty open set $O$. By
  32.(a), $O$ is also meager, which contradicts the Baire category theorem. 
\end{proof}

\paragraph{35.}
\begin{proof}
  Let $E$ be a subset set of the complete metric space $X$. If $E$ is residual,
  then $E^c = \bigcup_n K_n$ where $K_n$ are nowhere dense sets. Hence, 
  \[
    E = \bigcap K_n^c \supset \bigcap (\cl K_n)^c.
  \]
  Note that $(\cl K_n)^c$ is a $G_\delta$. Meanwhile, since each $(\cl K_n)^c$
  is a dense open set and $X$ is complete, By the theorem of Baire, 
  $\bigcap (\cl K_n)^c$ is also dense. The proof of the reverse direction is
  similar.
\end{proof}

\paragraph{39.}
\begin{proof}
  Let $E_m$ has the same meaning as in the proof of Theorem 32. Let $O := 
  \bigcup_m E_m^\circ$. By Prop. 31, $O$ is a dense residual open set. For
  every $x\in O$, $x \in E_m^\circ$ for some $m$. Since $E_m^\circ$ is open,
  there is a neighborhood $U$ of $x$ which is contained by $E_m^\circ$. Thus,
  $\mcal{F}$ is uniformly bounded by $m$ in $U$.
\end{proof}


\setcounter{subsection}{9}
\subsection{The Ascoli-Arzel\'{a} Theorem}
\paragraph{47.}
\begin{proof}
  Let $x$ be an arbitrary fixed point in $X$ and $(x_n)\subset X$ a sequence
  that converges to $x$. Clear that $K = \{x\}\cup(x_n)$ is (sequently)
  compact. Hence, $f_n$ converges to $f$ uniformly on $K$. Thus, $f$ is also
  continuous on $K$ and therefore at $x$. Namely, $f$ is continuous. 
\end{proof}

\paragraph{49.}
\begin{proof}
  Let $x$ be an arbitrary point in $X$. For every $\vep > 0$, since $\mcal{F}$
  is equicontinuous, there is an open neighborhood $O$ of $x$ s.t. 
  $\sigma(f(x), f(y)) < \vep$ for every $y\in O$ and $f\in\mcal{F}$. Now, for
  every $f^+\in \mcal{F}^+$, suppose $f_n \to f$. We have
  \[
    \sigma(f^+(x), f^+(y)) 
    = \sigma\left(\lim_n f_n(x), \lim_m f_m(y)\right) 
    = \lim_n\lim_m \sigma(f_n(x), f_m(y)) \le \vep,
  \]
  where the second equality comes from the continuity of $\sigma$ and the last
  inequality comes from the equicontinuity of $\mcal{F}$.
\end{proof}

\paragraph{50.} I assume that the norm on $C[0, 1]$ is the sup norm. 
\begin{proof}
  Let $\mcal{F} := \{f\in C[0, 1]\mid \|f\|_\alpha \le 1 \}$. First, we show
  that $\mcal{F}$ is equicontinuous. By the definition of $\|\cdot\|_\alpha$,
  $f\in\mcal{F}$ only if it is bounded by $1$ and $|f(x) - f(y)| \le 
  |x - y|^\alpha$ for all $x, y \in [0, 1]$. For every $x\in X$ and $\vep > 0$,
  consider the $\sqrt[\alpha]{\vep}$-ball $B$ centered at $x$. For every
  $y \in B$, $|f(x) - f(y)| \le |x - y|^\alpha < \vep$. Hence, $\mcal{F}$ is
  equicontinuous. Let $(f_n)$ be a sequence in $\mcal{F}$. Since $f_n$ is
  bounded by $1$, by Corollary 41, $(f_n)$ contains a convergent subsequence. 
  Thus, $\mcal{F}$ is a (sequently) compact subset of $C[0, 1]$.
\end{proof}








