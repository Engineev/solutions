\section{The Classical Banach Spaces}
\subsection{The $L^p$ Spaces}
  \paragraph{1.}
  \begin{proof}
    Put $S=\|f\|_\infty$ and $T=\|g\|_\infty$. Then $|f(t)|\le S$ and $|g(t)|\le
    T$ a.e. Hence, $S+T\ge |f(t)|+|g(t)| \ge |f(t)+g(t)|$ a.e. Namely, $m\{t:
    \,|f(t)+g(t)|>S+T\}=0$. Thus, $S+T\ge \|f+g\|_\infty$ by the definition of 
    $\esssup$.
  \end{proof}

  \paragraph{2.}
  \begin{proof}
    Put $S=\|f\|_\infty$. Since $S\ge |f|$ a.e., 
    \[
      \|f\|_p=
      \left\{\int_0^1|f|^p\right\}^{1/p} \le 
      \left\{\int_0^1S^p\right\}^{1/p} = S.
    \]
    Therefore, $\uplim_{p\to\infty}\|f\|_p \le S$. For the converse part, let 
    $\vep$ be any positive number. Then the measure $\delta$ of $E=\{t:\,|f(t)|>
    S-\vep\}$ is positive. Hence,
    \[
      \left\{\int_0^1|f|^p\right\}^{1/p} \ge
      \left\{\int_E|f|^p\right\}^{1/p} \ge 
      \delta^{1/p}(S-\vep) \to 
      S-\vep\quad\text{as $p\to\infty$.}
    \]
    Hence, $\lowlim_{p\to\infty}\ge S$, completing the proof.
  \end{proof}

  \paragraph{3.}
  \begin{proof}
    \[
      \|f+g\|_1 = \int|f+g| \le \int|f|+\int|g| = \|f\|_1+\|g\|_1.
    \]
  \end{proof}

  \paragraph{4.}
  \begin{proof}
    For every $M>\|g\|_\infty$, $|g|\le M$ a.e. Hence,
    \[
      \int|fg|\le M\int|f| = \|f\|_1 M.
    \]
    Since the choice of $M$ is arbitrary, $\int|fg|\le \|f\|_1\|g\|_\infty$.
  \end{proof}
% end

\subsection{The Minkowski and Hölder Inequalities}
  \paragraph{8}
  \begin{proof}
    $\,$\par
    (a) The logarithm function is concave, so 
    \[
      \log(a^p/p+b^q/q) \ge \frac{1}{p}\log a^p + \frac{1}{q}\log b^q = \log ab.
    \]
    Taking $\exp$ on the both sides yields the inequality. The equality holds 
    iff $a^p=b^q$. \par
    (b) The case where $p=\infty$ has been proved in Problem 4 and the case 
    where $\|f\|_p=0$ or $\|g\|=0$ is straightforward. Hence, we assume that 
    $1<p,q<\infty$ and $\|f\|_p\|g\|_q\ne 0$. \par
    Suppose $\alpha=\|f\|_p$ and $\beta=\|g\|_q$. By Young's inequality, 
    \[
      \left|\frac{fg}{\alpha\beta}\right| \le 
      \frac{1}{p}\left(\frac{|f|}{\alpha}\right)^p + 
      \frac{1}{q}\left(\frac{|g|}{\beta}\right)^q
    \]
    for every $x$. Therefore,
    \begin{equation}
      \label{eq:6.2.8.1}
      \int|fg| = \alpha\beta\int\left|\frac{fg}{\alpha\beta}\right|
      \le \alpha\beta\int\left\{\frac{1}{p}\left(\frac{|f|}{\alpha}\right)^p + 
      \frac{1}{q}\left(\frac{|g|}{\beta}\right)^q\right\}
      = \alpha\beta.
    \end{equation}
    The equality holds iff the equality in Young's inequality holds a.e. iff 
    $\beta|f|^p = \alpha|g|^q$ a.e.\par
    (c) Let $p\hp=1/p$ and $q\hp=1-p\hp=-q/p$. Then for any nonnegative $c$ and
    $d$, by Young's inequality,
    \[
      cd \le \frac{c^{p\hp}}{p\hp}+\frac{d^{q\hp}}{q\hp}=
      pc^{1/p} - \frac{p}{q}d^{-q/p} \quad\Rightarrow\quad
      c^{1/p} \ge \frac{cd}{p} + \frac{d^{-q/p}}{q}.
    \]
    Putting $c=(ab)^p$ and $d=b^{-p}$ yields the desired inequality.\par
    (d) Just reverse the inequality in \eqref{eq:6.2.8.1}.
  \end{proof}
% end

\subsection{Convergence and Completeness}
  \paragraph{9.}
  \begin{proof}
    Suppose $\langle f_n\rangle\subset X$ converges to $f\in X$. Namely, for
    every $\vep>0$, there exists some $N$ such that for all $n>N$, $\|f_n-f\|<
    \vep$. Hence, for every $n,m>N$, by Minkowski inequality,
    \[
      \|f_n-f_m\| \le \|f_n-f\|+\|f-f_m\| < 2\vep.
    \]
    Hence, $\langle f_n\rangle$ is a Cauchy sequence.
  \end{proof}

  \paragraph{10.}
  \begin{proof}
    Suppose $f_n\to f$. Then $M_n=\|f_n-f\|_\infty=\esssup|f_n-f|\to 0$. Let
    $E_n=\{x:\,|f_n(x)-f(x)|>M_m\}$, each of which is with measure zero. And
    therefore $E=\bigcup_{n=1}^\infty E_n$ is with measure zero. Note that 
    $\tilde{E}=\{x:\,|f_n(x)-f(x)|<M_n,\forall\,n\}$, which implies the uniform
    convergence of $f_n$ since $M_n\to 0$.\par
    For the converse part, the uniform convergence on $\tilde{E}$ implies that
    for every $\vep>0$, there exists some $N$ such that for every $n>N$ and $x
    \in\tilde{E}$, $|f_n(x)-f(x)|<\vep$. Since $mE=0$, this implies $\|f_n-f
    \|_\infty=\esssup|f_n(x)-f(x)|<\vep$. Hence, $f_n\to f$ in $L^\infty$.
  \end{proof}

  \paragraph{11.}
  \begin{proof}
    Let $\langle f_n\rangle\subset L^\infty$ be absolutely summable. Put $M_n=
    \|f_n\|_\infty$ and $A_n=\{t:\,|f_n(t)|>M_n\}$. By the definition of $\|
    \cdot\|_\infty$, $mA_n=0$. Hence, $A=\bigcup_{n=1}^\infty A_n$ is of measure
    zero. \par
    Note that $|f_n(x)|\le M_n$ for every $n$ and $x\in E\setminus A$. Thus, by
    the Weierstrass M-test, $\sum_{n=1}^\infty f_n$ converges uniformly. Hence,
    on $E\setminus A$, $\sup|\sum_{n=1}^\infty f_n - \sum_{n=1}^N f_n|\to 0$ as
    $N\to\infty$. Since $mA=0$, this implies the summability of $\langle f_n
    \rangle$.
  \end{proof}

  \paragraph{13.}
  \begin{proof}
    Suppose $\langle f_n\rangle\subset C$ be absolutely summable. Since for
    every $x$, $0\le|f_n(x)|\le\|f_n\|$, $\langle f_n\rangle$ is uniformly
    convergent on $[0,1]$. Put $s=\sum_{n=1}^\infty f_n$. Since each $f_n$ is 
    continuous, so is $s$. Therefore, $s\in C$. \par
    For every $\vep>0$, there exists some $N$ such that for every $n>N$ and $x
    \in[0,1]$, $\left|s(x)-\sum_{k=1}^nf_k(x)\right|<\vep$. Hence, $\|s-
    \sum_{k=1}^nf_k\|<\vep$. Thus, $\langle f_n\rangle$ is summable and
    therefore $C$ is a Banach space.
  \end{proof}

  \paragraph{16.}
  \begin{proof}
    Since $\|f_n-f\| \ge |\|f_n\|-\|f\||$, $f_n\to f$ in $L^p$ implies $\|f_n\|
    \to \|f\|$. For the converse part, note that $2^p(|f_n|^p+|f|^p)-|f_n-f|^p
    \ge 0$ and for almost every $x$,
    \[
      2^p(|f_n|^p+|f|^p)-|f_n-f|^p \to 2^{p+1}|f|^p.
    \]
    By Fatou's Lemma,
    \begin{align*}
      2^{p+1}\|f\|^p = 2^{p+1}\int|f|^p 
      &\le \lowlim\int\{2^p(|f_n|^p+|f|^p)-|f_n-f|^p\} \\
      &= 2^{p+1}\|f\|^p - \uplim\|f_n-f\|^p.
    \end{align*}
    Hence, $\uplim\|f_n-f\|^p \le 0$. Since clear that $\lowlim\|f_n-f\|^p\ge0$,
    $\lim\|f_n-f\|=0$, i.e., $f_n \to f$ in $L^p$.
  \end{proof}

  \paragraph{17.}
    I assume that $1/p+1/q=1$.
  \begin{proof}
    Since $g\in L^p$, $|g|^q$ is integrable on $E=[0,1]$ and therefore for every
    $\vep>0$, there exists some $\delta$ such that for every $A\subset E$ with
    $mA<\delta$, $\int_A|g|^q<\vep$. Meanwhile, since $f_n(x)\to f(x)$ for 
    almost every $x$, by Egoroff's Theorem, there exists some $A\subset E$ with
    $mA<\delta$ such that $f_ng$ converges to $fg$ uniformly on $E\setminus A$.
    \par
    From the uniform convergence we conclude
    \begin{equation}
      \label{eq:6.17.1}
      \int_{E\setminus A} fg = \lim_{n\to\infty}\int_{E\setminus A}f_ng.
    \end{equation}
    Meanwhile, by Hölder inequality,
    \begin{align*}
      \left|\int_A (f-f_n)g\right| \le \int_A|(f-f_n)g|
      &\le\left\{\int_A|f_n-f|^p\right\}^{1/p}\left\{\int_A|g|^q\right\}^{1/q}
      \le M\vep^{1/q}.
    \end{align*}
    Hence, \eqref{eq:6.17.1} can be extended to $E$.\par
    For $p=1$, this is not true. $f_n=n\chi_{[0,1/n]}$ and $g=\chi_{[0,1]}$ 
    gives a counterexample.
  \end{proof}

  \paragraph{18.}
  \begin{proof}
    By Minkowski inequality, 
    \[
      \|g_nf_n-gf\| = \|g_n(f_n-f)+(g_n-g)f\| \le \|g_n(f_n-f)\|+\|(g_n-g)f\|.
    \]
    Fix $\vep>0$. Since $f,g_n,g\in L^p$, $|g_n-g|^p|f|^p$ is integrable and 
    therefore there exists some $\delta>0$ such that for all subsets with 
    measure $<\delta$, the integral of over it $<\vep$. Meanwhile, since $g_n\to
    g$ a.e., by Egoroff's Theorem, there exists some $A\subset E=[0,1]$ with $mA
    <\delta$ such that $g_n\to g$ uniformly on $E\setminus A$ and therefore 
    there exists some $N_1>0$ such that for all $n>N_1$, $|g_n(x)-g(x)|^p<\vep$
    for $x\in E\setminus A$. Thus, for every $n>N_1$,
    \begin{align*}
      \|(g_n-g)f\|
      &=\left\{\int_{E\setminus A}|g_n-g|^p|f|^p\right\}^{1/p} + 
        \left\{\int_A|g_n-g|^p|f|^p\right\}^{1/p} \\
      & \le \sqrt[p]{\vep}\|f\| + \sqrt[p]{\vep} \le (\|f\|+1)\vep.
    \end{align*}
    Since $|g_n|\le M$, $\|g_n(f_n-f)\|\le M\|f_n-f\|$. And since $f_n\to f$ in 
    $L^p$, there exists some $N_2>0$ such that for all $n>N_2$, $\|f_n-f\|<
    \vep$. Put $N=\max(N_1,N_2)$, then for every $n>N$,
    \[
      \|g_nf_n-gf\| \le (\|f\|+1+M)\vep.
    \]
    Hence, $g_nf_n\to gf$ in $L^p$.
  \end{proof}
% end
