\section{The Classical Banach Spaces}
\subsection{The $L^p$ Spaces}
  \paragraph{1.}
  \begin{proof}
    Put $S=\|f\|_\infty$ and $T=\|g\|_\infty$. Then $|f(t)|\le S$ and $|g(t)|\le
    T$ a.e.. Hence, $S+T\ge |f(t)|+|g(t)| \ge |f(t)+g(t)|$ a.e.. Namely, $m\{t:
    \,|f(t)+g(t)|>S+T\}=0$. Thus, $S+T\ge \|f+g\|_\infty$ by the definition of 
    $\esssup$.
  \end{proof}

  \paragraph{2.}
  \begin{proof}
    Put $S=\|f\|_\infty$. Since $S\ge |f|$ a.e., 
    \[
      \|f\|_p=
      \left\{\int_0^1|f|^p\right\}^{1/p} \le 
      \left\{\int_0^1S^p\right\}^{1/p} = S.
    \]
    Therefore, $\uplim_{p\to\infty}\|f\|_p \le S$. For the converse part, let 
    $\vep$ be any positive number. Then the measure $\delta$ of $E=\{t:\,|f(t)|>
    S-\vep\}$ is positive. Hence,
    \[
      \left\{\int_0^1|f|^p\right\}^{1/p} \ge
      \left\{\int_E|f|^p\right\}^{1/p} \ge 
      \delta^{1/p}(S-\vep) \to 
      S-\vep\quad\text{as $p\to\infty$.}
    \]
    Hence, $\lowlim_{p\to\infty}\ge S$, completing the proof.
  \end{proof}

  \paragraph{3.}
  \begin{proof}
    \[
      \|f+g\|_1 = \int|f+g| \le \int|f|+\int|g| = \|f\|_1+\|g\|_1.
    \]
  \end{proof}

  \paragraph{4.}
  \begin{proof}
    For every $M>\|g\|_\infty$, $|g|\le M$ a.e.. Hence,
    \[
      \int|fg|\le M\int|f| = \|f\|_1 M.
    \]
    Since the choice of $M$ is arbitrary, $\int|fg|\le \|f\|_1\|g\|_\infty$.
  \end{proof}
% end

\subsection{The Minkowski and Hölder Inequalities}
  \paragraph{8}
  \begin{proof}
    $\,$\par
    (a) The logarithm function is concave, so 
    \[
      \log(a^p/p+b^q/q) \ge \frac{1}{p}\log a^p + \frac{1}{q}\log b^q = \log ab.
    \]
    Taking $\exp$ on the both sides yields the inequality. The equality holds 
    iff $a^p=b^q$. \par
    (b) The case where $p=\infty$ has been proved in Problem 4 and the case 
    where $\|f\|_p=0$ or $\|g\|=0$ is straightforward. Hence, we assume that 
    $1<p,q<\infty$ and $\|f\|_p\|g\|_q\ne 0$. \par
    Suppose $\alpha=\|f\|_p$ and $\beta=\|g\|_q$. By Young's inequality, 
    \[
      \left|\frac{fg}{\alpha\beta}\right| \le 
      \frac{1}{p}\left(\frac{|f|}{\alpha}\right)^p + 
      \frac{1}{q}\left(\frac{|g|}{\beta}\right)^q
    \]
    for every $x$. Therefore,
    \begin{equation}
      \label{eq:6.2.8.1}
      \int|fg| = \alpha\beta\int\left|\frac{fg}{\alpha\beta}\right|
      \le \alpha\beta\int\left\{\frac{1}{p}\left(\frac{|f|}{\alpha}\right)^p + 
      \frac{1}{q}\left(\frac{|g|}{\beta}\right)^q\right\}
      = \alpha\beta.
    \end{equation}
    The equality holds iff the equality in Young's inequality holds a.e. iff 
    $\beta|f|^p = \alpha|g|^q$ a.e..\par
    (c) Let $p\hp=1/p$ and $q\hp=1-p\hp=-q/p$. Then for any nonnegative $c$ and
    $d$, by Young's inequality,
    \[
      cd \le \frac{c^{p\hp}}{p\hp}+\frac{d^{q\hp}}{q\hp}=
      pc^{1/p} - \frac{p}{q}d^{-q/p} \quad\Rightarrow\quad
      c^{1/p} \ge \frac{cd}{p} + \frac{d^{-q/p}}{q}.
    \]
    Putting $c=(ab)^p$ and $d=b^{-p}$ yields the desired inequality.\par
    (d) Just reverse the inequality in \eqref{eq:6.2.8.1}.
  \end{proof}
% end
