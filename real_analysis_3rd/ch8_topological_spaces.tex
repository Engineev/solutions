\section{Topological Spaces}
\subsection{Fundamental Notions}
\paragraph{3.}
\begin{proof}
  If $A$ is open, then clear that $x \in A \subset A$. For the converse, let 
  $E = \bigcup_{x\in A}O_x$ where $x\in O_x \subset A$. Since $O_x\subset A$,
  $E \subset A$. Meanwhile, for every $x\in A$, $x \in O_x\subset E$. Hence, 
  $A = E$. Thus, $A$ is open since $E$ is the union of open sets.
\end{proof}

\paragraph{7.}
\begin{proof}
  $\,$\par
  (a) We argue by contradiction. Assume $x \in F^c$. Since $F$ is closed, $F^c$
  is open and therefore there is neighborhood $O$ of $x$ s.t. $O \cap F
  \ne \varnothing$, which contradicts the fact that $x$ is a cluster point.
  Thus, $x \in F$.
  
  (b) Let $y := f(x)$ and $y_n := f(x_n)$. Let $O$ be an arbitrary
  neighborhood of $y$. Since $f$ is continuous, there is a neighborhood $U$ of
  $x$ s.t. $f(U) \subset O$. Since $x = \lim x_n$, there is an integer $N$ s.t.
  $x_n \in U$ for every $n > N$. Hence, $y_n = f(x_n) \in f(U) \subset O$ for
  all $n > N$. Thus, $y_n \to y$.
  
  (c) The previous argument, \textit{mutatis mutandis}, yields the result.
\end{proof}

\paragraph{10.}
\begin{proof}
  $\,$\par
  (a) Suppose that both $A_1$ and $A_2$ are open.
  Let $f_1 := f|_{A_1}$, $x\in A$ and $y = f(x)$. We may assume without
  loss of generality that $x \in A_1$. For every neighborhood $O$ of $y$, since
  $f_1$ is continuous, there is a neighborhood $A_1\cap U$ s.t. $f(A_1\cap U)
  \subset O$ where $U$ is an open set of $X$. Since $A_1\cap U$ is still open,
  $f$ is continuous at $x$. Thus, $f$ is continuous.
  
  (b) Let $f(x) = 0$ on $A_1 = (-1, 0)$ and $f(x) = 1$ on $A_2 = [0, 1]$.
\end{proof}

\subsection{Bases and Contability}
\paragraph{11.}
\begin{proof}
  $\,$\par
  (a) Suppose for every $B\in\mcal{B}$ containing $x$, there is a $y\in 
  B\cap E$. Let $O$ be an arbitrary open set containing $x$. By the definition
  of bases, there is a base set $B$ s.t. $x \in B \subset \mcal{B}$. Thus,
  $O\cap E \supset B\cap E \ne \varnothing$. Namely, $x \in \cl E$. The
  reversed direction follows immediately from the definition of cluster points.
  
  (b) If there is sequence in $E$ that converges to $x$, clear that $x\in
  \cl E$. Now we show the reverse. Let $\mcal{B}_x = (B_k)$ be a countable base
  at $x$ and $S_n = \bigcap_{k=1}^n B_k$. Choose $x_n \in S_n$. For every
  open set $O$ containing $x$, there exists a $B_m$ s.t. $x \in B_m
  \subset O$. Hence, for every $n > m$, $x_n \in S_n\subset S_m
  \subset S_m \subset O$. Thus, $x_n \to x$.
  
  (c) It follows immediately from (b).
\end{proof}

\paragraph{13.}
\begin{proof}
  By the construction of $\mcal{B}$ and Prop. 5, $\mcal{B}$ is a base for some
  topology on $X$. Let $\mcal{T}$ be a topology containing $\mcal{C}$. Since
  $X \in \mcal{T}$ and $\mcal{T}$ is closed under finite intersection,
  $\mcal{T} \supset \mcal{B}$. Thus, $\mcal{T}$ contains the topology generated
  by $\mcal{B}$. Since the choice of $\mcal{T}$ is arbitrary, we conclude that
  $\mcal{B}$ is a base for the weakest topology containing $\mcal{C}$.
\end{proof}


\paragraph{16.}
\begin{proof}
  Let $\mcal{B}$ be a countable base for the topology on $X$ and 
  $\mcal{U}$ an open cover of $X$. For each $B\in\mcal{B}$, if there is
  some $U \in \mcal{U}$ with $B \subset U$, then pick that $U$. This yields
  a at most countable subset $\mcal{V}$ of $\mcal{U}$. Now we show that 
  $\mcal{V}$ covers $X$. For every $x\in X$, since $\mcal{U}$ covers $X$, there
  is a $U\in\mcal{U}$ s.t. $x\in U$. Meanwhile, since $\mcal{B}$ is a base, 
  there is a $B \in \mcal{B}$ s.t. $x\in B \subset U$. Hence, by our
  construction, there is a $U\hp \in \mcal{U}$ containing $x$ is picked. Thus,
  $x$ is covered by $\mcal{V}$.
\end{proof}























