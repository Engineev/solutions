\section{Differentiation and Integration}
\subsection{Differentiation of Monotone Functions}
  \paragraph{3.}
    "maximum" needs to be changed to "minimum" in both (a) and (b).
  \begin{proof}
    $\,$\par
    (a) We may assume without loss of generality that $c=0$. Since $f$ attains a
    local minimum at $x=0$, $f(h)\ge f(0)$ for every $h$ sufficiently small.
    Hence, for every small $h>0$, $(f(c+h)-f(c))/h>0$ and therefore $D_+f(c)\ge
    0$. Meanwhile, by Problem 2.b, 
    \[
      -D_{-}f(0) = D^+f(0) \ge 0 \quad\Rightarrow\quad 
      D_{-}f(0) \le 0.
    \]
    The other two inequalities follow immediately from the definitions of upper 
    and lower limits.\par
    (b) If $f$ has a local minimum at $a$ or $b$, then we only have the right or
    left half of the inequalities.
  \end{proof}

  \paragraph{4.}
  \begin{proof}
    We first show this for $g$ with $D^+g\ge\vep>0$. For every $a\le x<y\le b$,
    as $g$ is continuous on $[a,b]$, $g$ has a maximum in $[a,b]$ and by Problem
    2 and 3, $g$ can not attain the maximum in $[a,b)$. Namely, the restrict of
    $f$ to $[x,y]$ attains the maximum at $y$. Hence, $g(x)\le g(y)$.\par
    For every $f$ with nonnegative $D^+$, let $g(x)=f(x)+\vep x$ where $\vep>0$.
    Then $D^+g\ge\vep >0$. Hence $g$ is nondecreasing. Therefore, for every $a
    \le x<y\le b$, 
    \[
      g(x)\le g(y) \quad\Rightarrow\quad f(x)+\vep x \le f(y)+\vep y.
    \]
    Since the choice of $\vep$ is arbitrary, this implies $f(x)\le f(y)$.
  \end{proof}

  \paragraph{5.a}
  \begin{proof}
    \begin{align*}
       \sup_{t\in(0,h)}\frac{(f+g)(x+t)-(f+g)(x)}{t} 
      =&\sup_{t\in(0,h)}\left(\frac{f(x+t)-f(x)}{t}+
        \frac{g(x+t)-g(x)}{t}\right) \\
      \le&\sup_{t\in(0,h)}\frac{f(x+t)-f(x)}{t}+
          \sup_{t\in(0,h)}\frac{g(x+t)-g(x)}{t}.
    \end{align*}
    Letting $h\to 0$ yields $D^+(f+g)\le D^+f+D^+g$.
  \end{proof}
% end

\subsection{Functions of Bounded Variation}
  \paragraph{7.}
  \begin{proof}
    $\,$\par
    (a) It suffices to show this for monotone functions as each function of 
    bounded variation is the difference of two monotone functions. Suppose that
    $f$ is nondecreasing. Then the set $E=\{f(x):x>c\}$ is bounded below and 
    hence $A=\inf E$ is finite. For every $\vep>0$, there exists some $y>c$ such 
    that $A\le f(c) <A+\vep$. Hence, as $f$ is nondecreasing, for every $x\in(c,
    y)$, $|f(x)-A|<\vep$. Namely, $\lim_{x\to c+}f(x)=A$. Similarly, $\lim_{x\to
    c-}f(x)$ exists.\par
    Let $D_n=\{x:\, |f(x+)-f(x-)|>1/n\}$. Since $f$ is nondecreasing, $|f(x)-
    f(y)|\le f(b)-f(a)<\infty$ for every $x,y\in[a,b]$. Hence, $D_n$ is finite,
    otherwise we can choose a sequence $x_1<\dots<x_N$ with $N>(f(b)-f(a))/n$
    such that $f(X_N)-f(x_1)> f(b)-f(a)$. Therefore, $\bigcup_{n=1}^\infty E_n$,
    the set of discontinuities, is countable.\par
    (b) Suppose $\{x_1,\dots,x_n,\dots\}=\mathbb{Q}\cap[0,1]$ and define $f(x)=
    \sum_{x_n<x}2^{-n}$. Clear that $f$ is monotone and continuous at every 
    irrational point. For each rational $x=x_k$, $f(x+)-f(x-)=2^{-n}$. Hence,
    $f$ is discontinuous at each rational point.
  \end{proof}

  \paragraph{8.}
  \begin{proof}
    $\,$\par
    (a) For every $\vep>0$, there exists some subdivision $a=x_0<\cdots<x_p=c$ 
    and $c=x_p<\cdots<x_q=b$ of $[a,c]$ and $[c,b]$ such that $T_a^c<t_a^c+\vep$
    and $T_c^b<t_c^b+\vep$. Hence, $T_a^c+T_c^b-2\vep < t_a^c+t_c^b$. Meanwhile,
    as $a=x_0<\cdots<x_q=b$ forms a subdivision of $[a,b]$, $T_a^b\ge t_a^b =
    t_a^c+t_c^b$. Therefore, $T_a^c+T_c^b-2\vep < T_a^b$. Since the choice of
    $\vep$ is arbitrary, $T_a^b+T_c^b\le T_a^b$.\par
    To show that $T_a^b+T_c^b\ge T_a^b$, let $a=x_0<\cdots<x_q=b$ be 
    any subdivision of $[a,b]$ and by adding $c$ into it, we get subdivisions of
    $[a,c]$ and $[c,b]$. Suppose that $c\in(x_k, x_{k+1}]$, then
    \[
      |f(x_k)-f(c)|+|f(c)-f(x_{k+1})|+t_a^b = t_a^c+t_c^b+|f(x_k)-f(x_{k+1})|,
    \]
    which implies $t_a^b\le t_a^c+t_c^b$. Hence, 
    \[
      T_a^b = \sup t_a^b \le \sup(t_a^c+t_c^b) \le T_a^c + T_c^b.
    \]\par
    Thus, $T_a^b=T_a^c+T_c^b$ and therefore $T_a^c\le T_a^b$.\par
    (b) 
    \begin{align*}
      T_a^b(f+g) 
      &=   \sup\sum_{i=1}^k|f(x_i)+g(x_i)-f(x_{i-1})-g(x_{i-1})| \\
      &\le \sup\sum_{i=1}^k|f(x_i)-f(x_{i-1})| + 
           \sup\sum_{i=1}^k|g(x_i)-g(x_{i-1})| \\
      &\le T_a^b(f) + T_a^b(g).
    \end{align*}
    \begin{align*}
      T_a^b(cf) 
      = \sup\sum_{i=1}^k|cf(x_i-cf(x_{i-1})| 
      = |c|\sup\sum_{i=1}^k|f(x_i-f(x_{i-1})| 
      = |c|T_a^b(f).
    \end{align*}
  \end{proof}

  \paragraph{9.}
  \begin{proof}
    For every $\vep>0$, there exists a subdivision $a=x_0<\cdots<x_k=b$ such
    that $t_a^b(f)\ge T_a^b(f)-\vep$. Meanwhile, as $f_n$ converges to $f$
    pointwisely
    \begin{align*}
      t_a^b(f) &= t_a^b(\lim f_n) \\
      &= \sum_{i=1}^k|(\lim f_n)(x_i) - (\lim f_n)(x_{i-1})| \\
      &= \lim \sum_{i=1}^k|f_n(x_i)-f_n(x_{i-1})| \\
      &\le \lowlim\sup\sum_{i=1}^k|f_n(x_i)-f_n(x_{i-1})| 
      = \lowlim T_a^b(f_n).
    \end{align*}
    Hence, $T_a^b(f)-\vep \le \lowlim T_a^b(f_n)$. Since the choice of $\vep$ is
    arbitrary, $T_a^b(f)\le\lowlim T_a^b(f_n)$.
  \end{proof}

  \paragraph{10.a}
  \begin{solution}
    No. Let $x_k = (k\pi+\pi/2)^{-1/2}$, $k=0,1,\dots$ and consider the 
    subdivision $-1<0<x_n<\cdots<x_0<1$. Then
    \[
      t_n \ge \sum_{k=1}^n|f(x_k)-f(x_{k-1})|
      \ge\sum_{k=1}^n\frac{2}{(k+1/2)\pi}.
    \]
    $t_n\to\infty$ as $n\to\infty$ and therefore $f$ is not of bounded 
    variation on $[-1,1]$.
  \end{solution}

  \paragraph{11.}
  \begin{proof}
    By Lemma 4, $f(x)=f(a)+P_a^x-N_a^x$. Since $P_a^x$ and $N_a^x$ are monotone,
    by Theorem 3, they are differentiable almost everywhere as $f$, a function 
    of bounded variation, does. Hence, for almost every $x\in[a,b]$,
    \[
      \frac{\rd}{\rd x}f(x) = \frac{\rd}{\rd x}P_a^x - \frac{\rd}{\rd x}N_a^x
      \quad\Rightarrow\quad
      |f\hp(x)| \le \frac{\rd}{\rd x}P_a^x+\frac{\rd}{\rd x}N_a^x = 
      \frac{\rd}{\rd x}T_a^x.
    \]
    Integrating on the both sides yields $\int_a^b|f\hp|\le T_a^b(f)$.
  \end{proof}

% end

\setcounter{subsection}{3}
\subsection{Absolute Continuity}
  \paragraph{12.}
  \begin{solution}
    The continuous extension of $x^2\sin(1/x^2)$ to $[0,1]$ is absolutely 
    continuous for all $[\vep, 1]$ but is not of bounded variation on $[0,1]$
    and therefore is not absolutely continuous on $[0,1]$.\par
    Suppose that $f$ is also of bounded variation on $[0,1]$. Then $f$ is 
    differentiable almost everywhere. Hence $g(x)=\int_0^xf\hp(t)\rd t+f(a)$ is
    well-defined. For every $\vep>0$, we have
    \[
      g(x) = \int_0^\vep f\hp(t)\rd t+\int_\vep^x f\hp(t)\rd t + f(0)
      = \int_0^\vep f\hp(t)\rd t + f(x) - f(\vep) + f(0),
    \]
    where the second equality comes from the absolute continuity on $[\vep,1]$.
    By the continuity of $f$ at $x=0$, $f(\vep)\to f(0)$. Hence, letting $\vep
    \to 0$ yields $g(x)=f(x)$. Namely, $f$ is an indefinite integral. Thus, by
    Theorem 14, it is absolutely continuous.
  \end{solution}

  \paragraph{13.}
  \begin{proof}
    Since absolute continuity implies bounded variation, $\int_a^b|f\hp|\le 
    T_a^b(f)$ by Problem 11. By the definiton of $T$, for every $\vep>0$, there
    exists some subdivision $a=x_0<\cdots<x_n=b$ such that $t_a^b(f)>T_a^b(f)-
    \vep$. Meanwhile, for every $i=1,\dots,n$,
    \[
      \int_{x_{i-1}}^{x_i}|f\hp| \ge 
      \left|\int_{x_{i-1}}^{x_i}f\hp \right| =
      |f(x_i) - f(x_{i-1})|,
    \]
    where the second equality is guaranteed by the absolute continuity. Hence,
    $\int_a^b|f\hp|>T_a^b(f)-\vep$ for every $\vep>0$. Thus, $T_a^b(f)=
    \int_a^b|f\hp|$.\par
    By Lemma 4, $2P_a^b(f) = T_a^b(f) + f(b)-f(a)$. Hence,
    \[
      P_a^b(f) = \frac{1}{2}\left(\int_a^b|f\hp| + f(b)-f(a)\right)
      =\frac{1}{2}\int_a^b (|f\hp|+f\hp) = \int_a^b[f\hp]^+.
    \]
  \end{proof}

  \paragraph{14.}
  \begin{proof}
    $\,$\par
    (a) Suppose that $f$ and $g$ are absolutely continuous. Then for every 
    $\vep>0$, there exists some $\delta>0$ such that for all finite 
    nonoverlapping $\langle(x_n,y_n)\rangle$ with $|x_n-y_n|<\vep$, 
    \[
      \sum |f(x_n)+g(x_n)-f(y_n)-g(y_n)| \le
      \sum |f(x_n)-f(y_n)|+|g(x_n)-g(y_n)| \le 2\vep.
    \]
    Hence, $f+g$ is also absolutely continuous. Since $-g$ is absolutely 
    continuous as long as $g$ is, so is $f-g$. \par
    (b) Suppose that $f$ and $g$ are absolutely continuous. Then they are 
    bounded, by $M$ for example. Hence for every $\vep>0$, there exists some
    $\delta>0$ such that for all finite nonoverlapping $\langle(x_n,y_n)\rangle$
    with $|x_n-y_n|<\vep$, 
    \begin{align*}
      &\sum |f(x_n)g(x_n)-f(y_n)g(y_n)| \\
      =&\sum |f(x_n)g(x_n)-f(x_n)g(y_n) + f(x_n)g(y_n)-f(y_n)g(y_n)| \\ 
      \le&\sum \{|f(x_n)||g(x_n)-g(y_n)| + |f(x_n)-f(y_n)||g(y_n)|\} \\
      \le& M\vep.
    \end{align*}
    Thus, $fg$ is also absolutely continuous.\par
    (c) Since $f$ is continuous on $[a,b]$, $f$ can achieve its minimum in $[a,
    b]$. Hence, $|f(x)|\ge m>0$ as $f$ is never zero. Therefore for every $\vep>
    0$, there exists some $\delta>0$ such that for all finite nonoverlapping
    $\langle(x_n,y_n)\rangle$ with $|x_n-y_n|<\vep$,
    \begin{align*}
        \sum\left|\frac{1}{f(x_n)}-\frac{1}{f(y_n)}\right|
      = \sum\left|\frac{f(x_n)-f(y_n)}{f(x_n)f(y_n)}\right|
      \le \frac{1}{m^2}\sum|f(x_n)-f(y_n)|\le \frac{\vep}{m^2}.
    \end{align*}
  \end{proof}

  \paragraph{17.}
    Part (a) is wrong. It can be fixed if we further require $g$ to be monotone
    increasing.
  \begin{proof}
    $\,$\par
    (a) For every $\vep>0$, let $\delta_1$ be the number in the definition of 
    $F$ corresponding to $\vep$ and $\delta_2$ the number in the definition of
    $g$ corresponding to $\delta_1$. Then for every finite nonoverlapping
    $\langle (x_n, y_n)\rangle$ with $|x_n-y_n|<\delta_2$, $\sum |g(x_n)-g(y_n)|
    < \delta_1$. Since $g$ is monotone increasing, $(g(x_n), g(y_n))$ are 
    nonoverlapping. Therefore, $\sum|F(g(x_n))-F(g(y_n))|<\vep$. Hence, $F\circ
    g$ is absolutely continuous.
  \end{proof}

  \paragraph{18.}
  \begin{proof}
    Without loss of generality, we assume that $g$ is nondecreasing.
    Since $mE=0$, for every $\vep>0$, by Proposition 3.15, there exists an open
    set $O\supset E$ with $mO<\vep$. Meanwhile, there exists a sequence of 
    disjoint open intervals $\langle I_n=(a_n,b_n)\rangle$ such that 
    $\bigcup_{n=1}^\infty I_n=O$ and $l(I_n)<\delta$ where $\delta$ is the 
    number in the definition of absolute continuity. Then $g[E]\subset
    \bigcup_{n=1}^\infty g[I_n\cap[0,1]]$. Since $g$ is continuous, the image of
    an interval is still an interval and since $g$ is also nondecreasing, $g[I_n
    \cap[0,1]]=(g(a_n\hp), g(b_n\hp))$, where $a_n\hp=\max\{a_n,0\}$ and $b_n\hp
    =\min\{b_n,1\}$. Finally,
    \[
     m(g[E]) \le \sum_{n=1}^\infty m(g[I_n]) = \sum_{n=1}^\infty|g(b_n\hp)-
     g(a_n\hp)| \le \vep,
    \]
    where the last inequality comes from the absolute continuity of $g$. Since
    the choice of $\vep$ is arbitrary, $m(g[E])=0$.
  \end{proof}

  \paragraph{20.}
  \begin{proof}
    $\,$\par
    (a) For every $\vep>0$, let $\delta=\vep/M$. Then for every $\langle x_n
    \rangle_{i=1}^n$ and $\langle y_n\rangle_{i=1}^n$ with $|x_n-y_n|\le\delta$,
    \[
      \sum_{i=1}^n|f(x_n)-f(y_n)|\le M\sum_{i=1}^n|x_n-y_n| \le \vep,
    \]
    as $f$ satisfies the Lipschitz condition.\par
    (b) Suppose that $f$ is absolute continuous and $|f\hp|$ is bounded by $M$.
    Then for every $x$ and $y$ in the interval,
    \[
      |f(x)-f(y)|=\left|\int_x^y f\hp(t)\rd t\right| \le M|x-y|.
    \]
    Hence, $f$ satisfies the Lipschitz condition. The converse part has been 
    proved in (a).\par
    (c) It is wrong. A counterexample is $f(x)=\chi_{[0,1]}$, $x\in(-1,1)$
  \end{proof}

  \paragraph{21.}
  \begin{proof}
    $\,$\par
    (a) Suppose that $O=\bigcup_{n=1}^\infty(c_n,d_n)$ where $(c_n,d_n)$ are 
    disjoint. Since $g$ is continuous and increasing, $g\inv(c_n,d_n)$ is still
    an open interval, denoting it by $(a_n,b_n)$, and $(a_n,b_n)$ are also 
    disjoint. Meanwhile, $d_n-c_n=f(a_n)-f(b_n)=\int_{a_n}^{b_n}g\hp $. Hence,
    \[
      mO = m\left(\bigcup_{n=1}^\infty(c_n,d_n)\right) = 
      \sum_{n=1}^\infty(d_n-c_n) = 
      \sum_{n=1}^\infty \int_{a_n}^{b_n}g\hp = 
      \int_{g\inv[O]}g\hp.
    \]\par
    (b) Without loss of generality, we assume that $d\notin E$. For every $\vep
    >0$, there exists an open set $O\supset E$ with $mO<\vep$. By Part (a),
    \[
      \int_{g\inv[O]\cap H}g\hp = \int_{g\inv[O]}g\hp = mO < \vep.
    \]
    Since the choice of $\vep$ is arbitrary, $\int_{g\inv[O]\cap H}g\hp = 0$.
    Since $g\hp > 0$ on $H$, $g\inv[O]\cap H$ has measure zero.\par
    (c) Since $E$ is measurable, so is $g\inv[E]$. Meanwhile, by Theorem 3, 
    $g\hp$ is measurable, hence $H$ is also measurable. Therefore, $F$ is
    measurable.\par
    We may assume without loss of generality that $c,d\notin E$. By Proposition
    3.15, there exists some $G\in G_\delta$ such that $E\subset G\subset(c,d)$
    and $m(G\setminus E)=0$. Since $g$ is increasing, $g\inv[G]\cap H = F\cup
    (g\inv[G\setminus E]\cap H)$ and by (b), $g[G\setminus E]\cap H$ is of
    measure zero. Therefore, $\int_F g\hp=\int_{g\inv[G]\cap H}g\hp$. Namely, it 
    suffices to show the result for $G\in G_\delta$. \par
    Suppose that $G=\bigcap_{n=1}^\infty O_n$ where each $O_n\subset(c,d)$ is 
    open and $mO_1<\infty$. Without loss of generality, we may assume that 
    $\langle O_n\rangle$ is decreasing. Then $mG=\lim_{n\to\infty}mO_n$. By (a),
    \[
      mO_n = \int_{g\inv[O_n]}g\hp = \int_a^b\chi_{O_n}(g(x))g\hp(x)\rd x.
    \]
    As $\chi_{O_n}(g(x))g\hp(x)$ is bounded by $|g\hp|$, 
    \[
      \lim_{n\to\infty}\int_a^b\chi_{O_n}(g(x))g\hp(x)\rd x =
      \int_a^b\chi_G(g(x))g\hp(x)\rd x.
    \]
    Hence, $mG=\int_{g\inv[G]\cap H}g\hp$, completing the proof.\par
    (d) By Problem 3.25, $f\circ g$ is measurable. And since $g\hp$ is 
    measurable by Theorem 3, $(f\circ g)g\hp$ is also measurable. \par
    Let $\langle\varphi_n\rangle$ be an increasing sequence of nonnegative
    simple functions which converges to $f$, the existence of which is 
    guaranteed by Problem 4.4. By the monotone convergence theorem, $\int_c^d f
    =\lim\int_c^d\varphi_n$.\par
    For each $n$, suppose that $\varphi_n(y)=\sum_{k=1}^ma_k^{(n)}(y)
    \chi_{E_k^{(n)}}(y)$. Then
    \[
      \int_c^d\varphi_n = \sum_{k=1}^ma_k^{(n)}mE_{k}^{(n)}=
      \sum_{k=1}^ma_k^{(n)}\int_a^b\chi_{E_k^{(n)}}(g(x))g\hp(x)\rd x=
      \int_a^b\varphi_n(g(x))g\hp(x)\rd x,
    \]
    where the second equality comes from (c). Since $g$ is increasing, 
    $\langle\varphi_n(g(x))g\hp(x)\rangle$ is an increasing sequence. Hence,
    \[
      \int_a^b f(g(x))g\hp(x)\rd x =
      \lim_{n\to\infty}\int_a^b\varphi_n(g(x))g\hp(x)\rd x.
    \]
    Thus,
    \[
      \int_c^d f(y)\rd y = 
      \lim_{n\to\infty}\int_c^d\varphi_n(y)\rd y =
      \lim_{n\to\infty}\int_a^b\varphi_n(g(x))g\hp(x)\rd x =
      \int_a^b f(g(x))g\hp(x)\rd x.
    \]
  \end{proof}

% end

