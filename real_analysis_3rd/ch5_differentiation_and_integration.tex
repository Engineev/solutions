\section{Differentiation and Integration}
\subsection{Differentiation of Monotone Functions}
  \paragraph{3.}
    "maximum" needs to be changed to "minimum" in both (a) and (b).
  \begin{proof}
    $\,$\par
    (a) We may assume without loss of generality that $c=0$. Since $f$ attains a
    local minimum at $x=0$, $f(h)\ge f(0)$ for every $h$ sufficiently small.
    Hence, for every small $h>0$, $(f(c+h)-f(c))/h>0$ and therefore $D_+f(c)\ge
    0$. Meanwhile, by Problem 2.b, 
    \[
      -D_{-}f(0) = D^+f(0) \ge 0 \quad\Rightarrow\quad 
      D_{-}f(0) \le 0.
    \]
    The other two inequalities follow immediately from the definitions of upper 
    and lower limits.\par
    (b) If $f$ has a local minimum at $a$ or $b$, then we only have the right or
    left half of the inequalities.
  \end{proof}

  \paragraph{4.}
  \begin{proof}
    We first show this for $g$ with $D^+g\ge\vep>0$. For every $a\le x<y\le b$,
    as $g$ is continuous on $[a,b]$, $g$ has a maximum in $[a,b]$ and by Problem
    2 and 3, $g$ can not attain the maximum in $[a,b)$. Namely, the restrict of
    $f$ to $[x,y]$ attains the maximum at $y$. Hence, $g(x)\le g(y)$.\par
    For every $f$ with nonnegative $D^+$, let $g(x)=f(x)+\vep x$ where $\vep>0$.
    Then $D^+g\ge\vep >0$. Hence $g$ is nondecreasing. Therefore, for every $a
    \le x<y\le b$, 
    \[
      g(x)\le g(y) \quad\Rightarrow\quad f(x)+\vep x \le f(y)+\vep y.
    \]
    Since the choice of $\vep$ is arbitrary, this implies $f(x)\le f(y)$.
  \end{proof}

  \paragraph{5.a}
  \begin{proof}
    \begin{align*}
       \sup_{t\in(0,h)}\frac{(f+g)(x+t)-(f+g)(x)}{t} 
      =&\sup_{t\in(0,h)}\left(\frac{f(x+t)-f(x)}{t}+
        \frac{g(x+t)-g(x)}{t}\right) \\
      \le&\sup_{t\in(0,h)}\frac{f(x+t)-f(x)}{t}+
          \sup_{t\in(0,h)}\frac{g(x+t)-g(x)}{t}.
    \end{align*}
    Letting $h\to 0$ yields $D^+(f+g)\le D^+f+D^+g$.
  \end{proof}
% end

\subsection{Functions of Bounded Variation}
  \paragraph{7.}
  \begin{proof}
    $\,$\par
    (a) It suffices to show this for monotone functions as each function of 
    bounded variation is the difference of two monotone functions. Suppose that
    $f$ is nondecreasing. Then the set $E=\{f(x):x>c\}$ is bounded below and 
    hence $A=\inf E$ is finite. For every $\vep>0$, there exists some $y>c$ such 
    that $A\le f(c) <A+\vep$. Hence, as $f$ is nondecreasing, for every $x\in(c,
    y)$, $|f(x)-A|<\vep$. Namely, $\lim_{x\to c+}f(x)=A$. Similarly, $\lim_{x\to
    c-}f(x)$ exists.\par
    Let $D_n=\{x:\, |f(x+)-f(x-)|>1/n\}$. Since $f$ is nondecreasing, $|f(x)-
    f(y)|\le f(b)-f(a)<\infty$ for every $x,y\in[a,b]$. Hence, $D_n$ is finite,
    otherwise we can choose a sequence $x_1<\dots<x_N$ with $N>(f(b)-f(a))/n$
    such that $f(X_N)-f(x_1)> f(b)-f(a)$. Therefore, $\bigcup_{n=1}^\infty E_n$,
    the set of discontinuities, is countable.\par
    (b) Suppose $\{x_1,\dots,x_n,\dots\}=\mathbb{Q}\cap[0,1]$ and define $f(x)=
    \sum_{x_n<x}2^{-n}$. Clear that $f$ is monotone and continuous at every 
    irrational point. For each rational $x=x_k$, $f(x+)-f(x-)=2^{-n}$. Hence,
    $f$ is discontinuous at each rational point.
  \end{proof}

  \paragraph{8.}
  \begin{proof}
    $\,$\par
    (a) For every $\vep>0$, there exists some subdivision $a=x_0<\cdots<x_p=c$ 
    and $c=x_p<\cdots<x_q=b$ of $[a,c]$ and $[c,b]$ such that $T_a^c<t_a^c+\vep$
    and $T_c^b<t_c^b+\vep$. Hence, $T_a^c+T_c^b-2\vep < t_a^c+t_c^b$. Meanwhile,
    as $a=x_0<\cdots<x_q=b$ forms a subdivision of $[a,b]$, $T_a^b\ge t_a^b =
    t_a^c+t_c^b$. Therefore, $T_a^c+T_c^b-2\vep < T_a^b$. Since the choice of
    $\vep$ is arbitrary, $T_a^b+T_c^b\le T_a^b$.\par
    To show that $T_a^b+T_c^b\ge T_a^b$, let $a=x_0<\cdots<x_q=b$ be 
    any subdivision of $[a,b]$ and by adding $c$ into it, we get subdivisions of
    $[a,c]$ and $[c,b]$. Suppose that $c\in(x_k, x_{k+1}]$, then
    \[
      |f(x_k)-f(c)|+|f(c)-f(x_{k+1})|+t_a^b = t_a^c+t_c^b+|f(x_k)-f(x_{k+1})|,
    \]
    which implies $t_a^b\le t_a^c+t_c^b$. Hence, 
    \[
      T_a^b = \sup t_a^b \le \sup(t_a^c+t_c^b) \le T_a^c + T_c^b.
    \]\par
    Thus, $T_a^b=T_a^c+T_c^b$ and therefore $T_a^c\le T_a^b$.\par
    (b) 
    \begin{align*}
      T_a^b(f+g) 
      &=   \sup\sum_{i=1}^k|f(x_i)+g(x_i)-f(x_{i-1})-g(x_{i-1})| \\
      &\le \sup\sum_{i=1}^k|f(x_i)-f(x_{i-1})| + 
           \sup\sum_{i=1}^k|g(x_i)-g(x_{i-1})| \\
      &\le T_a^b(f) + T_a^b(g).
    \end{align*}
    \begin{align*}
      T_a^b(cf) 
      = \sup\sum_{i=1}^k|cf(x_i-cf(x_{i-1})| 
      = |c|\sup\sum_{i=1}^k|f(x_i-f(x_{i-1})| 
      = |c|T_a^b(f).
    \end{align*}
  \end{proof}

  \paragraph{9.}
  \begin{proof}
    For every $\vep>0$, there exists a subdivision $a=x_0<\cdots<x_k=b$ such
    that $t_a^b(f)\ge T_a^b(f)-\vep$. Meanwhile, as $f_n$ converges to $f$
    pointwisely
    \begin{align*}
      t_a^b(f) &= t_a^b(\lim f_n) \\
      &= \sum_{i=1}^k|(\lim f_n)(x_i) - (\lim f_n)(x_{i-1})| \\
      &= \lim \sum_{i=1}^k|f_n(x_i)-f_n(x_{i-1})| \\
      &\le \lowlim\sup\sum_{i=1}^k|f_n(x_i)-f_n(x_{i-1})| 
      = \lowlim T_a^b(f_n).
    \end{align*}
    Hence, $T_a^b(f)-\vep \le \lowlim T_a^b(f_n)$. Since the choice of $\vep$ is
    arbitrary, $T_a^b(f)\le\lowlim T_a^b(f_n)$.
  \end{proof}

  \paragraph{10.a}
  \begin{solution}
    No. Let $x_k = (k\pi+\pi/2)^{-1/2}$, $k=0,1,\dots$ and consider the 
    subdivision $-1<0<x_n<\cdots<x_0<1$. Then
    \[
      t_n \ge \sum_{k=1}^n|f(x_k)-f(x_{k-1})|
      \ge\sum_{k=1}^n\frac{2}{(k+1/2)\pi}.
    \]
    $t_n\to\infty$ as $n\to\infty$ and therefore $f$ is not of bounded 
    variation on $[-1,1]$.
  \end{solution}

  \paragraph{11.}
  \begin{proof}
    By Lemma 4, $f(x)=f(a)+P_a^x-N_a^x$. Since $P_a^x$ and $N_a^x$ are monotone,
    by Theorem 3, they are differentiable almost everywhere as $f$, a function 
    of bounded variation, does. Hence, for almost every $x\in[a,b]$,
    \[
      \frac{\rd}{\rd x}f(x) = \frac{\rd}{\rd x}P_a^x - \frac{\rd}{\rd x}N_a^x
      \quad\Rightarrow\quad
      |f\hp(x)| \le \frac{\rd}{\rd x}P_a^x+\frac{\rd}{\rd x}N_a^x = 
      \frac{\rd}{\rd x}T_a^x.
    \]
    Integrating on the both sides yields $\int_a^b|f\hp|\le T_a^b(f)$.
  \end{proof}

% end

