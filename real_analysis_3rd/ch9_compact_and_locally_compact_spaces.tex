\section{Compact and Locally Compact Spaces}
\subsection{Compact Spaces}

\paragraph{2.}
  I further assume that $X$ is Hausdorff. 
\begin{proof}
  Assume, to obtain a contradiction, that $F_n = K_n\setminus O \ne\varnothing$
  for every $n$. Since $F_n \supset F_{n+1}$, $\{F_n\}$ is collection of closed
  subsets of compact set $K_1$ with finite intersection property. Hence, 
  $\bigcap F_n$ is nonempty, contradicting with $\bigcap K_n \subset O$. 
\end{proof}

\paragraph{3.}
\begin{proof}
  Let $F$ be a closed set and $x \notin F$. Since $X$ is Hausdorff, for every
  $y \in F$, there are two disjoint open sets $U_y$ and $O_y$ s.t. $y \in U_y$
  and $x \in O_y$. Since $X$ is compact, so is $F$. Note that $\{U_y\}$ is an
  open cover for $F$. Hence, it has a finite subcover $\{U_{y_i}\}_{i=1}^n$.
  Let $U = \bigcup U_{y_i}$ and $O = \bigcap O_{y_i}$. Clear that they are
  disjoint open sets s.t. $F \subset U$ and $x \in O$.
\end{proof}

\paragraph{6.}
\begin{proof}
  Let $\vep > 0$ be fixed. Since $\mcal{F}$ is equicontinuous, for every 
  $x \in X$, there is a neighborhood $O_x$ s.t. for every $x\hp \in X$,
  $\sigma(f(x), f(x\hp)) < \vep$. Clear that $\{O_x\}$ is an open cover and
  since $X$ is compact, it has a finite subcover $\{O_{x_i}\}_{i=1}^m$. 
  
  For $x_i$, since $f_n(x_i) \to f(x_i)$, there is an integer $N_i$ s.t. for  
  every $n > N$, $\sigma(f_n(x_i), f(x_i)) < \vep$. Since $\sigma(f_n(x_i), 
  f_n(x)) < \vep$ holds for all $n$, $\sigma(f(x_i), f(x)) \le \vep$.
  Hence, for every $x \in O_{x_i}$ and $n > N$,
  \[
    \sigma(f_n(x), f(x))
    \le \sigma(f_n(x), f_n(x_i)) + \sigma(f_n(x_i), f(x_i)) 
      + \sigma(f(x_i), f(x))
    < 3\vep.
  \]
  Let $N = \max N_{x_i}$ and we get the desired result.
\end{proof}

\subsection{Countable Compactness and the Bolzano-Weierstrass Property}
\paragraph{9.}
\begin{proof}
  $\,$\par
  (a) It follows immediately from the definition and Problem 8.20.
  
  (b) For every $\alpha \in \mathbb{R}$,
  \[
    f + g < \alpha
    \quad\text{iff}\quad
    f < \alpha - g 
    \quad\text{iff}\quad 
    \exists q\in \mathbb{Q} \text{ s.t. } f < q,\, q < \alpha - g.
  \]
  Hence, 
  \[
    \{f + g < \alpha\} 
    = \bigcup_{q \in \mathbb{Q}} \{f < q\} \cap \{g < \alpha - q\},
  \]
  which is open. Thus, $f + g$ is also upper semicontinuous. 
  
  (c) Since $(f_n)$ is a decreasing sequence, we can write $f(x) = \inf_n
  f_n(x)$. Hence, for every $\alpha \in \mathbb{R}$, $f < \alpha$ iff there
  exists some $n$ s.t. $f_n(x) < \alpha$. Hence,
  \[
    \{f < \alpha\} = \bigcup_n \{f_n < \alpha\},
  \]
  which is open. Thus, $f$ is also upper semicontinuous. 
  
  (d) Note that $(f_n - f)$ is a decreasing sequence of upper semicontinuous
  functions that converges to $0$. Hence, by Dini's theorem, the convergence is
  uniform.
  
  (e) Suppose that $x \in \{f < \alpha\}$. Let $\vep$ be a positive real
  number. Since $f_n \to f$ uniformly, there is an integer $n$ s.t. 
  $|f(y) - f_n(y)| < \vep$ for all $y \in X$. Meanwhile, since $f_n$ is upper
  semicontinuous, there is a $\delta > 0$ s.t. for every $y$ in the
  $\delta$-ball $B$ centered at $x$, $f_n(y) < f_n(x) + \vep$. Hence, for every
  $y \in B$,
  \begin{align*}
    f(y) = f(y) - f_n(y) + f_n(y) - f_n(x) + f_n(x) - f(x) + f(x) 
    \le 3\vep + f(x). 
  \end{align*}
  Thus, for sufficiently small $\vep > 0$, we have $B \subset \{f < \alpha\}$.
  Namely, $\{f < \alpha\}$ is open whence $f$ is upper semicontinuous. 
\end{proof}

\paragraph{10.}
\begin{proof}
  $\,$\par
  (i. $\Rightarrow$ iii.) Let $f$ be a bounded continuous real-valued function
  and $M := \sup f < \infty$. Let $F_n = \{f \ge M - 1/n\}$. Since $f$ is
  continuous, $F_n$ is closed. Note that $(F_n)$ is a countable family of
  closed sets with finite intersection property. Hence, $\bigcap F_n =
  \{f \ge M\}$ is nonempty as $X$ is countably compact. Namely, the maximum
  can be attained. 
  
  (iii. $\Rightarrow$ ii.) Let $f$ be a continuous function and assume, to
  obtain a contradiction, that $f$ is unbounded. Then the function 
  $-1/(|f| + 1)$ is a continuous bounded function whose maximum can not be 
  attained. Contradiction.
  
  (ii. $\Rightarrow$ i.) Assume, to obtain a contradiction, that $X$ does not
  have the Bolzano-Weierstrass property, that is, there is a sequence $(x_n)$
  in $X$ that has no cluster point. Then $F := \{x_n\}_{n=1}^\infty$ is closed.
  Define $f: F \to \mathbb{R}$ by $f(x_n) = n$. Note that $f$ is continuous on
  $F$ and by Tietze's extension theorem, it can be continuously extended to 
  $X$. However, $f$ is unbounded, contradicting (ii.). Thus $X$ has the
  Bolzano-Weierstrass and, therefore, is countably compact. 
\end{proof}























