\section{Compact and Locally Compact Spaces}
\subsection{Compact Spaces}

\paragraph{2.}
  I further assume that $X$ is Hausdorff. 
\begin{proof}
  Assume, to obtain a contradiction, that $F_n = K_n\setminus O \ne\varnothing$
  for every $n$. Since $F_n \supset F_{n+1}$, $\{F_n\}$ is collection of closed
  subsets of compact set $K_1$ with finite intersection property. Hence, 
  $\bigcap F_n$ is nonempty, contradicting with $\bigcap K_n \subset O$. 
\end{proof}

\paragraph{3.}
\begin{proof}
  Let $F$ be a closed set and $x \notin F$. Since $X$ is Hausdorff, for every
  $y \in F$, there are two disjoint open sets $U_y$ and $O_y$ s.t. $y \in U_y$
  and $x \in O_y$. Since $X$ is compact, so is $F$. Note that $\{U_y\}$ is an
  open cover for $F$. Hence, it has a finite subcover $\{U_{y_i}\}_{i=1}^n$.
  Let $U = \bigcup U_{y_i}$ and $O = \bigcap O_{y_i}$. Clear that they are
  disjoint open sets s.t. $F \subset U$ and $x \in O$.
\end{proof}

\paragraph{6.}
\begin{proof}
  Let $\vep > 0$ be fixed. Since $\mcal{F}$ is equicontinuous, for every 
  $x \in X$, there is a neighborhood $O_x$ s.t. for every $x\hp \in X$,
  $\sigma(f(x), f(x\hp)) < \vep$. Clear that $\{O_x\}$ is an open cover and
  since $X$ is compact, it has a finite subcover $\{O_{x_i}\}_{i=1}^m$. 
  
  For $x_i$, since $f_n(x_i) \to f(x_i)$, there is an integer $N_i$ s.t. for  
  every $n > N$, $\sigma(f_n(x_i), f(x_i)) < \vep$. Since $\sigma(f_n(x_i), 
  f_n(x)) < \vep$ holds for all $n$, $\sigma(f(x_i), f(x)) \le \vep$.
  Hence, for every $x \in O_{x_i}$ and $n > N$,
  \[
    \sigma(f_n(x), f(x))
    \le \sigma(f_n(x), f_n(x_i)) + \sigma(f_n(x_i), f(x_i)) 
      + \sigma(f(x_i), f(x))
    < 3\vep.
  \]
  Let $N = \max N_{x_i}$ and we get the desired result.
\end{proof}

\subsection{Countable Compactness and the Bolzano-Weierstrass Property}
\paragraph{9.}
\begin{proof}
  $\,$\par
  (a) It follows immediately from the definition and Problem 8.20.
  
  (b) For every $\alpha \in \mathbb{R}$,
  \[
    f + g < \alpha
    \quad\text{iff}\quad
    f < \alpha - g 
    \quad\text{iff}\quad 
    \exists q\in \mathbb{Q} \text{ s.t. } f < q,\, q < \alpha - g.
  \]
  Hence, 
  \[
    \{f + g < \alpha\} 
    = \bigcup_{q \in \mathbb{Q}} \{f < q\} \cap \{g < \alpha - q\},
  \]
  which is open. Thus, $f + g$ is also upper semicontinuous. 
  
  (c) Since $(f_n)$ is a decreasing sequence, we can write $f(x) = \inf_n
  f_n(x)$. Hence, for every $\alpha \in \mathbb{R}$, $f < \alpha$ iff there
  exists some $n$ s.t. $f_n(x) < \alpha$. Hence,
  \[
    \{f < \alpha\} = \bigcup_n \{f_n < \alpha\},
  \]
  which is open. Thus, $f$ is also upper semicontinuous. 
  
  (d) Note that $(f_n - f)$ is a decreasing sequence of upper semicontinuous
  functions that converges to $0$. Hence, by Dini's theorem, the convergence is
  uniform.
  
  (e) Suppose that $x \in \{f < \alpha\}$. Let $\vep$ be a positive real
  number. Since $f_n \to f$ uniformly, there is an integer $n$ s.t. 
  $|f(y) - f_n(y)| < \vep$ for all $y \in X$. Meanwhile, since $f_n$ is upper
  semicontinuous, there is a $\delta > 0$ s.t. for every $y$ in the
  $\delta$-ball $B$ centered at $x$, $f_n(y) < f_n(x) + \vep$. Hence, for every
  $y \in B$,
  \begin{align*}
    f(y) = f(y) - f_n(y) + f_n(y) - f_n(x) + f_n(x) - f(x) + f(x) 
    \le 3\vep + f(x). 
  \end{align*}
  Thus, for sufficiently small $\vep > 0$, we have $B \subset \{f < \alpha\}$.
  Namely, $\{f < \alpha\}$ is open whence $f$ is upper semicontinuous. 
\end{proof}

\paragraph{10.}
\begin{proof}
  $\,$\par
  (i. $\Rightarrow$ iii.) Let $f$ be a bounded continuous real-valued function
  and $M := \sup f < \infty$. Let $F_n = \{f \ge M - 1/n\}$. Since $f$ is
  continuous, $F_n$ is closed. Note that $(F_n)$ is a countable family of
  closed sets with finite intersection property. Hence, $\bigcap F_n =
  \{f \ge M\}$ is nonempty as $X$ is countably compact. Namely, the maximum
  can be attained. 
  
  (iii. $\Rightarrow$ ii.) Let $f$ be a continuous function and assume, to
  obtain a contradiction, that $f$ is unbounded. Then the function 
  $-1/(|f| + 1)$ is a continuous bounded function whose maximum can not be 
  attained. Contradiction.
  
  (ii. $\Rightarrow$ i.) Assume, to obtain a contradiction, that $X$ does not
  have the Bolzano-Weierstrass property, that is, there is a sequence $(x_n)$
  in $X$ that has no cluster point. Then $F := \{x_n\}_{n=1}^\infty$ is closed.
  Define $f: F \to \mathbb{R}$ by $f(x_n) = n$. Note that $f$ is continuous on
  $F$ and by Tietze's extension theorem, it can be continuously extended to 
  $X$. However, $f$ is unbounded, contradicting (ii.). Thus $X$ has the
  Bolzano-Weierstrass and, therefore, is countably compact. 
\end{proof}

\subsection{Products of Compact Spaces}
\paragraph{13.}
\begin{proof}
  Let $E$ be a closed and bounded set in $\mathbb{R}^n$. Then it is contained
  in some closed cube $K = \prod_{i=1}^n[a_i, b_i]$. By Tychonoff's theorem,
  $K$ is compact. Thus, $K$, a closed subset of a compact set, is also compact.
\end{proof}

\paragraph{15.}
\begin{proof}
  Let $X = \prod_{n=1}^\infty$ be the product of sequentially compact spaces
  $(X_n)$. Let $(x_n)$ be a sequence in $X$. Since $X_1$ is sequentially
  compact, we may choose a subsequence $x_n^1$ of $x_n$ s.t. the first
  coordinate of $(x_n^1)$ converges to some $x^1$. Similarly, from $(x_n^1)$ we
  may choose a subsequence $x_n^2$ whose second coordinate converges to some
  $x^2$. Proceed inductively and we get a sequence of sequence. Finally,
  consider the sequence $(x_n^n)$. Since each coordinate converges and we are
  dealing with the product topology, $x_n^n$ converges to $(x_1, x_2, \dots)$.
\end{proof}


\subsection{Locally Compact Spaces}
\paragraph{18.}
\begin{proof}
  For every $x \in K$, since $X$ is locally compact, there is an open set 
  $O_x$ with $\cl O_x$ compact. Note that $\{O_x\}_{x\in K}$ is an open cover
  for the compact set $K$. Hence, it has a finite subcover
  $\{O_{x_i}\}_{i=1}^n$. Then, $O = \bigcup_{i=1}^n O_{x_i}$ is an open set
  containing $K$ whose closure is compact. 
\end{proof}

\paragraph{19.}
\begin{proof}
  $\,$\par
  (a) Since $X$ is a locally compact Hausdorff space and $K$ is compact, there
  exists an open set $V$ with compact closure s.t. $V\supset K$. Since 
  $\cl V$ is compact, it is normal. Therefore, by Urysohn's lemma, there is
  a continuous function $f: \cl V \to [0, 1]$ s.t. $f\equiv 1$ on $K$ and
  $f\equiv 0$ on $\partial V$. Extend $f$ to $X$ by setting $f \equiv 0$
  outsides $\cl V$. Then $f$ is continuous and $f \equiv 1$ on $K$. Meanwhile,
  since $\supp f\subset \cl V$, it is also compact.
\end{proof}

\paragraph{24.}
  Assume that $X$ is also Hausdorff.
\begin{proof}
  $\,$\par
  (a) Clear that if $F$ is closed, so is $F\cap K$ for each closed compact
  $K$. For the reverse, we show that $F^c$ is open.
  Let $x \notin F$. Since $X$ is locally compact, there is a neighborhood $U$
  of $x$ whose closure is compact. If $F\cap\cl U = \varnothing$, then we are
  done. If $F\cap \cl U \ne \varnothing$, then by the hypothesis, it is closed.
  Therefore, $U\setminus(F\cap\cl U)$ is again an open neighborhood of $x$.
  Since $X$ is a locally compact Hausdorff space, we can find an open
  neighborhood $V$ with $\cl V \subset U\setminus(F\cap\cl U)$. In both cases,
  $F^c$ is open. 
  
  (b) Suppose that for each closed compact $K$, $F\cap K$ is closed. For every
  $x \in \cl F$, since $X$ is first-countable, there exists a sequence $(x_n)
  \subset F$ which converges to $x$. Then $E := \{x\}\cup\{x_n\}_{n=1}^\infty$
  is closed and compact. Thus, by the hypothesis, $F\cap E$ is also closed,
  which implies that $x \in F$. Hence, $F$ is closed. 
\end{proof}

\paragraph{26.}
  Assume that $X$ is Hausdorff.
\begin{proof}
  Let $x$ be an arbitrary point in $X$ and $V$ any neighborhood of $x$. Since 
  $X$ is a locally compact Hausdorff space, we may choose a open neighborhood
  $U_1$ of $x$ whose closure is compact and contained by $V$. Since $O_1$ is
  dense, $U_1\cap O_1$ is a nonempty neighborhood of $x$. Then, choose a 
  neighborhood $U_2$ of $x$ s.t. $\cl U$ is compact and $\cl U \subset 
  U_1\cap O_1$. Proceed inductively and we get a sequence $(U_n)$ s.t.
  $\cl U_n$ is compact and $\cl U_{n+1} \subset O_n\cap U_n$.
  
  Since $(\cl U_n)$ is a nested sequence of compact sets, $\bigcap \cl U_n$
  is nonempty. Choose $x_* \in \bigcap\cl U_n$. For every $O_n$, $x_*
  \in \cl U_{n+1} \subset O_n$. Thus, $x_* \in V\cap\bigcap_{n=1}^\infty O_n$.
  Namely, $\bigcap_{n=1}^\infty O_n$ is dense.
\end{proof}

\paragraph{29.}
\begin{proof}
  $\,$\par
  (a) Let $F$ be a closed subset of a locally compact space $X$. For every 
  $x\in F\subset X$, there is a neighborhood $U$ of $x$ whose closure is 
  compact in $X$. Then, $U\cap F$ is also compact. Thus, $F$ is locally
  compact.
  
  (b) Let $O$ be an open subset of a locally compact Hausdorff space $X$. For
  every $x\in O \subset X$, there is a neighborhood $U$ of $x$ whose closure is
  compact in $X$ and contained by $O$. Note that $\cl U$ is also compact in
  $O$. Thus, $O$ is locally compact.
\end{proof}
















