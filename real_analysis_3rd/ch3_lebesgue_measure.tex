\section{Lebesgue Measure}
\subsection{Introduction}
  \paragraph{1.}
  \begin{proof}
    Since $\mathfrak{M}$ is an $\sigma$-algebra, $B\setminus A \in\mathfrak{M}$
    as long as $A,B\in\mathfrak{M}$. Since $B\setminus A$ and $A$ are disjoint,
    $mB=mA+m(B\setminus A)\ge mA$ since $m$ is nonnegative.
  \end{proof}

  \paragraph{2.}
  \begin{proof}
    Let $A_0 = E_0$ and $E_k=A_k\setminus A_{k-1}$ for $k\ge 1$. Clear that 
    $E_i$ and $E_j$ are disjoint for distinct $i$ and $j$, $\bigcup A_n=\bigcup
    E_n$ and $A_i\subset E_i$ for every $i$. Hence,
    \[
      m\left(\bigcup E_n\right) = m\left(\bigcup A_n\right)
      = \sum mA_n \le \sum mE_n,
    \]
    where the last inequality comes from Exercise 1.
  \end{proof}

  \paragraph{3.}
  \begin{proof}
    Suppose that $mA<\infty$. Then $mA=m(A\cup\varnothing)=mA+m\varnothing$, 
    implying that $m\varnothing=0$.
  \end{proof}
% end

\subsection{Outer Measure}
  \paragraph{5.}
  \begin{proof}
    We show that $\{I_n\}$ must cover the entire $[0,1]$ by contradiction. 
    Assume that $x\notin I_k$ for $k=1,2,\dots,n$. Then, as $I_k$ are open and 
    $n$ is finite, there exists some $\vep>0$ such that $(x-\vep,x+\vep)$ and 
    $I_k$ are disjoint for every $k$. Since $\mathbb{Q}$ is dense in 
    $\mathbb{R}$, there exists some rational number in $(x-\vep, x+\vep)$, 
    contradicting with the hypothesis that $\{I_k\}$ covers all rational numbers
    between $0$ and $1$.
  \end{proof}

  \paragraph{6.}
  \begin{proof}
    By the definition of the outer measure, for every $\vep > 0$, there exists 
    some collection $\{I_n\}$ of open intervals that covers $A$ and $\sum l(I_n)
    \le m^*A+\vep$. Let $O=\bigcup I_n$. $O$ is a countable union of open sets 
    and therefore is also open. And by Proposition 2, $m^*O\le \sum l(I_n)$. 
    Thus, $m^*O\le m^*A+\vep$. \par
    Let $\vep_n = 1/n$ and for each $n$, by the previous discussion, we can 
    always get an open set $O_k$ such that $A\subset O_k$ and $m^*O\le m^*A+
    \vep_m$. Let $G$ be the countable intersection of these open sets. Clear 
    that $G$ is a $G_\delta$ set covering $A$ and $m^*A=m^*G$.
  \end{proof}

  \paragraph{7.}
  \begin{proof}
    If $m^*E=\infty$, it is trivial. Suppose that $m^*E\le\infty$. For any $x\in
    \mathbb{R}$, collection $\{I_n\}$ of open intervals covers $E+x$ iff $\{I_n
    -x\}$ covers $E$. Since the length of intervals is translation invariant, 
    this implies $m^*(E+x)=m^*E$.
  \end{proof}

  \paragraph{8.}
  \begin{proof}
    Clear that $m^*A\le m^*(A\cup B)$. Meanwhile, $m^*(A\cup B) = m^*A + m^*B =
    m^*B$. Hence, $m^*(A\cup B)=m^*B$.
  \end{proof}
% end

\subsection{Measurable Sets and Lebesgue Measure}
  \paragraph{10.}
  \begin{proof}
    \begin{align*}
      mE_1+mE_2 
      &= mE_1 + m(E_2\setminus E_1) + m(E_1\cap E_2) \\
      &= m(E_1\cup(E_2\setminus E_1)) + m(E_1\cap E_2) \\
      &= m(E_1\cup E_2) + m(E_1\cap E_2).
    \end{align*}
  \end{proof}
    
  \paragraph{11.}
  \begin{proof}
    $E_n = (n,\infty)$.
  \end{proof}

  \paragraph{12.}
    This is the countable version of Lemma 9.
  \begin{proof}
    It suffices to prove $m^*(A\cap \bigcup E_i) \ge \sum m^*(A\cap E_i)$. Since
    $\bigcup_{i=1}^\infty E_i\supset \bigcup_{i=1}^n E_i$ for every $n$, 
    \begin{align*}
      m^*\left(A\cap \bigcup_{i=1}^\infty E_i\right)
      \ge m^*\left(A\cap \bigcup_{i=1}^n E_i\right)
      = \sum_{i=1}^n m^*(A\cap E_i),
    \end{align*}
    where the equality comes from Lemma 9. Since the left hand side is 
    independent of $n$, we have
    \[
      m^*\left(A\cap \bigcup_{i=1}^\infty E_i\right) \ge
      \sum_{i=1}^\infty m^*(A\cap E_i),
    \]
    completing the proof.
  \end{proof}

  \paragraph{13.}
  \begin{proof}
    First we suppose that $m^*E<\infty$. By Proposition 5, there exists some 
    open set $O\supset E$ such that $m^*O\le m^*E+\vep$. If $E$ is measurable,
    then by the definition,
    \[
      m^*(O\setminus E) = m^*O-m^*E \le \vep.
    \]
    Namely, (ii) holds. Meanwhile, $O\subset\mathbb{R}$ is a countable union of 
    disjoint open intervals $\{I_n\}$. Since $mO=m^*O$ is bounded and $mO=\sum
    l(I_n)$, there exists some integer $N>0$ such that $mO-\sum_{n=1}^N l(I_n)<
    \vep$. Let $U=\bigcup_{n=1}^N I_n$.
    \begin{align*}
      m^*(U\bigtriangleup E) 
      &= m^*((U\cup E) \setminus (U\cap E)) \\
      & \le m^*(O\setminus (U\cap E))  \\
      & = m^*((O\setminus U) \cup (O\setminus E)) \\
      &\le m^*(O\setminus U) + m^*(O\setminus E)  \\
      &\le 2\vep.
    \end{align*}
    Hence, (ii) implies (vi). Now we show that (vi) implies (ii). If $m^*(U
    \bigtriangleup E)<\vep $, then there exists some countable collection $\{J_n
    \}$ of open interval such that 
    \[
      \sum l(J_n)\le m^*(U\bigtriangleup E)+\vep<2\vep.
    \]
    Let $J=\bigcup J_n$ and $O= U\cup J$. $m^*J < 2\vep$. And $O$ is open and 
    covers $E$. Meanwhile,
    \[
      m^*(O\setminus E) \le m^*(U\setminus E)+m^*(J\setminus E) < 3\vep.
    \]
    Hence, (ii) holds.\par
    Now, let $E$ be an arbitrary set and $E_n=E\cap(-n, n)$, which is a set with
    finite measure. Then by the previous discussion, there exists some open set
    $O_n\supset E_n$ with $m^*(O_n\setminus E_n)<\vep/2^n$. Let $O=\bigcup 
    O_n$, an open set covering $E$ and
    \[
      m^*(O\setminus E) \le \sum m^*(O_n\setminus E_n) < 2\vep. 
    \]
    Hence, (i) implies (ii). Now we suppose (ii) holds and let $\vep_n=1/n$, 
    then there exists a sequence of open sets $<O_n>$ such that $m^*(O_n
    \setminus E)<1/n$. Let $G=\bigcap O_n\in G_\delta$. $m^*(G\setminus E)\le
    m^*(O_n\setminus E)\le 1/n$. Since the left hand side is independent of 
    $n$, $m^*(G\setminus E)=0$. If (iv) holds, then by Lemma 6, $G\setminus E$
    is measurable. Since $G\in G_\delta$ is also measurable, $E$ is measurable.
    Hence, (iv) implies (i).\par
    By the previous result, for any measurable $E$, there exists some closed set
    $F\subset E$ such that $\bar{F}$, which is open, contains $bar{E}$ and $m^*(
    \bar{F}\setminus \bar{E})<\vep$. Hence, $m^*(E\setminus F)<\vep$. We can 
    proceed in a similar manner as we did in the last paragraph to prove that
    (iii) $\Rightarrow$ (v) $\Rightarrow$ (i), leading to the final conclusion.
  \end{proof}

% end
