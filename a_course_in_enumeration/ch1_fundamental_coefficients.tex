\section{Fundamental Coefficients}
\subsection{Elementary Counting Principles}
  \paragraph{2.}
  \begin{solution}
    We compute $N=\#\{(i,j,k)\in\mathbb{N}^3\,|\,i+j+k=151,\max\{i,j,k\}\le 
    75\}$. For fixed $1\le i\le 75$, $j$ can be chosen between $76-i$ and $75$.
    Thus,
    \[
      N=\sum_{i=1}^{75}\sum_{j=76-i}^{75}1=\sum_{i=1}^{75}i=2850.
    \]
  \end{solution}
  
  \paragraph{3.}
  \begin{proof}
    The number of subsets of $\{1,\dots,n+1\}$ is $2^{n+1}$. Classify these 
    subsets according the biggest elements in them. The number of subsets whose
    biggest elements are $k$ equals to the number of subsets of $\{1,\dots,k\}$
    containing $k$, that is, $2^{k-1}$. Thus,
    \[
      2^{n+1}=1+\sum_{k=1}^{n+1}2^{k-1}
      \quad\Rightarrow\quad
      2^{n+1}-1=\sum_{k=0}^n 2^k.
    \]\par
    Similarly, we may classify these subsets according to the biggest two 
    elements. Then
    \[
      2^{n+1}-1-(n+1)=\sum_{i=1}^n\sum_{j=i+1}^{n+1}2^{i-1}=
      \sum_{i=1}^n2^{i-1}(n-i+1)=\sum_{i=1}^n2^{i-1}(n-i)+2^n-1.
    \]
    Thus, $\sum_{k=1}^n(n-k)2^{k-1}=2^n-n-1$.
  \end{proof}

  \paragraph{5.}
  \begin{proof}
    We count the number $N$ of triples in $\{1,\dots,n+1\}$. By definition, 
    $N=\binom{n+1}{3}$. Let $S_k$ be the collection of triples the last
    elements of which are $k$. Then $|S_k|=\binom{k-1}{2}$. Thus
    \[
      \binom{n+1}{3}=\sum_{k=3}^{n+1}\binom{k-1}{2}=\sum_{k=2}^n\binom{k}{2}
      =\sum_{k=1}^n\binom{k}{2}.
    \]
  \end{proof}
  
  \paragraph{8.}
  \begin{proof}
    Let $E$ be valid $k$-subset. If $1\in  E$, then we need to choose $k-1$
    numbers from $3,\dots,n$ so that no pair of consecutive integers exists. 
    The number of possible choices is $f(n-2,k-1)$. If $1\notin E$, then we 
    need to choose $k$ numbers from $2,\dots,n$ such that no pair of 
    consecutive integers exists. The number of possible choices is $f(n-1,k)$.
    Hence, we obtain the recurrence relation
    \[
      f(n,k)=f(n-2,k-1)+f(n-1,k).
    \]\par
    Now we argue by induction on $n$. Clear that $f(n,0)=1$ for all $n$, 
    $f(2,1)=2$ and $f(n,k)=0$ for all $n<2k-1$, all satisfying $f(n,k)=
    \binom{n-k+1}{k}$. Assume that for all $m<n$, $f(m,k)=\binom{m-k+1}{k}$ 
    holds. Thus,
    \[
      f(n,k)=f(n-2,k-1)+f(n-1,k)=
      \binom{n-k}{k-1}+\binom{n-k}{k}=\binom{n-k+1}{k}.
    \]
    Let $s(n)$ denote $\sum_{k=0}^nf(n,k)$. Then
    \begin{align*}
      s(n-1)+s(n-2)
      &=f(n-1,0)+\sum_{k=1}^{n-1}f(n-1,k)+\sum_{k=1}^{n-1}f(n-2,k-1) \\
      &=1+\sum_{k=1}^{n-1}\{f(n-1,k)+f(n-2,k-1)\} \\
      &= f(n,0)+\sum_{k=1}^{n-1}f(n,k)=\sum_{k=0}^{n-1}f(n,k)=s(n).
    \end{align*}
    This recurrence relation, together with the fact $s(1)=2$ and $s(2)=3$, 
    imply that $s(n)=F_{n+2}$.
  \end{proof}
  
  \paragraph{11.}
  \begin{proof}
    We argue by contradiction. Let $S_p=\{p+9k\,|\,p+9k\le 100, p=0,1,\dots\}$, 
    $p=1,\dots,9$. Clear that $\{S_p\}$ partitions $\{1,\dots,100\}$; $|S_1|
    =12$ and $|S_p|=11$ for $p=2,\dots,9$. If $A$ does not contain two numbers 
    with difference $9$. Then for each $p$, no consecutive elements of $S_p$ 
    can belong to $A$, which implies that $|S_p\cap A|\le 6$. Hence, $|A|=
    \sum_{p=1}^9|S_p\cap A|\le 54$. Contradiction. Thus, $A$ must contain two
    numbers with difference $9$.\par
    For the case $|A|=54$, this is not true. For a counterexample, put
    \[
      A=\bigcup_{p=1}^9\{p+9k\,|\,p+9k\le 100, k=1,3,5,\dots\}.
    \]
  \end{proof}
% end
\subsection{Subsets and Binomial Coefficients}
  \paragraph{12.}
  \begin{proof}
    We count the pairs $(A,B)$ of subsets of $N$ with $|A|=k$ and $|B|=m-k$ and
    $A\cap B=\varnothing$. We may choose $A\cup B$ first and, then, the 
    elements of $A$ from $A\cup B$. This way yields the left-hand side. Or,
    we may choose the elements of $A$ from $N$ first and then the elements of
    $B$ from $N\setminus A$, which yields the right-hand side. Thus, $\binom{n}
    {m}\binom{m}{k}=\binom{n}{k}\binom{n-k}{m-k}$.\par
    If we let $|A|=k$ range over $0$ to $m$, then we count all subsets of $A
    \cup B$. Thus, $\sum_{k=0}^m\binom{n}{k}\binom{n-k}{m-k}=2^m\binom{n}{m}$.
  \end{proof}
  
  \paragraph{13.}
  \begin{proof}
    \begin{align*}
      \lhs
      &=\left\{\binom{2n}{2n-2k}\binom{2n-2k}{n-k}\right\}\binom{2k}{k}\\
      &=\binom{2n}{n-k}\binom{n+k}{n-k}\binom{2k}{k}\\
      &=\binom{2n}{n+k}\left\{\binom{n+k}{2k}\binom{2k}{k}\right\}\\
      &=\binom{2n}{n+k}\binom{n+k}{k}\binom{n}{k}\\
      &=\left\{\binom{2n}{n+k}\binom{n+k}{n}\right\}\binom{n}{k}\\
      &=\binom{2n}{n}\binom{n}{k}^2 =\rhs.
    \end{align*}
  \end{proof}
  
  \paragraph{17.}
  \begin{proof}
    If there exists $i$ and $j$ such that $n_i-n_j\ge 2$, then we may increase
    the coefficient by replace $n_i$ with $n_i-1$ and $n_j$ with $n_j+1$. Since
    the choices of $n_1,\dots,n_k$ are finite, this implies that the 
    coefficient must attain its maximum at some $n_1,\dots,n_k$ with $|n_i-n_j|
    \le 1$ for all $i$ and $j$. Meanwhile, since $\max n_i-\min n_i\le 1$ and
    $\sum n_i=n$, a fixed number, the choice of $n_1,\dots,n_k$ is unique up to
    permutation. Thus, the coefficient attains its maximum at every $n_1,\dots,
    n_k$ with $|n_i-n_j|\le 1$.\par
    We show that $\binom{n}{n_1n_2n_3}\le\frac{3^n}{n+1}$ by induction on $n$.
    Some computation shows that the inequality holds for $n=1$. Assume that 
    the inequality holds for all cases less than $n$. And suppose that the
    coefficient attains its maximum at $m_1\le m_2\le m_3$. Then
    \[
      \binom{n}{n_1n_2n_3}\le
      \frac{n!}{m_1!m_2!m_3!}=
      \frac{(n-1)!}{m_1!m_2!(m_3-1)!}\frac{n}{m_3}\le
      \frac{3^{n-1}}{n}\frac{n}{(n+1)/3}
    \]
  \end{proof}
  
  \paragraph{18.}
  \begin{proof}
    Consider the $(k+1)\times(n-k)$-lattice. The number of paths is $\binom{n+
    1}{k+1}$. Classify the paths according to the height $j$ right before the
    last $(1,0)$-step. For each $j=0,\dots,n-k$, the number equals the number
    of paths of the $k\times j$-lattice, which is $\binom{k+j}{k}$. Thus,
    \[
      \binom{n+1}{k+1}=\sum_{j=0}^{n-k}\binom{k+j}{k}
      =\sum_{i=k}^n\binom{i}{k}=\sum_{i=0}^n\binom{i}{k}.
    \]
    This argument, \textit{mutatis mutandis}, also gives (9).
  \end{proof}
% end
















