\section{Convex Sets}

\subsection{Definition of convexity}
  \paragraph{1.}
  \begin{proof}
    For $k=2$, $\theta_1x_1+\theta_2x_2\in C$ holds by by definition. We argue
    by induction on $k$ and assume that the inclusion holds for $k<m$. When $k=
    m$, denoting $\sum_{i=1}^{m-1}\theta_i$ by $s$,
    \[
      \sum_{i=1}^m \theta_ix_i = 
      s\sum_{i=1}^{m-1}\frac{\theta_ix_i}{s} + \theta_mx_m.
    \]
    Since $\sum_{i=1}^{m-1}\theta_i/s=1$, by the induction hypothesis,
    $\sum_{i=1}^{m-1}\theta_ix_i/s\in C$ . Meanwhile, as $s+\theta_m=1$, 
    $\sum_{i=1}^m\theta_ix_i\in C$, completing the proof.
  \end{proof}

  \paragraph{2.}
  \begin{proof}
    Clear that the intersection of two convex sets is still convex. Hence, the 
    intersection of $C\subset\mathbb{R}^n$ and any line is convex as long as $C$
    is convex.\par
    Now we suppose that the intersection of $C$ and any line is convex. For any 
    $x_1,x_2\in C$, $C_l=C\cap\{\theta x_1+(1-\theta)x_2:\,\theta\in\mathbb{R}
    \}$ is convex and therefore $\theta x_1+(1-\theta)x_2\in C_l\subset C$ for
    every $0\le\theta\le 1$. Thus, $C$ is convex.\par
    The above argument, \textit{mutatis mutandis}, gives the second result.
  \end{proof}

  \paragraph{3.}
  \begin{proof}
    For every $\theta\in[0,1]$, the process of bisecting the interval implies 
    there exists a series $\langle\delta_n\rangle$ whose sum is $\theta$. Hence,
    for every $a,b\in C$, $x_n = a+(b-a)\sum_{n=1}^\infty\delta_n$ converges to
    $a+\theta(b-a)$. Meanwhile, the midpoint convexity implies $x_n\in C$ for 
    every $n$. And since $C$ is closed, $a+\theta(b-a)\in C$. Thus, $C$ is 
    convex.
  \end{proof}

  \paragraph{4.}
  \begin{proof}
    Let $D$ be the intersection of all convex sets containing $C$. If $x\in C$,
    then its is a convex combination of some points in $C$. Hence, for every 
    convex set containing $C$, it contains $x$. Therefore, $\conv C\subset D$.
    For the converse, since $\conv C$ itself is a convex set containing $C$, $D
    \subset\conv C$. Thus, $\conv C=D$.
  \end{proof}

% end

\subsection{Examples}
  \paragraph{5.}
  \begin{solution}
    $|b_2-b_1|/\|a\|_2$.
  \end{solution}

  \paragraph{7.}
  \begin{proof}
    $\|x-a\|_2\le\|x-b\|_2$ iff $\langle x-a, x-a\rangle \le \langle x-b, x-b
    \rangle$ iff $2\langle x,b-a\rangle \le \langle b,b\rangle-\langle a,a
    \rangle$. Namely, $2(b-a)^Tx \le \|b\|_2^2-\|a\|_2^2$.
  \end{proof}

  \paragraph{2.8}
  \begin{proof}
    $\,$\par
    (a) It is trivial when $a_1$ and $a_2$ are linearly dependent, so we assume 
    that $a_1$ and $a_2$ are linearly independent. We first tackle the problem
    for orthonormal $a_1$ and $a_2$ and then reduce the general situation to it.
    \par
    Suppose that $a_1$ and $a_2$ are orthonormal. Let $S_0=\spn(a_1,a_2)$ and 
    $(b_1,\dots,b_{n-2})$ a basis of $S_0^\perp$. Then
    \[
      x\in S_0 \quad\Leftrightarrow\quad
      \begin{bmatrix}
        b_1^T \\ \vdots \\ b_{n-2}^T 
      \end{bmatrix}x = Bx = 0.
    \]
    For $y=y_1a_1+y_2a_2\in S_0$, $y_1\le 1$ iff $a_1^Ty \le 1$ as $(a_1,a_2)$
    is an orthonormal basis of $S_0$. Hence, 
    \[
      -1\le y_1,y_2\le 1 \quad\Leftrightarrow\quad
      \begin{bmatrix}
        a_1^T \\ a_2^T \\ -a_1^T \\ -a_2^T
      \end{bmatrix}y = Ay \preceq \mathbf{1}.
    \]
    Thus, for orthonormal $a_1$ and $a_2$, $S=\{x:\,Bx=0,\,Ax\preceq\mathbf{1}\}
    $, a polyhedron.\par
    Now we only assume the liner independence of $a_1$ and $a_2$. We know that
    there exists some invertible $n$-by-$n$ matrix\footnote{We can use $QR$ 
    factorization to construct the matrix explicitly} $R$ such that $[
    \tilde{a}_1,\tilde{a_2}]=R[a_1,a_2]$ and $\tilde{a}_1$ and $\tilde{a}_2$ are
    orthonormal. Denoting the set described in the problem with respect to $u_1$
    and $u_2$ by $S(u_1,u_2)$, $x\in S(a_1,a_2)$ iff $Rx\in S(\tilde{a}_1,
    \tilde{a_2})$ iff $Rx \in \{x:\,\tilde{B}x=0,\, \tilde{A}x\preceq 1\}$ where
    the meaning of $\tilde{A}$ and $\tilde{B}$ are described in the previous
    passage. Hence, 
    \[
      S(a_1,a_2) = \{x:\, \tilde{B}Rx=0,\, \tilde{A}Rx\preceq 1\}.
    \]
    
    (b) Yes, and the provided form has already satisfied the requirement.\par
    (c) No. Note that $\langle x,y\rangle_2\le1$ for all $y$ with $2$-norm $1$ 
    implies
    \[
      \|x\|_2 = \langle x,x/\|x\|\rangle_2 \le 1.
    \]
    And by the Cauchy-Schwarz inequality, for every $\|x\|\le 1$, $\langle x,y
    \rangle_2$ holds for every $\|y\|_2=1$. Hence, $S$ is the intersection of
    the unit ball and $\{x:\,x\succeq 0\}$, which is not a polyhedron.\par
    (d) Yes. Let $\tilde{S}=\{x\in\mathbb{R}^n:\,x\succeq 0,\,\|x\|_\infty\le 1
    \}$, which is clearly a polyhedron since when $x\succeq 0$, $\|x\|_\infty\le
    1$ is equivalent to $[e_1,\dots,e_n]x\preceq \mathbf{1}$ where $e_i$ is the 
    $i$-th vector in the standard basis of $\mathbb{R}^n$.\par
    Now we show that $S=\tilde{S}$. Suppose that $x\succeq 0$. If $\langle x,y
    \rangle_2\le 1$ for all $y$ with $1$-norm $1$, then $x_i = \langle x,e_i
    \rangle_2 \le 1$. Namely, $\|x\|_\infty\le 1$. Meanwhile, if $\|x\|_\infty
    \le 1$, 
    \[
      \langle x,y\rangle \le \sum_{i=1}^n x_i|y_i| \le 1
    \]
    as it is just the weighted average of $x_1,\dots,x_n$. Hence, $S=\tilde{S}$,
    completing the proof.
  \end{proof}

  \paragraph{2.9}
  \begin{proof}
    $\,$\par
    (a) By the definition, 
    \begin{align*}
      x\in V &\Leftrightarrow \|x-x_0\|_2^2-\|x-x_i\|_2^2 \le 0  \\
      &\Leftrightarrow 2\langle x,x_i-x_0\rangle 
        \le \langle x_i,x_i\rangle - \langle x_0,x_0\rangle
        \quad\text{for $i=1,\dots,K$}\\
      &\Leftrightarrow
        2\begin{bmatrix}
          \langle x,x_1-x_0 \rangle \\ \vdots \\ \langle x,x_K-x_0 \rangle
        \end{bmatrix}
        \preceq\begin{bmatrix}
          \|x_1\|_2^2-\|x_0\|_2^2 \\ \vdots \\ \|x_K\|_2^2-\|x_0\|_2^2
        \end{bmatrix} \\
      &\Leftrightarrow
      2\begin{bmatrix}
        (x_1-x_0)^T \\ \vdots \\ (x_K-x_0)^T
      \end{bmatrix}x
      \preceq\begin{bmatrix}
        \|x_1\|_2^2-\|x_0\|_2^2 \\ \vdots \\ \|x_K\|_2^2-\|x_0\|_2^2
      \end{bmatrix}
    \end{align*}
    Hence, $V$ is a polyhedron. Intuitively, the border of a Voronoi set are the 
    lines with the same distances to $x_0$ and $x_i$. \par
    (b) Suppose that $P=\{x:\,\alpha_k^Tx\le b_k, k=1,\dots,K\}$. Let $x_0$ be
    any point of $P$ and we construct the other points by reflection. For each 
    $k$, let $\tilde{x}_k$ be any point of $\{x:\,\alpha_k^Tx=b_k\}$, $U_k=I-2
    \alpha_k\alpha_k^T/\|\alpha_k\|^2_2$, the Householder matrix, and 
    \[
      R_k(x) = U_k(x-\tilde{x}_k)+\tilde{x}_k = 
      x + 2\frac{\alpha_k}{\|\alpha_k\|_2^2}(b_k-\alpha_k^Tx).
    \]
    It is easy to verified that $P$ is the Voronoi region of $x_0$ with respect
    to $R_1(x_0),\dots,R_K(x_0)$.\par
  \end{proof}

  \paragraph{10.}
  \begin{proof}
    $\,$\par
    (a) Suppose $x_1,x_2\in C$ and $\theta\in(0,1)$. Let $x=\theta x_1+(1-
    \theta)x_2$. Since $A$ is symmetric, $x_2^TAx_1 = x_1^TAx_2$. Thus,
    \begin{align*}
      f(x)
      &= x^TAx+b^Tx+c \\
      &=\theta^2 x_1^TAx_1+2\theta(1-\theta)x_1^TAx_2 + (1-\theta)^2x_2^TAx_2\\
      &\quad+\theta b^Tx_1 + (1-\theta)b^Tx_2 + \theta c + (1-\theta)c.
    \end{align*}
    Note that
    \begin{align*}
      \theta^2x_1^TAx_1+\theta b_1^Tx_1 + \theta c 
      &= \theta(x_1^TAx_1 + b_1^Tx_1+c) - \theta(1-\theta)x_1^TAx_1\\
      &\le -\theta(1-\theta)x_1^TAx_1
    \end{align*}
    and we can get a similar inequality for $x_2$. Hence,
    \begin{align*}
      f(x) &\le -\theta(1-\theta)(x_1^TAx_1 - 2x_1^TAx_2 + x_2^TAx_2) \\
      &= -\theta(1-\theta)(x_1-x_2)^TA(x_1-x_2) \le 0
    \end{align*}
    as $A\succeq 0$. Hence, $C$ is convex.\par
    (b) Put $H=\{x:\, g^Tx+h=0\}$, $B=A+\lambda gg^T$ and 
    \[
      C_B=\{x\in \mathbb{R}^n:\, x^TBx+b^Tx+c-\lambda h^2\le 0\}.
    \]
    By (a), $C_B$ is convex and so does $C_B\cap H$. Suppose $x\in H$, then 
    $x^TBx=x^TAx +\lambda h^2$. Therefore, $C_B\cap H=C$. Thus, $C$ is convex.
    
  \end{proof}
% end


\iffalse





\paragraph{2.16}
\begin{proof}
  For every $(a,b_1+b_2),(c,d_1+d_2)\in S$ and $0\le\theta\le 1$, let 
  \[
    z_\theta = \theta(a,b_1+b_2)+(1-\theta)(c,d_1+d_2) = (x,y_1+y_2)
  \]
  where
  \begin{align*}
    x = \theta a+(1-\theta) c,\quad
    y_i = \theta b_i+(1-\theta)d_i\quad\text{for $i=1,2$}.
  \end{align*}
  Since $S_i$ is convex and $(a,b_i),(c,d_i)\in S_i$,
  \[
    (x,y_i)=\theta(a,b_i) + (1-\theta)(c,d_i)\in S_i.
  \]
  Hence, $S$ is convex.
\end{proof}

\fi

% end