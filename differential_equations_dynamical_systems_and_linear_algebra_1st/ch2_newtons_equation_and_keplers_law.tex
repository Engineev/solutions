\section{Newton's Equation and Kepler's Law}

\paragraph{2.}
\begin{solution}
  $\,$\\
  (a) Yes. $V(x, y)=(x^3+2y^3)/3$. \\
  (b) No. If there exists some $V:\mathbb{R}^2\to\mathbb{R}$ such that $F = 
      -\nabla V$, then from 
      \[
        \frac{\partial V}{\partial x} = x^2-y^2
        \quad\text{and}\quad
        \frac{\partial V}{\partial y} = 2xy 
      \]
      we can respectively derive
      \[
        V(x,y)=x^3/3-xy^2+C_1 \quad\text{and}\quad V(x,y)=xy^2 + C_2,
      \]
      which is impossible.\\
  (c) Yes. $V(x, y)=x^2/2$.
\end{solution}

\paragraph{4.}
  I think that there should be a minus before the right hand side of the 
  definition equation of work. And I assume that $F$ is at least continuous.
\begin{proof}
  Suppose that $F$ is conservative and let $y(s)$ be any path from $x_0$ to 
  $x_1$. Note that 
  \[
    \frac{\rd}{\rd s}V(y(s)) = \langle \nabla_yV(y(s), y\hp(s)) \rangle.
  \]
  Hence,
  \[
    -\int_{s_0}^{s_1}\langle F(y(s)), y\hp(s)\rangle \rd s = 
    \int_{s_0}^{s_1}\langle \nabla_yV(y(s)), y\hp(s)\rangle \rd s =
    V(x_1)-V(x_0).
  \]
  Since the choice of $y(s)$ is arbitrary, the work is independent of the path.
  \par
  Now we suppose the work is independent of the path. Let $x_0$ be a fixed point
  and 
  \[
    V(x) = -\int_{s_0}^{s}\langle F(y(s)), y\hp(s)\rangle \rd s.
  \]
  where $y(s)$ is any path connecting $x_0 = y(s_0)$ and $x = y(s)$. And by the
  fundamental theorem of calculus, $V$ is continuously differentiable and $F=
  -\nabla V$. Hence, $F$ is conservative.
\end{proof}

\paragraph{6.}
\begin{proof}
  Let the notations have the same meanings. Note that $i=(\cos\theta,\sin
  \theta)$ and $j=(-\sin\theta,\cos\theta)$. Hence,
  \[
    \frac{\rd i}{\rd t} = (-\theta\hp\sin\theta, \theta\hp\cos\theta) 
    = \theta\hp j
    \quad\text{and}\quad
    \frac{\rd j}{\rd t} = -\theta\hp i.
  \]
  Therefore,
  \[
    \frac{\rd x}{\rd t} = \frac{\rd}{\rd t}(ri) = r\hp i + r i\hp
    = r\hp i + r\theta\hp j.  
  \]
  Differentiate again and we get
  \begin{align*}
    \frac{\rd^2 x}{\rd x^2} 
    &= r^{\prime\prime}i + r\hp i\hp + (r\theta\hp)\hp j + r\theta\hp j\hp \\
    &= r^{\prime\prime}i+r\hp\theta\hp j + (r\theta\hp)\hp j-r(\theta\hp)^2i\\
    &= (r^{\prime\prime}-r(\theta\hp)^2)i + (r\hp\theta\hp+(r\theta\hp)\hp)j \\
    &= (r^{\prime\prime}-r(\theta\hp)^2)i + \frac{1}{r}(r^2\theta\hp)\hp j.
  \end{align*}
  Since $F$ is central, $\langle F, j\rangle=0$ and so does $x^{\prime\prime}$.
  Thus, the coefficient of $j$ is zero, completing the proof.
\end{proof}
